\chapter{ASN.1 Primer}
\label{asn1-primer}

\define{ASN.1} (Abstract Syntax Notation number One) is a formalism
for the specification of abstract data types. Such a specification is
made independently of programming language or wire transmission
format.

From such an abstract specification, automated encoder/decoder in any
programming languages can be generated, for various encoding formats
like strings of bytes (BER, CER and DER encoding), strings of bits
(PER encoding) or XML (BASIC-XER, CXER, EXTENDED-XER encoding).

ASN.1 is an international standard freely available
(\cite{asn1-standard}).

Example of encoder/decoder generator for ASN.1:
\begin{itemize}
\item asn1c: open source ASN.1 compiler
  \url{http://lionet.info/asn1c/compiler.html}
\item ASN1SCC compiler: open source ASN.1 compiler that targets C and
  Ada, while placing specific emphasis on embedded systems
  \url{https://github.com/ttsiodras/asn1scc}
\item Other ASN.1 tools can be found at
  \url{http://www.itu.int/ITU-T/studygroups/com17/languages/index.html}
\end{itemize}

\section{Overview of ASN.1 syntax}

We describe in this appendix a very basic knowledge of ASN.1 aiming at
understanding the ASN.1 definitions of this document.

The basic data types of ASN.1 are \ASN{INTEGER}, \ASN{REAL}
(mathematical real values following formula $m \times b^e$ with $m$
mantissa, $b$ 2 or 10 base and $e$ exponent), \ASN{BOOLEAN},
\ASN{NULL} (no type) and various types of strings\footnote{We ignore
  data types not used in this document like \ASN{OCTET STRING},
  \ASN{BIT STRING}, etc.}. $a$\ASN{..}$b$ denotes are range of values
from $a$ to $b$.

The basic structures of ASN.1 are as followed and can be composed at
will:
\begin{itemize}
\item \index{ENUMERATED@\ASN{ENUMERATED}}\ASN{ENUMERATED \{ \}}: an
  enumeration of fixed values;
\item \index{SEQUENCE@\ASN{SEQUENCE}}\ASN{SEQUENCE \{ \}}: a record of
  elements;
\item \index{SEQUENCE OF@\ASN{SEQUENCE OF}}\ASN{SEQUENCE OF}: a list
  of element;
\item \index{SEQUENCE (SIZE(a..b)) OF@\ASN{SEQUENCE
    (SIZE(}$a$\ASN{..}$b$\ASN{)) OF}}\ASN{SEQUENCE
  (SIZE(}$a$\ASN{..}$b$\ASN{)) OF}: an array of elements with indexes
  from $a$ to $b$;
\item \index{CHOICE@\ASN{CHOICE}}\ASN{CHOICE \{ \}}: a choice between
  several elements (i.e. alternative, like a union in C or a variant
  record in Ada).
\end{itemize}

A data type definition has form ``\emph{name} \ASN{::=}
\emph{definition}''. Names can use characters \ASN{A-Z a-z 0-9 -}.

ASN.1 is case sensitive. ASN.1 keywords are uppercase. Names starting
with a capital letter are reserved for type definitions while names
starting with a lower case later are used for field identifier or
enumeration values.

Definitions are grouped in modules grouping
\index{DEFINITIONS@\ASN{DEFINITIONS}}\ASN{DEFINITIONS}:
\begin{lstlisting}[language=ASN1]
The-module-name DEFINITIONS ::=
BEGIN
  -- some definitions
END
\end{lstlisting}

Comments start with \ASN{--} until end of line or next \ASN{--}.

\section{ASN.1 examples}

We now show some examples (all validated with asn1c tool).

A record of 2 elements, one Boolean and one integer:
\begin{lstlisting}[language=ASN1]
Record-2-elements ::= SEQUENCE {
  a-boolean  BOOLEAN,
  an-integer INTEGER
}
\end{lstlisting}

A choice between nothing or an array of 100 integer elements:
\begin{lstlisting}[language=ASN1]
Choice-between-2-elements ::= CHOICE {
  nothing     NULL,
  an-array    SEQUENCE (SIZE (1..100)) OF INTEGER
}
\end{lstlisting}

A list of which elements are made of three possible values (value1,
value2 or value3):
\begin{lstlisting}[language=ASN1]
Possible-value ::= ENUMERATED {
  value1, value2, value3
}

List-of-possible-values ::= SEQUENCE OF Possible-value
\end{lstlisting}

A value that can be either an integer between 0 and 1023 or a real
between 0 and 10 with a 0.1 increment ($0 \times 10^{-1}$ to $100
\times 10^{-1}$):
\begin{lstlisting}[language=ASN1]
A-value ::= CHOICE {
  a-limited-integer     INTEGER(0..1023),
  a-limited-real        REAL (WITH COMPONENTS { mantissa (0..100),
                                                base (10),
                                                exponent (-1) })
}
\end{lstlisting}


% LocalWords:  ASN lstlisting CER XER CXER
