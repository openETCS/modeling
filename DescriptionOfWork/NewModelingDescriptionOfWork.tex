\documentclass{template/openetcs_article}
% Use the option "nocc" if the document is not licensed under Creative Commons
%\documentclass[nocc]{template/openetcs_article} 
\usepackage{lipsum,url}
\usepackage{xspace}
\usepackage{graphicx}
\usepackage{fixme}
\usepackage{lscape} 
\usepackage{pgfgantt}
\usepackage{adjustbox}
\usepackage{datetime}



%user specified macros
%\newenvironment{activity}[2][planned]
	{\begin{tabular}{p{0.25\textwidth}@{\hspace{0.05\textwidth}}p{0.7\textwidth}}
			\multicolumn{2}{p{\textwidth}}{\colorbox{black}{\begin{minipage}{1.1cm}\begin{center}\textsc{\footnotesize \textcolor{white}{#1}}\end{center}\end{minipage}}~~\textbf{#2}}\\
	}
	{\end{tabular}}

\newcommand{\entry}[2]{#1:&#2\\}
\newcommand{\website}[1]{Website:&\url{#1}\\}
\newcommand{\desc}[1]{\multicolumn{2}{p{\textwidth}}{#1}\\}

\newcommand{\VV}{Verification \& Validation\xspace}
\newcommand{\vv}{verification \& validation\xspace}

\newcommand{\tbd}{\colorbox{cyan}{\%\%To Be Defined\%\%}}
\newcommand{\tbc}{\colorbox{cyan}{\%\%To Be Confirmed\%\%}}
\newcommand{\todo}[1]{\colorbox{cyan}{\%\%{#1}\%\%}}
\newcommand{\nthng}[1]{}


\graphicspath{{./template/}{.}{./images/}}
\begin{document}
\frontmatter
\project{openETCS}

%Please do not change anything above this line
%============================
% The document metadata is defined below

%assign a report number here
\reportnum{OETCS/WP3/DescriptionOfWork}

%define your workpackage here
\wp{Work Package 3: ``Modelling''}

%set a title here
\title{openETCS Modelling Work Package}

%set a subtitle here
\subtitle{Description of Work}

%set the date of the report here
\date{January 2014}

%define a list of authors and their affiliation here

\author{openETCS WP3 SRS task force}

\affiliation{The Team}
 
%\author{Jan Welte}

%\affiliation{WP3 Safety and Requirements Traceability}

\author{Marielle Petit-Doche}

\affiliation{Methodology}

\author{Uwe Steinke}

\affiliation{Objects)}

\author{Bernd Hekele}

\affiliation{SRS Findings)}

% define the coverart
\coverart[width=350pt]{openETCS_EUPL}

%define the type of report
\reporttype{Description of work}



\begin{abstract}
%define an abstract here

  This work package...



\end{abstract}

%=============================
%Do not change the next three lines
\maketitle
\tableofcontents
\listoffiguresandtables
\newpage
%=============================

% The actual document starts below this line
%=============================


%Start here



%-----------------------------------------------------------------------
\section{Introduction}
%-----------------------------------------------------------------------

%set the master document for easy compilation
%!TEX root = ../D3_5_3.tex

\chapter{Purpose of the document}

This document is managed as a deliverable of the modeling work package with denomination ~D3.7.x, and contains advices and recommendation for the design of a physical system architecture.  

The development of the functional model is done iteratively increasing the scope in steps, the last digit of the deliverable identifier, i.e.~x, denotes the release of the model to which it applies. If the functional model requires to update the system architecture a consistent version number will be applied to this document as required by the Model release version.

This document complements the indications contained in the API requirements specification and the documentation derived from this as the generic openETCS Application Programming Interface (API), available at \url{https://github.com/openETCS/modeling/blob/master/API/description/api-description.pdf}. \cite{alstom-api}

\section{Input Documents}

The following documents provide a context for the system perspective.

\begin{itemize}
	\item ERA Subset-026 \cite{subset-026}, V3.3.0
	\item ERA TSI CCS Documents
	\item openETCS API documentation, available at \url{https://github.com/openETCS/modeling/blob/master/API/description/api-description.pdf} \cite{alstom-api}\cite{alstom-api-app-layer}\cite{alstom-api-data-dict}
	%\item openETCS requirements, i.e.~D2.1, D2.2,$\ldots$, %D2.9, available at %\url{https://github.com/openETCS/requirements/tree/maste%r/Reference}
\end{itemize}




\chapter{Introduction}

Designing a sub system integrable with the train borne system is a complex task. The designer faces a large variety of serious challenges and design complexities. 

Before the functions are actually implemented, a system architect will have to select an appropriate hardware-software concept out of the large number of available boards, controllers, network and  bus constraints.  He  will as well include robustness criteria against environmental influences. 

Memories, operating systems, drivers, generic and application software segregation as well as selection criteria for sensors and actuators need to be correctly assessed. 

The target architecture has to meet a large variety of requirements. Criteria of timing, Bus bandwidth, processor and peripheral performance, memory size, safety principles and possible processing or data transfer bottlenecks. Environmental conditions, timing constraints, robustness against specific interferences shall constantly be tracked.

Power requirement as well as allocation of availability, maintainability figures to enumerate only the most relevant items accompany all the design phases.

On top of this a specific vital architecture has to be selected and the required integrity level has to be granted. The relative safety constraints have to be assured and maybe exported.

Selecting the components matching these is a critical phase. Over-dimensioning the architecture may impact on cost factors relevant for the market access of the system. Under-dimensioning the architecture design could result in not achieving performance constraints, thus compromising system quality and suitability. Early architectural choices have a dominant impact on the success of the new system. 

The system architects will commit to efficient design choices according to the target project margins and all this within the frame of a defined project delivery time schedule. 

Due to the fact that the design verification phase, requiring to have completed all the integration steps, may be very late in the release process a high precision during the system architecture design is mandatory. 

Therefore highly experienced System Designer are considered as the key factor for a reliable achievement of expected design result.

A primary goal of the openETCS ITEA2 project is to provide a formal specification and a model of an ETCS onboard functionality according to the specification defined in Subset-026 \cite{subset-026} by the European Railway Agency (ERA). 

The Model-Based Development process is an approach that allows engineers to specify the behavior of a system and to simulate and execute it in a very early development stage.

Once a model-based development process has been established, engineers should be able to apply new technologies and tools to enhance and shorten product development cycles,
e.g. by introducing generation of Model Validation test cases and target Code directly from the model. This enables to improve the V based development process to save development time and effort while preserving or improving the dependability of the developed systems. 


\begin{figure}[H]
	\center
	\includegraphics[width= 0.9\textwidth]{Y_process.pdf}
	\caption{SRS modeling cycle}\label{Y_process}
\end{figure}


The methodology makes it easier to understand requirements and increases the correctness of the requirements, the correctness of the design and the code with respect to the requirements. An integration of system-level and design-level modeling tools allows a virtually integrated V-process that is sharpened up to a Y-based process with the required steps at the bottom of the former V being considerably automated (see figure\ref{Y_process} )

Nevertheless when specifying the overall software architecture, the designer should be aware of the implications of software design decisions on the target end system.

\section{Safety Integrity and Functional Safety according CENELEC}
The Railway Industry currently relies on the international standard group of coordinated standards: EN 50126 “Railway
applications – The specification and
demonstration of Reliability, Availability,
Maintainability and Safety (RAMS)” 
the EN 50129 “Railway applications – Safety
related electronic systems for signalling” and
the EN 50128 “Railway applications -  Communications, signalling and processing systems – Software for railway control and  protection systems” to provide a rational and consistent approach for the development of safety-related systems.

This group of standards owes much of its direction and contents to the IEC 61508 standard that is a generic safety standard for electrical/electronic/programmable electronics safety-related systems.

Both of these IEC and EN standards share the same philosophy in the sense that they:

\begin{itemize}
\item consider all relevant product and software safety life-cycle phases, from an initial concept phase to maintenance and decommissioning when these systems are used to perform safety functions;
\item intend to shape a safety awareness; \item have been conceived with a rapidly developing technology in mind;
\item provide methods and rules for defining safety requirements necessary to achieve defined functional safety.
\item use Safety Integrity Levels (SIL) for specifying the target level of safety integrity for the safety functions to be implemented.
\item adopt a statistical risk-based approach for the determination of the SIL requirements;
\item distinguish between safe and unsafe failure modes and requires precautions against undetected failures. 
\end{itemize}

According the Cenelec norms the product is subject to a certification process. The definition of the equipment under control (EUC) depends on the scope of the certification. It can be, for example the complete ERTMS/ETCS subsystem or a  module of it.

The term safety-related is used to describe systems that are required to perform a specific function to ensure that risks are kept at an acceptable level. Such functions are, by definition, safety functions. Two types of requirements are necessary to achieve functional safety: 

\begin{itemize}
\item Safety function requirements (what the function does),
\item Safety integrity requirements (the required likelihood of a safety function being performed satisfactorily).
\end{itemize}

The safety function requirements are derived from a risk analysis phase, in the scope of EN 50126, where significant risks for equipment and any associated control system in its intended environment have to be identified. This analysis determines whether functional safety is necessary to ensure adequate protection from unacceptable risks. Functional safety is therefore
a method of dealing with risks to eliminate them or reduce them to an acceptable level. EN 50128 specifies four levels of safety
performance for a safety function. These are called Software Safety Integrity Levels (SwSIL).

\section{Reference to the openETCS functional Model}
The openETCS OBU partial model has been developed according to the specification given in ERA Subset-026 \cite{subset-026}, Version 3.3.0. The software release is publicly available on a repository at 
\begin{quotation}
\centering
\url{https://github.com/openETCS/modeling/tree/v0.3-D3.6.3}
\end{quotation}




%-----------------------------------------------------------------------
\section{Goals of the openETCS Modelling Work}
%-----------------------------------------------------------------------
%\tbc
by Uwe

\subsection{Organisation of the Work package}

Due to historical reasons we have some projects linked to the openETCS modeling activities.
In the progress of our task we will make a clean-up on the history. Here is an overview on the different repositories with reasonable content.

\begin{itemize}
\item \url{https://github.com/openETCS/modeling}\\
The modeling repository is the place where modeling takes place. openETCS ToolChain is linked to this repository (folder "model". All code artefacts have to reside in this location. 


\item \url{https://github.com/openETCS/SRS-Analysis}\\
The SRS-Analysis Repository was originally introduced to support the task force led by Alstom and intended to start with the analysis work.
In this location you can find traces of the analysis work. The repository is still in use.

\item \url{https://github.com/openETCS/dataDictionary}\\
The data-Dictionary repository is related to the definition of common data and functions. The dataDictionry implementation is based on SysML/Papyrus. Therefore, not tools nor tool artefacts are stored here. Only documentation is needed. The repository is still in use.

\item \url{https://github.com/openETCS/SSRS}\\
The SSRS activities had been used before the modelling kick-off to prepare for the analysis work. There are stil some useful documents included. Especially the wiki is usefull.
However, the place is no longer used for active work and the repository can - after having moved valid information to other locations - be considered to be recycled.

\item Requirements on Process and Methods, Guidelines\\
The following guidelines are the basis for modeling work:
The openETCS development process: \url{https://github.com/openETCS/requirements/blob/master/D2.3/D2_3.pdf}\\
The openETCS requirements on methods: \url{https://github.com/openETCS/requirements/blob/master/D2.4/D2_4.pdf} 
This document also describes naming conventions which you need to respect for working in this task.

\end{itemize}


%-----------------------------------------------------------------------
\section{Methodology}
%-----------------------------------------------------------------------
%\tbc


This section gives some information on how to proceed to achieve the  objectives described in section \ref{sec:Objectives}.
As decided previously in the project, the means chosen to design the WP3 software, compliant to SIL4 requirements of EN50128 are SysML on Papyrus and SCADE for the modelling part, C language for the executive software. The first versions of the OpenETCS toolchain already involve Papyrus.

Other tools can be involved to support the task of the workpackage (indeed to manage requirements, data, traceability,...) but will be defined latter depending of the needs and the propositions.

For further information on means selection and detailed description of the process, consider the deliverables [D7.1]\footnote{\url{https://github.com/openETCS/toolchain/blob/master/T7.1/D7.1/D7.1.pdf}}, [D7.2]\footnote{\url{https://github.com/openETCS/toolchain/blob/master/T7.2/D7.2/D7_2.pdf}} and [D2.4]\footnote{\url{https://github.com/openETCS/requirements/blob/master/D2.4/D2_4.pdf}}.

\subsection{Main step of the analysis}

The figure \ref{fig:Steps} gives the main steps of the design approach and main used and produced artefacts. All is detailed in D2.4.

\begin{figure}[ht]
  \centering
  \includegraphics[width=\textwidth]{sections/Step.png}
  \caption{Main steps of the design approach}
  \label{fig:Steps}
\end{figure}


Four main steps are defined:
\begin{description}
\item [System Analysis] to identify the main functions and their interactions and to clarify the requirements allocated to them.
\item [Architectural Modelling] to specify the functional architecture of the system to design and internal and external interfaces. In parallel  the main data exchanged between the functions shall be identified.
\item [Functional and Behavioural Modelling] to provide a formal model of each function.
\item [Executable Software Design] to provide executable part of software.
\end{description}

These steps are sharing common artifacts which going to be updated and completed during all the design phase. Some recommendations and guidelines are defined in D2.4 for naming of elements or specifing a model.

As the artifacts are shared, an iterative process can be easily applied at each level.

\subsection{Shared artifacts}

\subsubsection{Functional architecture and API}

API\footnote{\url{https://github.com/openETCS/requirements/blob/master/D2.7-Technical_Appendix/OETCS_API\ Requirements_v1.0Draft_130301.pdf}} is a document provided as input which can be updated according to the needs.

Functional architecture is specified as a SysML  model, which can be automatically translated in a Scade model. An initial SysML model to cover the initial functional scope \ref{sec:FunctionalScopeTheMinimumOBUKernelFunction} is available on github: \url{https://github.com/openETCS/modeling/tree/master/model/sysml}

See D2.4 for how to use SysML to define the Functional Architecture.

\subsubsection{Set of Requirements}

The initial  set of requirements is the contains of Subset\_026 v3.3.0. These requirements are available as ReqIF format on github: \url{https://github.com/openETCS/modeling/tree/master/model/subset26}
.

This set is going to be increased during the design with updated requirements and added requirements.

Data model of D2.4 gives the information to manage on the requirements.

For the moment, ProR is involved in the openETCS toolchain to define new requirements and links between them.
It is still to clarify how to manage traceability.



\subsubsection{Data Dictionary}

The Data Dictionary is the set of all the types, constants and variables defined to  describe the system.
Data model of D2.4 describes how to define a data.

A preliminary task of the system analysis is to identified and specify the data structure which allow to describe the system.

Then the data structure and data definition shall be implemented in the data dictionary.

For the moment the mean to do this implementation is not clearly identified (UML library ? XML files ?)


\subsubsection{Formal models}

Two  models are provided during this phase:
\begin{itemize}
\item  a semi-formal model in SysML which gives the functional architecture of the system, including interaction between each function  and definition of the data.
\item a formal model in SCADE which follows and completes the same architecture and gives a behavioural description of each function. Then C code can be automatically  generated from this model.
\end{itemize}

\subsection{Function specification and design}


For the specification and the design of a functional block, a set of subtasks can be defined:

\begin{enumerate}
\item Identify the function and the input document to describe it (requirements of subset 26 or other subsets, API,, functional architecture,...)
\item Specify its environment and its external interfaces with an ibd diagram in SysML
\item Define and specify  its internal decomposition in subfunctions and its internal  interfaces with ibd and bdd diagrams in SysML
\item Link and complete to the data dictionary
\item Allocate and manage the requirements
\item Clarify and specify the behavior of each elementary function in SCADE
\item Complete Data dictionary if necessary
\item Complete requirement sets and manage traceability
\item Provide for review
\end{enumerate}

\section{Verification and Validation of Modeling artefacts}
The handling is based on the openETCS review process which is documented as an extension of the quality insurance plan. \url{https://github.com/openETCS/governance/tree/master/Review%20Process}.
In short, this is done in posting an issue with the issue tracker of the modeling repository. 
Before passing the artefact to WP4, the WP3 product owner has to agree to the triggering of verification and validation.

%-----------------------------------------------------------------------
\section{How to Handle Findings in the SRS}
%-----------------------------------------------------------------------
%\tbc
by Bernd


\nocite{*}
%===================================================
%Do NOT change anything below this line

\end{document}
