\documentclass{template/openetcs_report}
% Use the option "nocc" if the document is not licensed under Creative Commons
%\documentclass[nocc]{template/openetcs_article}
\usepackage{lipsum,url}
\usepackage{supertabular}
\usepackage{multirow}
\usepackage{color, colortbl}
\usepackage{hyperref}
%\usepackage{listings}
\usepackage{makeidx}
\definecolor{gray}{rgb}{0.8,0.8,0.8}
\usepackage[modulo]{lineno}
\graphicspath{{./template/}{.}{./images/}}

\newcommand{\define}[1]{\index{#1}\emph{#1}}

\begin{document}
\frontmatter
\project{openETCS}

%Please do not change anything above this line
%============================

%user specified macros
%\newenvironment{activity}[2][planned]
	{\begin{tabular}{p{0.25\textwidth}@{\hspace{0.05\textwidth}}p{0.7\textwidth}}
			\multicolumn{2}{p{\textwidth}}{\colorbox{black}{\begin{minipage}{1.1cm}\begin{center}\textsc{\footnotesize \textcolor{white}{#1}}\end{center}\end{minipage}}~~\textbf{#2}}\\
	}
	{\end{tabular}}

\newcommand{\entry}[2]{#1:&#2\\}
\newcommand{\website}[1]{Website:&\url{#1}\\}
\newcommand{\desc}[1]{\multicolumn{2}{p{\textwidth}}{#1}\\}

\newcommand{\VV}{Verification \& Validation\xspace}
\newcommand{\vv}{verification \& validation\xspace}

\newcommand{\tbd}{\colorbox{cyan}{\%\%To Be Defined\%\%}}
\newcommand{\tbc}{\colorbox{cyan}{\%\%To Be Confirmed\%\%}}
\newcommand{\todo}[1]{\colorbox{cyan}{\%\%{#1}\%\%}}
\newcommand{\nthng}[1]{}

% The document metadata is defined below

%assign a report number here
\reportnum{OETCS/WP3/D3.5.1.2}

%define your workpackage here
\wp{Work-Package 3: ``Modeling''}

%set a title here
\title{openETCS System Architecture and Design Specification}

%set a subtitle here
\subtitle{Second Iteration: Scope of openETCS ITEA2 Functions}

%set the date of the report here
\date{October 2014}


%document approval
%define the name and affiliation of the people involved in the documents approbation here
\creatorname{Baseliyos Jacob}
\creatoraffil{DB Netz}

\techassessorname{[assessor name]}
\techassessoraffil{[affiliation]}

\qualityassessorname{Izaskun de la Torre}
\qualityassessoraffil{SQS}

\approvalname{Klaus-R\"udiger Hase}
\approvalaffil{DB Netz}


%define a list of authors and their affiliation here

\author{Baseliyos Jacob, Bernd Hekele}

\affiliation{DB Netz AG}

\author{xxx}

\affiliation{DB Netz AG\\
  V\"olckerstrasse 5\\
  D-80959 M\"unchen Freimann, Germany}

\author{Marc Behrens}
\affiliation{DLR}

\author{David Mentre}
\affiliation{Mitsubishi Electric R\&D Centre Europe}

\author{Jos Holtzer, Jan Welvaarts, Vincent Nuhaan}
\affiliation{NS}

\author{Jacob G\"artner}
\affiliation{LEA Engineering}

% define the coverart
\coverart[width=350pt]{openETCS_EUPL}

%define the type of report
\reporttype{Architecture and Functional Specification}


\begin{abstract}
%define an abstract here
This document gives an introduction to the architecture of openETCS. The functional scope is tailored to cover the functionality required for the openETCS demonstration as a target of the ITEA2 project: the Utrecht Amsterdam use-case. It has to be read as an add-on to the models in SysML, Scade and to additional reading referenced from the document.
\end{abstract}

%=============================
\maketitle

%Modification history
%if you do not need a modification history table for your document simply comment out the eight lines below
%=============================


\chapter*{Modification History}
\tablefirsthead{
\hline 
\rowcolor{gray} 
Version & Section & Modification / Description & Author \\\hline}
\begin{supertabular}{| m{1.2cm} | m{1.5cm} | m{6.6cm} | m{3.7cm} |}
0.1 & Document & Initial document providing the structure & Baseliyos Jacob \\\hline
0.2 & Document & Workshop Results included and some pretty-printing & Bernd Hekele \\\hline

\end{supertabular}

% list subsubsections in table of contents
\setcounter{tocdepth}{3}


\tableofcontents
\listoffiguresandtables
\newpage
%=============================

%Uncomment the next line if you need line numbers for tracebility when the document is in review
%\linenumbers
%=============================


% The actual document starts below this line
%=============================

\mainmatter

\chapter{Introduction}

%set the master document for easy compilation
%!TEX root = ../D3_5_3.tex

\chapter{Purpose of the document}

This document is managed as a deliverable of the modeling work package with denomination ~D3.7.x, and contains advices and recommendation for the design of a physical system architecture.  

The development of the functional model is done iteratively increasing the scope in steps, the last digit of the deliverable identifier, i.e.~x, denotes the release of the model to which it applies. If the functional model requires to update the system architecture a consistent version number will be applied to this document as required by the Model release version.

This document complements the indications contained in the API requirements specification and the documentation derived from this as the generic openETCS Application Programming Interface (API), available at \url{https://github.com/openETCS/modeling/blob/master/API/description/api-description.pdf}. \cite{alstom-api}

\section{Input Documents}

The following documents provide a context for the system perspective.

\begin{itemize}
	\item ERA Subset-026 \cite{subset-026}, V3.3.0
	\item ERA TSI CCS Documents
	\item openETCS API documentation, available at \url{https://github.com/openETCS/modeling/blob/master/API/description/api-description.pdf} \cite{alstom-api}\cite{alstom-api-app-layer}\cite{alstom-api-data-dict}
	%\item openETCS requirements, i.e.~D2.1, D2.2,$\ldots$, %D2.9, available at %\url{https://github.com/openETCS/requirements/tree/maste%r/Reference}
\end{itemize}




\chapter{Introduction}

Designing a sub system integrable with the train borne system is a complex task. The designer faces a large variety of serious challenges and design complexities. 

Before the functions are actually implemented, a system architect will have to select an appropriate hardware-software concept out of the large number of available boards, controllers, network and  bus constraints.  He  will as well include robustness criteria against environmental influences. 

Memories, operating systems, drivers, generic and application software segregation as well as selection criteria for sensors and actuators need to be correctly assessed. 

The target architecture has to meet a large variety of requirements. Criteria of timing, Bus bandwidth, processor and peripheral performance, memory size, safety principles and possible processing or data transfer bottlenecks. Environmental conditions, timing constraints, robustness against specific interferences shall constantly be tracked.

Power requirement as well as allocation of availability, maintainability figures to enumerate only the most relevant items accompany all the design phases.

On top of this a specific vital architecture has to be selected and the required integrity level has to be granted. The relative safety constraints have to be assured and maybe exported.

Selecting the components matching these is a critical phase. Over-dimensioning the architecture may impact on cost factors relevant for the market access of the system. Under-dimensioning the architecture design could result in not achieving performance constraints, thus compromising system quality and suitability. Early architectural choices have a dominant impact on the success of the new system. 

The system architects will commit to efficient design choices according to the target project margins and all this within the frame of a defined project delivery time schedule. 

Due to the fact that the design verification phase, requiring to have completed all the integration steps, may be very late in the release process a high precision during the system architecture design is mandatory. 

Therefore highly experienced System Designer are considered as the key factor for a reliable achievement of expected design result.

A primary goal of the openETCS ITEA2 project is to provide a formal specification and a model of an ETCS onboard functionality according to the specification defined in Subset-026 \cite{subset-026} by the European Railway Agency (ERA). 

The Model-Based Development process is an approach that allows engineers to specify the behavior of a system and to simulate and execute it in a very early development stage.

Once a model-based development process has been established, engineers should be able to apply new technologies and tools to enhance and shorten product development cycles,
e.g. by introducing generation of Model Validation test cases and target Code directly from the model. This enables to improve the V based development process to save development time and effort while preserving or improving the dependability of the developed systems. 


\begin{figure}[H]
	\center
	\includegraphics[width= 0.9\textwidth]{Y_process.pdf}
	\caption{SRS modeling cycle}\label{Y_process}
\end{figure}


The methodology makes it easier to understand requirements and increases the correctness of the requirements, the correctness of the design and the code with respect to the requirements. An integration of system-level and design-level modeling tools allows a virtually integrated V-process that is sharpened up to a Y-based process with the required steps at the bottom of the former V being considerably automated (see figure\ref{Y_process} )

Nevertheless when specifying the overall software architecture, the designer should be aware of the implications of software design decisions on the target end system.

\section{Safety Integrity and Functional Safety according CENELEC}
The Railway Industry currently relies on the international standard group of coordinated standards: EN 50126 “Railway
applications – The specification and
demonstration of Reliability, Availability,
Maintainability and Safety (RAMS)” 
the EN 50129 “Railway applications – Safety
related electronic systems for signalling” and
the EN 50128 “Railway applications -  Communications, signalling and processing systems – Software for railway control and  protection systems” to provide a rational and consistent approach for the development of safety-related systems.

This group of standards owes much of its direction and contents to the IEC 61508 standard that is a generic safety standard for electrical/electronic/programmable electronics safety-related systems.

Both of these IEC and EN standards share the same philosophy in the sense that they:

\begin{itemize}
\item consider all relevant product and software safety life-cycle phases, from an initial concept phase to maintenance and decommissioning when these systems are used to perform safety functions;
\item intend to shape a safety awareness; \item have been conceived with a rapidly developing technology in mind;
\item provide methods and rules for defining safety requirements necessary to achieve defined functional safety.
\item use Safety Integrity Levels (SIL) for specifying the target level of safety integrity for the safety functions to be implemented.
\item adopt a statistical risk-based approach for the determination of the SIL requirements;
\item distinguish between safe and unsafe failure modes and requires precautions against undetected failures. 
\end{itemize}

According the Cenelec norms the product is subject to a certification process. The definition of the equipment under control (EUC) depends on the scope of the certification. It can be, for example the complete ERTMS/ETCS subsystem or a  module of it.

The term safety-related is used to describe systems that are required to perform a specific function to ensure that risks are kept at an acceptable level. Such functions are, by definition, safety functions. Two types of requirements are necessary to achieve functional safety: 

\begin{itemize}
\item Safety function requirements (what the function does),
\item Safety integrity requirements (the required likelihood of a safety function being performed satisfactorily).
\end{itemize}

The safety function requirements are derived from a risk analysis phase, in the scope of EN 50126, where significant risks for equipment and any associated control system in its intended environment have to be identified. This analysis determines whether functional safety is necessary to ensure adequate protection from unacceptable risks. Functional safety is therefore
a method of dealing with risks to eliminate them or reduce them to an acceptable level. EN 50128 specifies four levels of safety
performance for a safety function. These are called Software Safety Integrity Levels (SwSIL).

\section{Reference to the openETCS functional Model}
The openETCS OBU partial model has been developed according to the specification given in ERA Subset-026 \cite{subset-026}, Version 3.3.0. The software release is publicly available on a repository at 
\begin{quotation}
\centering
\url{https://github.com/openETCS/modeling/tree/v0.3-D3.6.3}
\end{quotation}




\section{Assumption and Preconditions}
\begin{itemize}
\item All future contributions sahll be fully aligned an compliant with final documents
\item Alls documents produced by the partners with i.e. constrains with document; other contributions will be discarted
\end{itemize}

\section{Process}
\begin{itemize}
\item Alstom as WP 3 leader will be responsible for plannings
\item Time and quality aspects
\item openETCS tools and methodology
\end{itemize}


\section{Functions ERTMS/ETCS}


The ERTMS / ETCS system was developed with a view to interoperability of trains on the 
different European rail networks. It is divided into "tracks" - and "board" finishes 
and shall establish a mutual message operation, by beacons or through a "radio" - 
The transmission system (in this case a mobile telephone network GSM-R) is performed. 
It defines several operating levels, and the system must also interfaces with the 
existing monitoring systems of the trains (using STM) have. 
The ERTMS / ETCS system provides the transport operator (the track) the choice of conditions 
concerning the use and operation. 
The train must therefore may go with different operating conditions on routes. 
Thus has the onboard equipment but must be implemented, 
to the interoperability of the train to ensure on the other networks. 
These functions must therefore correspond to one standard: the SRS (version x.x.x). 

application functions, which have two different species of origin: 
defined in the SRS: here one finds in particular the 
speed monitoring- and transfer functions; these functions 
must be implemented in full accordance with the SRS; they can in 
indeed be on any network on which the train is used; these functions 
are described below in Section x.x.x; 


Moreover, there are functions to adapt to the train: so, for example, the processing 
a "separation distance" in the airborne equipment trigger: 
This is dependent on the distribution of functions between the 
Control monitoring equipment (which the ERTMS / ETCS), and the other 
CCS Systems.

\section{openETCS Architecture: Iterations and History}

The openETCS Architecture and Design is implemented in iterations \cite{deployment}. The current step (second iteration) is based on a step to implement the kernel functions of the ETCS system \cite{firstIteration}. For a better understanding of the scope the Iteration is described in the following.

\subsection{First Iteration Functional Scope: The Minimum OBU Kernel Function}
\label{sec:FunctionalScopeTheMinimumOBUKernelFunction}

The openETCS first iteration architecture and the design of the openETCS OBU software as mainly specified in \cite{subset-026} UNISIG Subset\_026 version\_3.3.0. 

The appropriate functionality has been divided into a list of functions of different complexity (see the WP3 function list \cite{functions}).

All these functions are object of the openETCS project and have to be analysed from their requirements and subsequently modelled and implemented. With limited manpower, a reasonable selection and order of these functions is required for the practical work that allows the distribution of the workload, more openETCS participants to join and leads to an executable---limited---kernel function as soon as possible. 

While the first version of this document focuses on the first version of the limited kernel function, it is intended to grow in parallel to the growing openETCS software.

The first objective of WP3 software shall be
\begin{itemize}
	\item ``Make the train run as soon as possible, with a very minimum functionality, and in the form of a rapid prototype.''
\end{itemize}
This does not contradict the openETCS goal to conform to EN50128.
\begin{itemize}
	\item After a phase of prototyping, the openETCS software shall be implemented in compliance to EN50128 for SIL4 systems.
\end{itemize}
Additional goals for this document are
\begin{itemize}
	\item Identification of the functions required for a minimum OBU kernel
	\item Architecture overview regarding the minimum OBU kernel
	\item Technical approach: Description of the proceeding and methods to be used
	\item Road map of the minimum OBU kernel functions
	\item Road map thereafter
\end{itemize}

\subsection{How to find the functions of the First Iteration in the Architecture}
The functions will be merged with the new architecture. Wherever a function has already been in the scope it will be marked as "first iteration".

\section{Glossary and Abbreviations}

\textbf{API} Application Programming Interface\\
\textbf{EVC} European Vital Computer\\
\textbf{BTM} Balise Transmission Module\\
\textbf{SRS} System Requirements Specification\\


\textbf{SRS-Subset 26}\\
\textbf{QA-Plan: D1.3.1}\\
\textbf{Process: D2.3}\\
\textbf{Methods: D2.4}\\
\textbf{API: D2.7}\\

\chapter{The openETCS Architecture}


\begin{figure}[hbtp]
\chapter{SRS Architecture}
\centering
\includegraphics[angle=90, scale=0.9] {images/HighLevelArchitecture.png}
\caption{SRS architecture}
\end{figure}



\begin{figure}[hbtp]
\chapter{Functional Breakdown}
\centering
\includegraphics [angle=90, scale=0.45] {images/HighLevelFunctionalbreakdown}
\caption{SRS Breakdown}
\end{figure}
\newpage



\chapter{Description of the SRS Functions}

\section{Reference abstract hardware architecture}

\subsection{Why a reference abstract hardware architecture?}

For proper understanding of openETCS API and of constraints imposed on
both sides of the API, we need to define a \define{reference abstract hardware architecture}. This hardware architecture is ``abstract''
is the sense that the actual vendor specific hardware architecture
might be totally different of the abstract architecture described in
this chapter. For example, several units might be grouped together on
the same processor.

However the actual vendor specific architecture shall fulfil all the
requirements and constraints of this reference abstract hardware
architecture and shall not request additional constraints.

\subsection{Definition of the reference abstract hardware architecture}

\begin{figure}
  \centering
  \includegraphics[width=\linewidth]{abstract-hardware-architecture.pdf}
  \caption{Reference abstract hardware architecture}
  \label{fig:hardware-arch}
\end{figure}

The reference abstract hardware architecture is shown in figure
\ref{fig:hardware-arch}.

The reference abstract hardware architecture is made of a bus on which
are connected \define{units}:
\begin{itemize}
\item EVC (European Vital Computer);
\item TIU (Train Interface Unit);
\item ODO (Odometry);
\item DMI (Driver Machine Interface);
\item STM (Specific Transmission Module, up to 8 units);
\item BTM (Balise Transmission Module);
\item LTM (Loop Transmission Module);
\item EURORADIO;
\item JRU (Juridical Recording Unit);
\item Zero or more Vendor specific unit.
\end{itemize}

A given instance of openETCS might not have all of above
units. \FIXME{Define a set of mandatory units?}

Those units shall working concurrently. They shall exchange
information with other units through asynchronous message passing.

% LocalWords:  Alstom openETCS EVC BIU TIU Odometry DMI STM BTM Balise LTM API
% LocalWords:  EURORADIO JRU

\subsection{Reference abstract software architecture}
\label{software-arch}

\subsection{Overall architecture}

\begin{figure}[htbp]
  \centering
  \includegraphics[width=\linewidth]{software-architecture.pdf}
  \caption{Reference abstract software architecture}
  \label{fig:software-arch}
\end{figure}

The \define{reference abstract software architecture} is shown in figure
\ref{fig:software-arch}. This architecture is made of following
elements:
\begin{itemize}
\item \define{openETCS executable model} produced by the
  \cite{scade-model}. It shall contain the program implementing core
  ETCS functions;
\item\define{openETCS model run-time system} shall help the execution
  of the openETCS executable model by providing additional functions
  like encode/decode messages, proper execution of the model through
  appropriate scheduling, re-order or prioritize messages, etc. This
  block shall be described in another openETCS document. \FIXME{ref?}
\item \define{Vendor specific API adapter} shall make the link between
  the Vendor specific platform and the openETCS model run-time system.
  It can buffer message parts, encode/decode messages, route messages
  to other EVC components, etc.
\item All above three elements shall be included in the EVC;
\item \define{Vendor specific platform} shall be all other elements of
  the system, bus and other units, as shown in figure
  \ref{fig:hardware-arch}.
\end{itemize}

We have thus three interfaces:
\begin{itemize}
\item \define{model interface}
 is the interface between openETCS
  executable model and openETCS model run-time system. It shall be
  described in another openETCS document \FIXME{ref?};
\item \define{openETCS API}
 is the interface between openETCS model
  run-time system and Vendor specific API adapter. It is described in
  this document;
\item \define{Vendor specific API}
 is the interface between Vendor
  specific API adapter and Vendor specific platform. This interface is
  not publicly described.
\end{itemize}

The two blocks openETCS executable model and openETCS model run-time
system are making the \define{Application software}
 part. This Application software might be either openETCS reference software or
vendor specific software.

The Vendor specific API adapter is making the \define{Basic software} part.

\subsection{Information exchange between blocks}

At this level of description, we do not explain how the various blocks
of above architecture are calling themselves. We only assume they are
exchanging \define{messages} in an asynchronous way. A message is a set
of information corresponding to an event of a particular unit, e.g. a
balise received from the BTM. The possible kind of messages are
described in chapter \ref{information-flows}.

How the exchange of messages in implemented in actual software,
e.g. function call, storage of data in a shared buffer, ..., is
described in chapter \ref{concrete-interface}.

\subsection{Architectural variations}
Please note that the reference abstract hardware and software
architectures do not forbid architectural variations. For example, the
Odometry function could be put within the EVC (see ODO on figure
\ref{fig:software-arch}) instead of a separate hardware unit (as it
was shown on figure \ref{fig:hardware-arch}). Such Odometry function
would be part of the Application software. But communication between
this Odometry function within EVC and the openETCS model run-time
system shall be done through the openETCS API and shall follow its
conventions.


\subsection{openETCS Model Runtime System}
The openETCS model runtime system also provides:

\begin{itemize}
\item Input Functions From other Units\\
In this entity messages from other connected units are received.
\item Output Functions to other Units\\
The entity writes messages to other connected units.
\item Conversation Functions for Messages (Bitwalker)\\
The conversion function are triggered by Input and Ouput Functions. The main task is to convert input messages from an bit-packed format into logical ETCS messages (the ETCS language) and Output messages from Logical into a bit-packed format. The logical format of the messages is defined for all used types in the openETCS data dictonary. \\
Variable size elements in the Messages are converted to fixed length arrays with an used elements indicator.\\
Optional elements are indicated with an valid flag.
The conversion routines are responsible for checking the data received is valid. If  faults are detected the information is passed to the openETCS executable model for further reaction. 
\item Model Cycle\\
The executable model is called in cycles. In the cycle 
\begin{itemize}
\item First the received input messages are decoded
\item The input data is passed to the executable model in a predefined order. \textbf{(Details for the interface to be defined)}.
\item Output is encoded according to the SRS and passed to the  buffers to the units.
\end{itemize}

\end{itemize}

As another example, part of Vendor specific platform could be on EVC
and thus the Vendor specific API would be within the EVC.

\newpage
\subsection{F1: openETCS API (Input)}
\begin{figure}[hbtp]
\centering
\includegraphics [scale=0.5] {images/F1_Exchange_input}
\caption{Exchange input - API input}
\end{figure}

\textbf{Interfaces aligned with the Alstom API to be added}\\
The exchange input needs to be seperated in 
Basic SW\\
and\\
Application SW \\
since the functions CRC and Bitwalker belong to a basic function. The integrity Check belongs to Application function.\\

\textbf{See Figure 5}\\
 
Inputs:\\
\begin{itemize}
\item TIU
\item DMI
\item BTM
\item EURORADIO
\item ODOMETRY
\item JRU
\item Parameterization board ?
\item Displacement measurement?
\end{itemize}

Outputs:\\
\begin{itemize}
\item The decoded and checked messages from other units are  are output of F1. The information is processed by procedure call provided by the executable model.
\item Details of the granualarity of those interfaces are to be clarified.
\end{itemize}

\section{F1.2: Manage Input}
\begin{figure}[hbtp]
\centering
\includegraphics [scale=0.9] {images/Manage_inputs}
\caption{Manage input}
\end{figure}

  Inputs:\\
``will be complete''\\

 Outputs:\\
 ``will be complete''\\
 
 \textbf{ Description:} ???
 
 \newpage
 \section{F1.4: Data structure}
\begin{figure}[hbtp]
\centering
\includegraphics [scale=0.5] {images/data_structure}\\
\caption{data structure}
\end{figure}

  Inputs:\\
``will be complete''\\

 Outputs:\\
 ``will be complete''\\
 
 \textbf{ Description:} ???
 

 \section{F2 Elaboration track messages}
 
 \textbf{See Figure 2 - Block F 2}\\
 
  Inputs:\\
``will be complete''\\

 Outputs:\\
 ``will be complete''\\
 
 Description: This function module provides the encoding and decoding of track messages that 
Management of the port and power off RBC, and the management of reference 
Beacons, which are based on the track messages.\\

 \subsection{	F21: Receive Eurobalise Messages} 
 \textbf{SRS:} § 3.4, § 3.6, § 3.16, § 3.17, § 4.8\\
 
  \textbf{Inputs:}\\
``will be complete''\\

 \textbf{Outputs:}\\
 ``will be complete''\\
 
 \textbf{Description:} 
 This function module provides the summary of telepowering information and the 
review of coherence the content of telepowering message. 
This module separates the eurobalise message as follows: 
 
 - The Balise telegram heads for checking the consistency of the different 
Telegrams, the management of the duplicated beacons and the acquisition of meaning. \\

- Then the information that the balise group, their reading direction and are 
Wegmessungsposition where the first balise group was read, in 
Baliseninformationen for monitoring the function "Manage the eurobalises" 
summarized (see the term Identifzierung and meaning in this function), which the 
Continuation of reading via the information Referenzbalise with a Balisenposition in 
Kernel-reference document approved. \\

\textbf{- The ETCS ETCS version and packages:} \\

The first review is the version ETCS ETCS track with the version Zugbord 
(Version 1.x Class1 =) to compare; these must be compatible (that means 1.y) to 
continue processing (a Balise version of the type is 0.y as incompatible 
considered, but they must not generate a negative reception report). \\

The second review is in compliance with the ETCS grammar in the resulting 
Packets with a packet 255, which terminates the beacon information of each of the group. \\

The reception of a packet 245 (standard package) in a eurobalise) includes the 
Rejection of all other packets received and generated a the same mistake as 
Grammatical errors ETCS, unless the OBU is in Level 2. \\

Only the packets 44, 65, 66 and 136 may be located several times in a same eurobalise message\\ 
For the 136 package: \\
• a single packet 136 per Balise telegram, \\
• the different packages 136 in a same message are identical. \\

The read packets are filtered as a function: \\

- from the direction of the transition to the balise group (the packet is the variable 
QDIR oriented), 
the bi-default state (level, level of announced and ERTMS / ETCS mode). \\

The tables of filtering per level and ERTMS / ETCS mode are in the description of the 
Function "received EUR radio messages" contain.\\

\textbf{These packets are transmitted in four different groups:}

- Geographical information (packet 79 for the "train locations") \\

- Information CNX / DCNX (packets 42, 131) for the "the CNX and DCNX 
Euro Radio Management "\\

- Information Balise (package 136: Referenzbalise for in-fill information) for the function 
"Manage eurobalises" \\

- Track-mail received (other packages) for the "monitor train". \\

\textbf{Any error of Referenzbalise, the version or the grammar ETCS is in CR (radio channel) 
Receiving reported.}\\
 
 
 \subsection{F22: Receive Euroradio Messages}
  \textbf{SRS} § 3.4, § 3.6, § 3.16, § 4.8\\
  
    \textbf{Inputs:}\\
``will be complete''\\

 \textbf{Outputs:}\\
 ``will be complete''\\
 
\textbf{Description:} 
This function module provides the summary of the Euro radio information and the 
Verify the consistency of the contents of the Euro radio message. 
The first guaranteed by this module check is to monitor the radio link. 
This only applies to a normal communication channel (see § 8.3.1 interface gSM-R). \\

\textbf{The review is broken down as follows:}

- Basic principle: the RBC is providing its messages with a time stamp based on the 
Time marking the bi-standard the equipment (if the timestamp of the messages on 
position is undefined, the message is accepted, but only during the initialization of the 
Communication session EVC - RBC).\\

Verification of the sequence: \\

- the ETCS OBU train equipment rejects any message that is "older" than the last 
received message.\\

- the ETCS OBU train equipment is up to any message that is "younger" than the last 
received message.\\
 	 
 	 
\subsection{F23: Manage Eurobalise Messages}
\textbf{SRS} § 3.4, § 3.6, § 3.16\\
  
   \textbf{Inputs:}\\
``will be complete''\\

 \textbf{Outputs:}\\
 ``will be complete''\\
 
\textbf{Description:} 
This function module manages the Balise reference document of the ETCS Onboard Unit, 
is said to know that the information supplied by the track following terms have: \\

- either the balise which provides the information, \\

- or delivered in the radio message Referenzbalise, \\

- or delivered in the infill balise message Referenzbalise (package 136). \\

A balise group contains 1-8 eurobalises. 
The balise group is referenced by an identification NIDLRBG (NIDC: identification 
+ NIDBG region: identification beacon). Each eurobalise has an internal number from 1 to 8, 
which describes the position of the beacon relaltive in the group. \\

A consisting of only one beacon balise group called "simple beacon", as 
any other balise group managed; However, it produces features that described in
Function module F23 and in the function module F25.\\

\subsection{F24: Manage Cnx and Dncx Euroradio}
\textbf{SRS} § 3.5, § 3.15.1, § 5.15\\

 \textbf{Inputs:}\\
``will be complete''\\

 \textbf{Outputs:}\\
 ``will be complete''\\
 
\textbf{Description:} 
This function module ensures the production ending a Euroradio- 
Communication. \\
For the ETCS OBU equipment it is possible to Euro Radio communication session
initiate: \\

- for a "start of mission": train data \\

- after one obtained from the track command: info CNX / DCNX. 
The command, an RBC to contact, the identity of the RBC and its telephone number 
(Packets 42 or 131).\\



 \subsection{F25: Send Euroradio Messages}
 \textbf{§ 3.4, § 3.16}\\
 
  \textbf{Inputs:}\\
``will be complete''\\

 \textbf{Outputs:}\\
 ``will be complete''\\
 
\textbf{Description:} 

This function module ensures the completeness (location and time stamp) and the 
Transmission of radio messages euro on the basis of information "radio session message", 
"Outgoing track message", "message acknowledgment" (for a radio message 136) and "CR 
Radio channel reception "(for a radio message 136 along with a package 4). \\

A radio message is sent only if between the ETCS OBU equipment and the 
RBC opened a euro radio session. \\

The radio messages, with the exception of the message confirmation (radio message 146) and the 
Session messages (radio message 154, 155, 156, 159) contain a "position report" package 
(O or 1 packet). \\

The structure of this "position report" package based on the following information: \\

- Info for locating the data LRBG, position, velocity, position error, moving direction,
by default state for the data concerning the level and the ERTMS / ETCS mode. \\

The package 1 "special position report" is used for one or two LRBG whose 
Crossing direction is not known ((LRBG type simple balise group). \\

The package 0 "position report" is used when the direction of the LRBG is known (group 
not simple beacons, or group simple beacons, whose Balisenrichtung over by the track 
was positioned a link or package 135) (see function F23)).\\
 	
 \textbf{See the functional breakdown of the function in F2 in the next paragraph (Figure 3)}\\
 
\subsection{F2 breakdown}	
 \begin{figure}[hbtp]
\centering
\includegraphics [angle=90, scale=0.3] {images/F2_Breakdown}
\caption{Elaboration track messages breakdown}
\end{figure}

\newpage
\section{F3 Train loction}
\textbf{See Figure 2 - Block F 3}\\

\textbf{SRS} § 3.6.4, § 3.6.5, 3.6.6\\

  \textbf{Inputs:}\\
``will be complete''\\

 \textbf{Outputs:}\\
 ``will be complete''\\
 
\textbf{Description:} 
This function module provides the location of the train in a kernel-reference document 
(Locating information), based on the following reference documents: \\

- path measurement calculation reference documents the function "distance measurement": train location, \\

- Balise the "track messages draw": Information computational comparison.\\


 \section{F4 Train Supervising in ERTMS Modus}
 \textbf{See Figure 2 - Block F 4}\\
 
\textbf{Inputs}:\\
``will be complete''\\
 
 \textbf{Outputs:}\\
 ``will be complete''\\
 
 \textbf{Description:} 
 This function module ensures the monitoring of the train by working out: \\
 
 - The level and mode of control,\\
  
- Monitoring the speed, position and the movement of the train,\\
 
- The driver interfaces,\\
 
The data, which is based on the monitoring.\\

 \subsection {F41: Manage Level ERTMS/ETCS}
 
 \begin{figure}[hbtp]
\centering
\includegraphics [scale=0.4] {images/Manage_Level_and_Modes}
\caption{Elaboration track messages breakdown}
\end{figure}
 
 \textbf{SRS}§ 5.1, § 3.6.5 \\
 
 \textbf{Inputs}:\\
``will be complete''\\
 
 \textbf{Outputs:}\\
 ``will be complete''\\
 
 \textbf{Description:} 
This function is only applicable for the Eurobalise or Euro radio messages 
announced transitions. The aspect of "use" of the transitions is in Chapter 5 (5.10]
described. \\

This function module manages the transitions of the level ERTMS / ETCS and the 
associated driver acknowledgments and releases associated with these transitions 
related actions. 
The function depends on the current level (managed by this function) and the 
following inputs: \\

- "Get track message": notice the transition, the immediate transition command; \\

- «Info locating": position of the train; \\

- "Request by the driver": acknowledgment of the transition; \\

- «Info train": information exchange of the active driver's cab.\\


 \subsection {F42: Manage Modus ERTMS/ETCS}
 \textbf{SRS}§ 4, § 3.6.5, § 3.15.4, § 5.5, § 5.6, § 5.7, § 5.9, § 5.11, § 5.13 \\
 
 \textbf{Inputs}:\\
``will be complete''\\
 
 \textbf{Outputs:}\\
 ``will be complete''\\
 
 \textbf{Description:} 
description 
This function module manages the transitions of the ERTMS / ETCS mode and the 
associated acknowledgments by the driver, and triggers associated with these transitions 
associated actions. \\

The function depends on the following inputs: \\

 - "Get track message": Adopted train data, SH mode approved indication signal 
(VMAIN = 0), Information «danger for SH" Information "stop if in SR"; \\

- Info locating": train position and indication information; \\

- "Mapper": available allocation to in SR mode, FS, OS or RV 
drive requirement of "train trip" (triggering the train-emergency); \\

-"Request driver»: acknowledgment of the driver 
proposed modes SH and NL, acknowledgment at the end of TR mode, 
Preventing the TR mode; \\

- Info train": TIU Inputs (off, active cab, signal leading / led 
Cab, field monitoring system, condition of equipment; \\

- "ETCS Mode": current mode and current level; \\

- "CR reception": a reported on a balise error or loss of the radio link 
and related response; \\

- "Train protection»: in SH mode Balise prohibited, banned in SR mode Balise 
and the status of theETCS OBU equipment forming equipment (TIU, DMI, 
BTM, Euro radio, board-parameterization, position measurement): any error in the equipment 
(not shown "material flow").\\

\subsection {F43: Train Speed supervision}
 \textbf{SRS}§ 3.7, § 3.8, § 3.10, § 3.11, § 3.12, § 3.13, § 5.7, § 5.8, § 5.9 \\
 
 \textbf{Inputs}:\\
``will be complete''\\
 
 \textbf{Outputs:}\\
 ``will be complete''\\
 
 \textbf{Description:} 
description 
This function module monitors the train speed in comparison to a 
Track assignment / a Movementauthority. \\

You must track the assignment using the information obtained track work out, ie the 
Completeness of the assignment using the information obtained from the track information
concerning MA, check profiles and parameters. \\

\textbf{A complete assignment is based on the following information:} \\

- The radio messages 2 (Approval SR), 3 (​​MA), 33 (MA reference it) and the packages 12 
(MA Level 1), 15 (MA Level 2), 80 (mode profiles), 138 (Reversing area information), 139 
(Reversing supervision information) give information on the approval of 
Movement on the track assignment; \\

- The radio messages 9 (requirement of shortening of MA), 15 (conditional urgent 
Stopping), 18 (refusal of urgent arrest) and 16 packets (repositioning), 
70 (Route Suitability) give the "extensions" of the movement permit information 
concerning the track assignment, that is, that they complement the original track assignment, 
without replacing them; \\

- The radio message 16 (unconditional urgent stop) changes the target point, so 
the train to travel (emergency) passes (equivalent to the train which his 
Has exceeded track assignment); \\

- The packages 21 (Gradientprofil), 27 (velocity profile), 51 (axle load), 65 
(temporary restriction) and 66 (refusal temporary restriction) 
enter the profile information on the track assignment; \\

- The package 57 (MA requirements) are the parameters concerning the track assignment; 
for the technical operation modes, the assignment only to provide information to 
technical mode are limited; \\

- The assignment can also on the position of the train (stop) at the end of a mission, 
a missed beacon or a radio or loss in accordance with the current technical 
Be reduced mode (SRS §4.10). \\

- The package 71 (adhesion factor) called the coefficients of the delay tables 
      Change train (train data service braking and emergency braking): 70\% or 100\%. 
      It should be mentioned that the change of these coefficients by the driver on the 
      ETCS onboard unit is not approved. To ensure interoperability must 
      the application but the "track adhesion" function on the driver 
      manage (see the national value of QNVDRIVERADHES). \\
      
- For mode FS and OS mapping is complete when the gradient profiles and SSP to 
End of the MA are known; Otherwise MA is denied.\\

\subsection {F44: Train Movement supervision}
 \textbf{SRS}§ 3.14 \\
 
 \textbf{Inputs}:\\
``will be complete''\\
 
 \textbf{Outputs:}\\
 ``will be complete''\\
 
 \textbf{Description:} 
This function module supervises train movements, ie it ensures protection against 
the involuntary movements of the train: \\

- Protection against the elopement of the train, \\

- Protection against rolling back of the train, \\

- and monitoring the maintenance of the train. \\

\textbf{This module triggers an unwanted movement of the train from the service brake.} \\

\textbf{Protection against the elopement of the train}\\
The protection against the elopement of the train to avoid the train in one direction 
moves that do not match the current position of the field monitoring of the active driver's cab 
matches. 
This protection is applicable only in SH mode, FS, SR, OS, UN, PT and RV.
If the removal of unwanted motion exceeds the national value DNVROLL, 
The brake is activated. 
When the train goes back again to stop, the brake is by consent of the 
Train driver released. \\

\textbf{Protection against rolling back of the train }\\
The protection against rolling back of the train to avoid the train in the his 
Direction opposite direction, i.e., his current assignment, moved. 
This protection is applicable only in FS mode, SR, OS, PT and RV. 
If the removal of unwanted motion the national value DNVPOTRP for the 
Mode PT and PT and et DNVROLL exceeds for the other modes, the brake is 
activated. When the train goes back again to stop, the brake is by consent of the Train driver released. \\

\textbf{Monitoring the stop of the train }\\
The monitoring of the stop of the train to avoid the train moves, if he in 
SB mode is stationary. 
If the removal of unwanted motion exceeds the national value DNVROLL, 
The brake is activated. 
When the train goes back again to stop, the brake is by consent of the 
Train driver released.\\


\subsection {F45: Train position supervision}
\textbf{SRS} § 3.6.5, 4.4.8, § 4.4.11\\

\textbf{Inputs}:\\
``will be complete''\\
 
 \textbf{Outputs:}\\
 ``will be complete''\\
 
 \textbf{Description:} 
 This function module monitors the trainposition, namely: 

- protection against overshooting the unapproved mode in SH or SR beacons, \\

- and the report "report" the pull position (on the conditions of the train position) to the 
RBC. \\

\textbf{Protection of unauthorized beacons}\\
The protection against overshooting of non-approved mode in SH or SR beacons based 
on the list of approved packages ETCS balise the 49 and 63rd 
This means in each affected mode (SH or SR): \\
 
- if a list was supplied, it contains the list of approved beacons: the 
Driving over a balise not included in this list causes the transition to mode TR 
the ETCS Onboard equipment; \\

- if an empty list is supplied, this means that no beacon is approved: the 
Driving over any Balise causes the transition to mode TR ETCS
On-board equipment; \\

- if no list is supplied, each beacon can be run over. \\


\textbf{position report }\\
At each stop the train in FS mode, OS and SR is systematically a position report on 
a radio message 136 produced. \\

Depending on the requirements of the package 58 ETCS another position report is generated 
("Position report parameters"), i.e. .: \\
 
- a temporary cyclical position report, \\

- a moderate distance zykischer position report, \\

- an immediate position report, \\

- a position report at each crossing a balise group, \\

- a position report at specific locations, ie if the max. Head of the train 
(max safe front end) or the min. End of the train (min safe rear end) a specific 
Has crossed position. \\

- Otherwise, without a request by the track, the position report is limited to each 
Driving over a balise group.\\

\subsection {F46: Data storage}
\textbf{SRS} (§ 4.3), § 3.18\\

\textbf{Inputs}:\\
``will be complete''\\
 
 \textbf{Outputs:}\\
 ``will be complete''\\
 
 \textbf{Description:} 
This function module guarantees the storage of various data in the 
"Train data"; these are then used for the entire function modules of the kernel available (it doesnt show the flow, they are distributed to the total functions).\\
 
\textbf{This train data are broken down as follows:}\\ 

- Parameterization train maintenance; These are the characteristics of the train and the track 
(Train data, fixed data and otherwise national standard data),\\
 
- Additional data; these are the data of the mission of the train. \\

The additional data from the driver at the end of the process "start of mission» 
adopted (see function "with the driver dialoguing) and § 6.3 procedure 
"Start Of Mission). This validates the additional data and allows the validation of the beginning 
a mission. \\

See § 4.3 DATA for details of the data. \\
The train data are sent over a track auszusendende message to the RBC 
(Radio message 129, Parcel 11). 
The data obtained from the track national values ​​(Package 3) will be stored and 
considered when the track region (NIDC the Balise of locating information) in one of the 
received national values ​​specified region.\\

\subsection {F47: Dialog with the driver}
\textbf{SRS} § 5.4, § 3.12.3\\

\textbf{Inputs}:\\
``will be complete''\\
 
 \textbf{Outputs:}\\
 ``will be complete''\\
 
 \textbf{Description:} 
This function module ensures the dialogue with the driver at a start 
a mission and for all on-board advertisements (messages, symbols, dynamic data) and the track 
Text messages. \\

The method "start of mission" consists in starting or restarting (end of mission 
the train) for a new mission. \\

\textbf{See Chapter 5 for the method "start of mission."} \\

- The track text message is a resulting free track text message (packet 72) with display and
Display end conditions that are dependent: \\
    • Level of ERTMS / ETCS, \\
    • mode of ERTMS / ETCS, \\
    • from a distance, \\
    • from one time interval, \\ 
    • of an acknowledgment by the driver. \\
The condition can be single or multiple (combination of multiple conditions). 
In the case of the expectation of an acknowledgment by the driver at the end of other 
Combined display conditions, and if these are met, the managed Eurocab- 
Board equipment service braking (lack of acknowledgment). \\

A resulting encoded text message (packet 76) generates an error message because the text codes 
are not defined. \\

The track text messages and the driver displays are in messages 
summarized, which are intended for the DMI: dynamic message (coming from the KV 
Information) symbols and train-text messages. \\

The train-text messages are managed by internal text codes that the DMI an ad in 
chosen by the driver language enable (choice at the beginning of the "start of 
mission "or during the mission).\\

\subsection {F48: Manage brake controll}
\textbf{SRS} § 3.14.1\\

\textbf{Inputs}:\\
``will be complete''\\
 
 \textbf{Outputs:}\\
 ``will be complete''\\
 
 \textbf{Description:} 
This function module ensures the release of the brake control (output 
"Brake control"). \\

The allowances are as follows: \\
 
 - The technical TR and SF mode (input "State ETCS") produce a 
emergency braking, \\

- The lack of acknowledgment of the level, the mode or text message (input 
"Lack of acknowledgment") causes a braking operation, \\

- The protection of the train by the movement monitoring (input "train protection") causes a 
Service braking \\

- FS a reaction to a Bali mustard Ehlers or a wireless loss (input "CR 
Receiving "causes a braking operation, \\

- The direction indicated by the speed monitoring overspeed FS or FU 
(Input "overspeed") causes an operating or emergency braking. \\

Each service brake application (except for overspeed) is an emergency braking 
replaced when the service brake: \\

- Does not exist (train data), \\

- Not after the brake reaction time (train data) after reading the service brake (input 
"Info train") is not working. \\

The solution braking is managed by the requesting functionalities (at a 
Change the mode or the rectification of the fault).\\

\subsection {F49: Manage Train Control}
\textbf{SRS} § 3.12.1, FFFIS TIU\\

\textbf{Inputs}:\\
``will be complete''\\
 
 \textbf{Outputs:}\\
 ``will be complete''\\
 
 \textbf{Description:} 
This function module ensures the development of the using of the train control: \\

- to Control: service brake or emergency brake\\

- the resulting track message "tracks conditions" (Package 68) \\

- the resulting track message "track condition big metal masses" (Package 67) \\

- the resulting track message "linking", which allows a calibration range to 
define (Package 5). \\

The drafting of "track conditions" and the "calibration range" is national in § 4.4 
And functions described in § 4.5. The "track conditions" (Package 68) activate the corresponding train control systems and give them 
the driver to. The package 68 provides the start and end of the "tracks 
conditions ". \\
The "big metal masses track condition" (Package 67) puts the transmission in BTM state 
OFF. \\
This causes the disruption of the broadcast antenna of the ETCS onboard equipment. 
The package 67 provides the beginning and end of the track condition "big metal masses." 
At the end of this condition in the transmission mode is "not modulated", so the the 
Default is to read the eurobalises. \\

There is a manual calibration of the measurement system exists, it is, however, directly from the 
Linear Position Sensing subsystem through the acquisition of the Wegmessungsdaten 
Bordparametrierungsmoduls managed. 
However, the bi-standard onboard equipment is capable of automatic calibration ranges of 
To capture Bordwegmessungssystems: 
A calibration range is announced between two Eurobalisegruppen 
Calibration area having the following properties: \\
     • Distance between the Euro balise 'groups greater than 1000m, \\
     • Linking error between the Euro balise 'groups smaller than 1 m, \\
     • Gradientprofil between the eurobalises, the closest to a zero-gradient 
         are, \\
     • Travel between the beacons with a least possible number of curves. \\
Therefore, these properties are the responsibility of implementing this track 
Areas, but the first two conditions are those of the ETCS onboard equipment 
allow to detect a calibration area.\\


 \subsection{F4 breakdown}	
 \begin{figure}[hbtp]
\centering
\includegraphics [angle=90, scale=0.25] {images/F4_Breakdown}
\caption{Train Supervising in ERTMS Modus}
\end{figure}

\newpage

 \subsection{F5 Exchange outpout}

 \textbf{ See Figure 2 - Block F 5}\\
 
 \textbf{Inputs}:\\
``will be complete''\\
 
 \textbf{Outputs:}\\
 ``will be complete''\\
 
 \textbf{Description:} 
 description 
This function module manages the exchange at the output of the ETCS On board equipment with 
the following modules: 

- TIU, \\
- DMI, \\
- BTM, \\
- EURORADIO,\\ 
- SAM, \\
- Displacement measurement. \\

It complements their functions at the level of exchange protocols and the "safety layer" for 
the entire module with the exception of Wegmessungsmoduls (that manages its own 
Exchange protocols). 
Note: The manner in which these functions are implemented, is free; the 
details of which are specified in the software specification documents.\\
 
\chapter{Data Flow Description}
\tablefirsthead{
\hline 
\rowcolor{gray} 
Number & Flow & Source/Sink & Description  \\\hline}
\begin{supertabular}{| m{3,5cm} | m{3,5cm} | m{3,5cm} | m{3,5cm} |}
1.0 & Missing Acknowledge & Kernel xx/Kernel xx & This information indicates the absence of the 
Acknowledgment of the driver to a 
confirmatory text message on the Level 
Transition, the transition mode or a track 
Train-text message\\\hline
2.0 & Acknowledge & Kernel xx/Kernel xx & Radio message 146 transmission Confirmation - Confirmation message depending on the variable 
§ 8.6.7  \\\hline
3.0 & Action Driver & Kernel xx/Kernel xx &  Election / confirmation / detection Driver via DMI \\\hline
4.0 & Display Driver & Kernel xx/Kernel xx & For the driver specific DMI display \\\hline
5.0 & Reference Balise & Kernel xx/Kernel xx & information; in the Flow "information Balise"  delivered Balise confirmed as a reference beacon.\\\hline
6.0 & Allocation available & Kernel xx/Kernel xx & Information message that an allocation for a given technical mode is available.\\\hline
7.0 & Calibration & Kernel xx/Kernel xx & Specifying a calibration range of the 
path measurement.\\\hline
8.0 & Cnx/Dcnx Euroradio & Kernel xx/Kernel xx & Command on the connection and disconnection of a 
or of a RBC, the identity and 
Phone RBC contains.\\\hline
9.0 & Brake Controll & Kernel xx/Kernel xx & Control of the service and the 
emergency braking. \\\hline
10.0 & Train Controll (brake) & Kernel xx/Kernel xx & Functional output.\\\hline
11.0 & CR Reception (Radio Channel) & Kernel xx/Kernel xx & CR (radio channel) on receipt of a balise or 
Radio message: Checking version ETCS, 
Grammar ETCS, testing reference balise 
(Position), monitoring radio link\\\hline
12.0 & Accepted Data & Kernel xx/Kernel xx & Information valid confirmation of the train data and 
Consequently, at the very beginning of the mission.\\\hline
13.0 & Additional Data & Kernel xx/Kernel xx & Data relating to the mission of the train: 
Identity of the driver, 
selected level, identity and mandatory. of the RBC, 
if Level 2, no. Mission\\\hline
14.0 & JRU Data & Kernel xx/Kernel xx & Recorded legal data.\\\hline
15.0 & Input Train & Kernel xx/Kernel xx & From the train coming TOR digital inputs and 
Digital outputs \\\hline
16.0 & State ERTMS & Kernel xx/Kernel & Current Level and operating mode ERTMS, announced ERTMS Level and Current Level and operating mode ERTMS, announced ERTMS Level\\\hline
17.0 & Indication Driver & Kernel xx/Kernel xx & each specific for the driver 
Specification. This includes the dynamic information 
speed monitoring the 
different messages and symbols, the 
Exchange during the "start of mission"\\\hline
18.0 & Info Train & Kernel xx/Kernel xx & From Train comming functional information TIU \\\hline
19.0 & Message Eurobalise & Kernel xx/Kernel xx & Information, which the ID 
Balise group (country ID + ID 
Balise), their reading direction and their 
Contains path measurement for review 
(see Flow "reference balise")\\\hline
20.0 & Information cnx/dcnx & Kernel xx/Kernel xx & Information identifying the RBC ID 
(Identification of land + the identification of the RBC), 
RBC his phone number and the authorization \\\hline
21.0 & Information Driver & Kernel xx/Kernel xx & Specific for the driver 
Function DMI information. They contain the 
different specifications for the 
Driver, which are generated from the 
different Function modules.\\\hline
22.0 & Geographical information & Kernel xx/Kernel xx & Absolute current position of the train, which the 
add the flow "location information" \\\hline
23.0 & Information Localisation & Kernel xx/Kernel xx & Locating the train: position and speed of the train in the 
different reference documents, 
including the absolute geographical 
Position.\\\hline
24.0 & Message Eurobalise & Kernel xx/Kernel xx & Flow, summarizing the message / messages eurobalise. 
BTM to F1: the entire telegrams 
F1 to F2: Message 
(the safety layer and the 
the safety layer and the transfer of 
Telegrams in a message by F1 
F1 guaranteed)\\\hline
25.0 & Infos Calculation Adjustment & Kernel xx/Kernel xx & Information, which contains the ID 
Balise group (country ID + ID 
Balise), their path measurement and her for the 
computational comparison of the on-board 
Contains reference document expected position\\\hline
26.0 & Localisation Train (ODO) & Kernel xx/Kernel xx & path measurement in comparison to a 
Balise (+ safety margin), Speed ​​(+ 
Safety distance), direction of travel and capture 
of stopping\\\hline
27.0 & Euroradio Message (train to track) & Kernel xx/Kernel xx & Transmitted Euro radio message 
(the safety layer is guaranteed by F5)\\\hline
28.0 & Euroradio Message (track to train) & Kernel xx/Kernel xx & Received Euro radio message 
(the safety layer is guaranteed by F1)\\\hline
29.0 & Radio Message Session & Kernel xx/Kernel xx & Message / parcel Euro Radio for creating and 
Termination of the radio connection\\\hline
30.0 & Outgoing Track Message & Kernel xx/Kernel xx & Before Inform creation and timestamp 
(except the message flow 
Radio link, message confirmation and CR 
Reception) sent to the track to 
function message\\\hline
31.0 & Message from Track Received & Kernel xx/Kernel xx & Via radio or beacon message received function 
(except the Riverside current 
Information, cnx / dcnx information, Balise 
information)\\\hline
32.0 & Parameter Train maintenance & Kernel xx/Kernel xx & Fixed data, train data, domestic values\\\hline
33.0 & Train Protection & Kernel xx/Kernel xx & Information, protection against rolling 
Twitching, monitoring of stopping, the 
Transition to an unauthorized Balise in SH 
or SR reports\\\hline
34.0 & Balise Reception & Kernel xx/Kernel xx & Information of receipt of a reference Balise NPIG = 0 from the associated type for the computational comparison of the displacement measurement\\\hline
35.0 & Requirement Driver & Kernel xx/Kernel xx & By driver outbound (see flow 
"Action Driver") or from DMI DMI produced 
function information\\\hline
36.0 & Euroradio Session & Kernel xx/Kernel xx & State of the Euroradio connection \\\hline
37.0 & Signal Eurobalise & Kernel xx/Kernel xx & Airgap Eurbalise -> BTM\\\hline
38.0 & Signal Euroradio & Kernel xx/Kernel xx & Airgab RBC ->Euroradio Board\\\hline
39.0 & Output Train & Kernel xx/Kernel xx & Train digital outuputs\\\hline
40.0 & Overspeed & Kernel xx/Kernel xx & Information message overspeed
of the train relative to the
speed curves\\\hline
41.0 & Synchronization & Kernel xx/Kernel xx & Synchronization (xw, V, T) between the kernel, 
the BTM and the path measurement\\\hline
42.0 & Transmission BTM & Kernel xx/Kernel xx & Control BTM - antenna\\\hline
\end{supertabular}

\chapter{Functional and data structure architecture ETCS on-board}
 Description of Functional and data structure  architecture ETCS on-board from NS (Jan Welvaart and Vincent Nuhaan)
 will be followed ...!!\\
 To be considered and merged to the overall architecture view!!\\
 
\section{Management of location based data}
A lot of information given from ETCS track side to on-board is “Location based”, i.e. the information is valid for a certain location given as a distance from a reference BG. The current function will manage the elaboration of packets containing “Location based data” into the (already available) sorted internal data structures which shall enable efficient further elaboration of the data (e.g. braking curve monitoring, constituting the planning area, etc.).

\textbf{Description:}
Elaboration of the packets containing “Location based data” into the (already available) sorted internal data structures.

\textbf{Inputs:} 
\begin{itemize}
\item Packets containing “Location based data” (see below)
\end{itemize}

\textbf{Affected data in the “data stored on-board”: 	}
\begin{itemize}
\item Internal data structures storing “internal data structures” 
\item Status of received packets (change to “elaborated”, thus may be forgotten) 
\end{itemize}
\textbf{Outputs:}  -

“Location based” data is categorized for the purpose of defining data structures:
\begin{itemize}
\item \textbf{movement authority (MA) list of sections}, message 37, packet 12 (level 1), message 3, packet 15 (level 2), 16 (repositioning, i.e. extending the current section), message 33 (??), packet 70 (route suitability), message 9 (request to shorten MA),    minimum number of elements to be stored: 6
\item \textbf{list of announced BG's linking information:} packet 5  minimum number of elements to be stored: 30
\item \textbf{adhesion factor:} packet 71;  only one element
\item the \textbf{“gradient profile”} (in: pkt 21)  minimum number of elements to be stored: 50
\item \textbf{Speed profiles:}  packet 27 (SSP)  {the worst case can be determined at reception} 
Packet 13	minimum number of elements to be stored: 50
\item \textbf{Speed restrictions} and non-continuous speed profiles: packet 51 (axle load profile), packet 52 (permitted braking distance), packets 65/66 (TSR), packet 88 (level crossing, incl. stop condition to be reset at standstill).  
minimum number of elements to be stored: TSR: 30, axle load: 30, permitted braking distance:  5 , level crossing: 10. 
Reversing area's: packets 138, 139	minimum number of elements to be stored: 1
Mode dependent speeds: message 2  and packet 80  minimum number of elements to be stored: 6
\item \textbf{Level transitions:} packet 41  minimum number of elements to be stored:  (see ss26, 5.10.1.6): 1
\item \textbf{DMI information:} packets 72,76 (text messages), packet 79 (geographical position information), message 34 (track ahead free request)  minimum number of elements to be stored: fixed text: 5, free text: 5, geographical position: 
\item \textbf{Track conditions} (to be passed to the TIU and displayed at the DMI): packet 39 (traction system), packet 40 (current limitation),  packet 67, 68 (diverse track conditions), packet 69 (platform conditions). Pkt 139  minimum number of elements to be stored: 20, + 1 for change power supply + 1 for platform conditions, + 1 for current limitation
\item \textbf{Route suitability:} minimum number of elements to be stored: 3
\item \textbf{Big Metal Mass:} Technical information (to be used for BG-filtering): packet 67 (ignore BG integrity)  minimum number of elements to be stored: 5
\item \textbf{Virtual balise covers:}  minimum number of elements to be stored: 10
\item \textbf{list of position report locations.} In: pkt. 58  minimum number of elements to be stored: 15
\end{itemize}

All above information contains “Locations”. These “Locations” can all be managed in the same manner. A location is defined using the following data structure:

For each location the “original reference” (ORBG) and the distance from this ORBG are remembered. 
In the data structure for passed BG's a correction distance is stored such that:

\begin{figure}[hbtp]
\centering
\includegraphics[angle=0, scale=0.9] {images/ORBG.png}
\caption{SRS architecture}
\end{figure}


$d_{ between~train~and~location} =
d_{from~ORBG} - P_{Train~to~ LRBG}  -  d_{correction~stored}$

\FIXME{formula needs to be documented with a picture or refined description of the variables}

where
\begin{description}
   \item[$d$] - represents a distance
   \item[$P$] - represents the train position
\end{description}
$the stored correction distance (all including tolerances)$.

A location is thus always stored as a distance from the nominal location of the ORBG (thus not having tolerances). Location based data is given from track-side as a chain of incremental distances. This shall be converted during the elaboration and storage.

Typically location based data has to be available in the order it is passed. This requires ordering and reordering if an element The following situations have to be taken into account:
\begin{itemize}
\item some types of location based data are not sent in the order they are passed, i.e. an element can be received which has to be included halfway the list.
\item some types of location based data can be withdrawn from the list.
\item Some types of location based data can be updated
\item Information stored for a location can change, e.g. if the axle load changes, then the speed restrictions stored for this item shall be changed.
\item Shifting of data shall be avoided.
\end{itemize}

\begin{figure}[hbtp]
\centering
\includegraphics[angle=0, scale=0.6] {images/indexing_of_ringbuffer.png}
\caption{Indexing of internal data structure}
\end{figure}


General data structure for location based data of a specific type: “LOCATION BASED DATA STRUCTURE”
\begin{description}
\item[closest:] The element related to the location which is the closest to the train front end
\item[furthest:] The element related to the location which is the closest to the train front end
\item[first free:] The element where the next received “location based data element of the specific type
\end{description}

Each element shall contain the index of the element where the preceding location is stored,  the index of the element where the next further location is stored, the location (see type description above) and fields for information depending of the type of location based data.


\begin{figure}[hbtp]
\centering
\includegraphics[angle=0, scale=0.9] {images/location_based_datastructure.png}
\caption{Location based data structure}
\end{figure}

Generic structure for an element in a data structure for location based data


General functions to be performed on a “LOCATION BASED DATA STRUCTURE” when new information is received are:
\begin{itemize}
\item Replace all data from a certain location onwards with the new received information.
\item Insert one element in the structure at the right location, i.e. between the last preceding and the first next location of the same type.
\item Delete one element of the structure, i.e. restore the order and free the memory where the information was stored.
\item Update the information for a specific (already stored) element containing location based data, e.g. update the speed of “axle load dependent speed restrictions” in case the axle load changes.
\end{itemize}

\textbf{ORDER IN WHICH PACKETS CONTAINING LOCATION BASED DATA SHALL BE ELABORATED}

There are a few requirements determining the order in which different types of data shall be analyzed:
\begin{itemize}
\item MA's may only be accepted if a static speed profile and gradient information is available. Therefore the latter ones shall be elaborated before the MA information is elaborated.
\item xxxx
\end{itemize}

Scope of the function
The current function includes all elaboration of packets containing location based data into the internal data structure for location based data, further elaboration is not included.
MRSP and the list of targets for braking curve monitoring are also location based, but they are a further elaboration of the received data. The building of the MRSP and list of braking curve targets is therefore not handled in the current function but in “Build MRSP + list of targets at the LRBG” (see figure~\ref{fig:lbd}).


\begin{figure}[hbtp]
\centering
\includegraphics[angle=0, scale=0.55] {images/build_structure_for_location_based_data.png}
\caption{Build Structure of Location Based Data}
\label{fig:lbd}
\end{figure}

 
 \newpage
 \chapter{current partly openETCS Architecture - first iteration}
 Needs to be integrated into the overall architecture .....
  \begin{figure}[hbtp]
\centering
\includegraphics [angle=90, scale=0.2]{images/Current_partly_openETCS_architecture}
\caption{partly openETCS architecture - first iteration}
\end{figure}
 
 \newpage
  \chapter{merge the first iteration architecture with the overall architecture}
 Needs to be integrated into the overall architecture .....
  \begin{figure}[hbtp]
\centering
\includegraphics [angle=90, scale=0.3]{images/architecture-db}
\caption{concept to merge}
\end{figure}
 
 \newpage
 \chapter{centralized data structure approach}
 \begin{figure}[hbtp]
\centering
\includegraphics [scale=0.6] {images/CentralizedDataStrukture_2}
\caption{Centralized data structure approach architecture}
\end{figure}


 \begin{figure}[hbtp]
\centering
\includegraphics [scale=0.7] {images/CentralizedDataStructure_1}
\caption{Centralized data structure approach breakdown}
\end{figure}

\newpage

\chapter{Alstom High Level Approach}
 \begin{figure}[hbtp]
\centering
\includegraphics [scale=0.8] {images/Alstom_High_Level_Approach}
\caption{Alstom SRS Architecture Approach}
\end{figure}

\newpage

 
\appendix

\bibliographystyle{unsrt}
\bibliography{architecture}

\newpage
\addcontentsline{toc}{chapter}{Index}
\printindex

%===================================================
%Do NOT change anything below this line

\end{document}
