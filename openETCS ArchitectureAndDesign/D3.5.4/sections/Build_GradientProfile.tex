%set the master document for easy compilation
%!TEX root = ../D3_5_3.tex

\paragraph{Component Requirements}

\begin{longtable}{p{.25\textwidth}p{.7\textwidth}}
\toprule
Component name			& Build\_GradientProfile \\
\midrule
Link to SCADE model		& {\footnotesize \url{https://github.com/openETCS/modeling/tree/master/model/Scade/
System/ObuFunctions/TrackAtlas/TA\_Gradient.xscade}} \\
\midrule
SCADE designer			& Jakob G\"artner, LEA Railergy  \\
\midrule
Description				& Receives Track to Train Packet 21 (Gradient Profile). References the data to the train coordinate system. Converts incremental distances to absolute distances in the train's coordinate system. Merges the information from sequentially received packets into a continuous Gradient Profile. Truncates the profile as required\\
\midrule
Input documents	& 
Subset-026, Chapter 3.11.12\newline
Subset-026, Chapter 7\\

\midrule
Safety integrity level	& 4 \\
\midrule
Time constraints		& n/a\\
\midrule
API requirements 		& n/a \\
\bottomrule
\end{longtable}


\paragraph{Interface}

For an overview of the interface of this internal component we refer to the SCADE model (cf.~link above) respectively the SCADE generated documentation.