%set the master document for easy compilation
%!TEX root = ../D3_5_3.tex

\section{F2.4: TrackAtlas}\label{s:F2.4}
\todo[inline]{Section needs to be completed}

\subsection{Component Requirements}

\begin{longtable}{p{.25\textwidth}p{.7\textwidth}}
\toprule
Component name			& TrackAtlas \\
\midrule
Link to SCADE model		& {\footnotesize \url{???}} \\
\midrule
SCADE designer			& Jakob G\"artner, LEA \\
\midrule
Description				& ??? \\
\midrule
Input documents	& 
Subset-026, Chapter ???\\
\midrule
Safety integrity level	& 4 \\
\midrule
Time constraints		& [If applicable description of time constraints, otherwise n/a] \\
\midrule
API requirements 		& [If applicable description of API requirements, otherwise n/a] \\
\bottomrule
\end{longtable}


\subsection{Interface}

An overview of the interface of component TrackAtlas is shown in Figure~\ref{f:manage_track_data_interface}. The inputs and outputs are described in detail in Section~\ref{s:manage_track_data_inputs} respectively \ref{s:manage_track_data_outputs}. Subcomponents are described in Section~\ref{s:manage_track_data_subcomponents}.

\begin{figure}
\center
\missingfigure{[Put SysML diagram of component here]}
\caption{TrackAtlas component SysML diagram}\label{f:manage_track_data_interface}
\end{figure}


\subsubsection{Inputs}\label{s:manage_track_data_inputs}

\paragraph{[Input 1 name]}

\begin{longtable}{p{.25\textwidth}p{.7\textwidth}}
\toprule
Input name				& [Name of the input] \\
\midrule
Description				& [Brief description of the input] \\
\midrule
Source					& [Name of the source component] \\ 
\midrule
Type					& [Type of the input] \\
\midrule
Valid range of values	& [Complete list of valid values] \\
\midrule
Behaviour when value is at boundary	& [Description of components behaviour when input value is at boundary] \\
\midrule
Behaviour for values out of valid range	& [Description of components behaviour when input value is out of valid range] \\
\midrule
Behaviour when value is erroneous, absent or unwanted (i.e. spurious) & [Description of components behaviour when value is erroneous, absent or unwanted (i.e. spurious)] \\
\bottomrule
\end{longtable}


\paragraph{[Input 2 name]}

\begin{longtable}{p{.25\textwidth}p{.7\textwidth}}
\toprule
Input name				& [Name of the input] \\
\midrule
Description				& [Brief description of the input] \\
\midrule
Source					& [Name of the source component] \\ 
\midrule
Type					& [Type of the input] \\
\midrule
Valid range of values	& [Complete list of valid values] \\
\midrule
Behaviour when value is at boundary	& [Description of components behaviour when input value is at boundary] \\
\midrule
Behaviour for values out of valid range	& [Description of components behaviour when input value is out of valid range] \\
\midrule
Behaviour when value is erroneous, absent or unwanted (i.e. spurious) & [Description of components behaviour when value is erroneous, absent or unwanted (i.e. spurious)] \\
\bottomrule
\end{longtable}


\subsubsection{Outputs}\label{s:manage_track_data_outputs}

\paragraph{[Output 1 name]}

\begin{longtable}{p{.25\textwidth}p{.7\textwidth}}
\toprule
Output name				& [Name of the output] \\
\midrule
Description				& [Brief description of the output] \\
\midrule
Destination				& [Name of the destination component(s)] \\ 
\midrule
Type					& [Type of the output] \\
\midrule
Valid range of values	& [Complete list of valid values] \\
\midrule
Behaviour when value is at boundary	& [Description of components behaviour when output value is at boundary] \\
\midrule
Behaviour for values out of valid range	& [Description of components behaviour when output value is out of valid range] \\
\midrule
Behaviour when value is erroneous, absent or unwanted (i.e. spurious) & [Description of components behaviour when value is erroneous, absent or unwanted (i.e. spurious)] \\
\bottomrule
\end{longtable}


\paragraph{[Output 2 name]}

\begin{longtable}{p{.25\textwidth}p{.7\textwidth}}
\toprule
Output name				& [Name of the output] \\
\midrule
Description				& [Brief description of the output] \\
\midrule
Destination				& [Name of the destination component(s)] \\ 
\midrule
Type					& [Type of the output] \\
\midrule
Valid range of values	& [Complete list of valid values] \\
\midrule
Behaviour when value is at boundary	& [Description of components behaviour when output value is at boundary] \\
\midrule
Behaviour for values out of valid range	& [Description of components behaviour when output value is out of valid range] \\
\midrule
Behaviour when value is erroneous, absent or unwanted (i.e. spurious) & [Description of components behaviour when value is erroneous, absent or unwanted (i.e. spurious)] \\
\bottomrule
\end{longtable}


\subsection{Subcomponents}\label{s:manage_track_data_subcomponents}

\subsubsection{StoreRaw\_NV}
\input{sections/StoreRaw_NV.tex}

\subsubsection{Build\_GradientProfile}
\input{sections/Build_GradientProfile.tex}

\subsubsection{Build\_MA}
\input{sections/Build_MA.tex}

\subsubsection{Build\_MRSP}
\input{sections/Build_MRSP.tex}

\subsubsection{Manage\_EmergencyStop}
\input{sections/Manage_EmergencyStop.tex}

\subsubsection{C\_P003V1\_OBU\_P003\_OBU}
\input{sections/C_P003V1_OBU_P003_OBU.tex}

\subsubsection{GradientProfile\_to\_DMI}
\input{sections/GradientProfile_to_DMI.tex}

\subsubsection{Manage\_MA\_Request}
\input{sections/Manage_MA_Request.tex}

\subsubsection{TA\_to\_ML}
\input{sections/TA_to_ML.tex}

\subsubsection{SSP\_to\_MRSP}
\input{sections/SSP_to_MRSP.tex}

\subsubsection{MRSP\_to\_MRSP\_to\_DMI}
\input{sections/MRSP_to_MRSP_to_DMI.tex}

