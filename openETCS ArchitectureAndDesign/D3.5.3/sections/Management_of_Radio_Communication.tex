%set the master document for easy compilation
%!TEX root = ../D3_5_3.tex

\paragraph{Component Requirements}

\begin{longtable}{p{.25\textwidth}p{.7\textwidth}}
\toprule
Component name			& MoRC\_Main\_v2 (Management\_of\_Radio\_Communication) \\
\midrule
Link to SCADE model		& {\footnotesize \url{https://github.com/openETCS/modeling/tree/master/model/Scade/System/ObuFunctions/Radio/MoRC}} \\
\midrule
SCADE designer			& Uwe Steinke, Siemens \\
\midrule
Description				& 
The function \emph{MoRC\_Main\_v2} implements the session states establishing, maintaining and terminating as described in Subset-026, chap. 3.5. A SCADE state machine reflects this state model  accurately. Within each of the states, the activities needed as long as the state is active, are performed. \newline

\emph{MoRC\_Main\_v2} is related to exactly one of the radio mobile modems onboard, monitors its status and controls the processes of registration to the radio network, connecting to one RBC and establishing a radio session with the RBC. \emph{MoRC\_Main\_v2} communicates with its mobile modem directly via the API.  \newline

As the OBU is required to manage up to two RBCs,  two instances of \emph{MoRC\_Main\_v2} are used.  \newline

In addition, \emph{MoRC\_Main\_v2} generates the radio connection indication for the driver.

\\
\midrule
Input documents	& 
Subset-026, Chapter 3.5 \\
\midrule
Safety integrity level		& 4 \\
\midrule
Time constraints		& Implements several time delays, therefore appropriate clocking required \\
\midrule
API requirements 		& Interfaces to the OBUs mobile modem hardware via API \\
\bottomrule
\end{longtable}


\paragraph{Interface}

For an overview of the interface of this internal component we refer to the SCADE model (cf.~link above) respectively the SCADE generated documentation.