\section{ETCS Kernel Overview}\label{s:ETCS_Kernel_Overview}

The ETCS Kernel module consists of the 13 subcomponents F1.1 to F1.13 as depicted in Figure~\ref{f:ETCS_Kernel}. 
\begin{figure}
\center
\missingfigure{[Put SysML diagram of component here]}
\caption{F2: ETCS Kernel SysML diagram}\label{f:ETCS_Kernel}
\end{figure}
In the following we briefly describe the functionality of these subcomponents (for a more detailed description we refer to Sections~\ref{s:F2.1} to \ref{s:F2.13})
\begin{description}
\item[F2.1: Manage\_TrackSideInformation\_Integration] This component is responsible for receiving Eurobalise telegrams and Euroradio messages from the API and performs several consistency checks on the inputs.
\todo[inline]{to be checked}
\item[F2.2: Manage\_ETCS\_Procedures] This component describes the Start of Mission procedure of the train until the current status will change to another mode, level or other procedure.
\todo[inline]{descriptions needs to be improved}
\item[F2.3: trainData] Implementation of the train data with the corresponding interfaces to track, driver and RBC.
\todo[inline]{descriptions needs to be improved}
\item[F2.4: TrackAtlas] \todo[inline]{to be completed}
\item[F2.5: Mode\_and\_Level] \todo[inline]{to be completed}
\item[F2.6: calculateTrainPosition] \todo[inline]{to be completed}
\item[F2.7: SpeedSupervision\_Integration] \todo[inline]{to be completed}
\item[F2.8: Provide\_Position\_Report] \todo[inline]{to be completed}
\item[F2.9: Manage\_Radio\_Communication] \todo[inline]{to be completed}
\item[F2.10: ManageDMIInput] \todo[inline]{to be completed}
\item[F2.11: ManageDMIOutput] \todo[inline]{to be completed}
\item[F2.12: ManageTIUInput] \todo[inline]{to be completed}
\item[F2.13: ManageTIUOutput] \todo[inline]{to be completed}
\end{description}


\subsection{External Interfaces}
This section gives a detailed overview of the external inputs and outputs of module F2: ETCS Kernel.

\subsubsection{External Inputs}

\paragraph{[Input 1 name]}

\begin{longtable}{p{.25\textwidth}p{.7\textwidth}}
\toprule
Input name				& [Name of the input] \\
\midrule
Description				& [Brief description of the input] \\
\midrule
Source					& [Name of the source component] \\ 
\midrule
Type					& [Type of the input] \\
\midrule
Valid range of values	& [Complete list of valid values] \\
\midrule
Behaviour when value is at boundary	& [Description of components behaviour when input value is at boundary] \\
\midrule
Behaviour for values out of valid range	& [Description of components behaviour when input value is out of valid range] \\
\midrule
Behaviour when value is erroneous, absent or unwanted (i.e. spurious) & [Description of components behaviour when value is erroneous, absent or unwanted (i.e. spurious)] \\
\bottomrule
\end{longtable}

\paragraph{[Input 2 name]}

\begin{longtable}{p{.25\textwidth}p{.7\textwidth}}
\toprule
Input name				& [Name of the input] \\
\midrule
Description				& [Brief description of the input] \\
\midrule
Source					& [Name of the source component] \\ 
\midrule
Type					& [Type of the input] \\
\midrule
Valid range of values	& [Complete list of valid values] \\
\midrule
Behaviour when value is at boundary	& [Description of components behaviour when input value is at boundary] \\
\midrule
Behaviour for values out of valid range	& [Description of components behaviour when input value is out of valid range] \\
\midrule
Behaviour when value is erroneous, absent or unwanted (i.e. spurious) & [Description of components behaviour when value is erroneous, absent or unwanted (i.e. spurious)] \\
\bottomrule
\end{longtable}


\subsubsection{External Outputs}

\paragraph{[Output 1 name]}

\begin{longtable}{p{.25\textwidth}p{.7\textwidth}}
\toprule
Output name				& [Name of the output] \\
\midrule
Description				& [Brief description of the output] \\
\midrule
Destination				& [Name of the destination component(s)] \\ 
\midrule
Type					& [Type of the output] \\
\midrule
Valid range of values	& [Complete list of valid values] \\
\midrule
Behaviour when value is at boundary	& [Description of components behaviour when output value is at boundary] \\
\midrule
Behaviour for values out of valid range	& [Description of components behaviour when output value is out of valid range] \\
\midrule
Behaviour when value is erroneous, absent or unwanted (i.e. spurious) & [Description of components behaviour when value is erroneous, absent or unwanted (i.e. spurious)] \\
\bottomrule
\end{longtable}


\paragraph{[Output 2 name]}

\begin{longtable}{p{.25\textwidth}p{.7\textwidth}}
\toprule
Output name				& [Name of the output] \\
\midrule
Description				& [Brief description of the output] \\
\midrule
Destination				& [Name of the destination component(s)] \\ 
\midrule
Type					& [Type of the output] \\
\midrule
Valid range of values	& [Complete list of valid values] \\
\midrule
Behaviour when value is at boundary	& [Description of components behaviour when output value is at boundary] \\
\midrule
Behaviour for values out of valid range	& [Description of components behaviour when output value is out of valid range] \\
\midrule
Behaviour when value is erroneous, absent or unwanted (i.e. spurious) & [Description of components behaviour when value is erroneous, absent or unwanted (i.e. spurious)] \\
\bottomrule
\end{longtable}
