%set the master document for easy compilation
%!TEX root = ../D3_5_3.tex

\paragraph{Component Requirements}

\begin{longtable}{p{.25\textwidth}p{.7\textwidth}}
\toprule
Component name			& copyTrackDescription \\
\midrule
Link to SCADE model		& {\footnotesize \url{https://github.com/openETCS/modeling/tree/master/model/Scade/System/ObuFunctions/manageData/manageDMI}} \\
\midrule
SCADE designer			& Bernd Hekele, DB Netz AG \\
\midrule
Description				& in this module, the information about  will be processed and provided.
1-The point at which the driver needs to start braking to avoid intervention by the ETCS onboard equipment.\newline
2- The distance that the train has permission to travel\newline
3- The maximum speed which the train should not exceed\newline
4- The point at which the driver needs to start braking to avoid intervention by the ETCS onboard equipment. \\
\midrule
Input documents	& 
ERA ERTMS 015560\newline
ETCS DRIVER MACHINE INTERFACE\newline
ERSA API\\
\midrule
Safety integrity level	& 4 \\
\midrule
Time constraints		&  \todo[inline]{section and corresponding subsections have to be completed} \\
\midrule
API requirements 		& \todo[inline]{section and corresponding subsections have to be completed} \\
\bottomrule
\end{longtable}


\paragraph{Interface}

For an overview of the interface of this internal component we refer to the SCADE model (cf.~link above) respectively the SCADE generated documentation.