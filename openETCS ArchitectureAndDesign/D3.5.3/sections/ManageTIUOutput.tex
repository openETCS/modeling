%set the master document for easy compilation
%!TEX root = ../D3_5_3.tex

\section{F2.13: manageTIU\_output}\label{s:F2.13}


\subsection{Component Requirements}

\begin{longtable}{p{.25\textwidth}p{.7\textwidth}}
\toprule
Component name			& manageTIU\_output \\
\midrule
Link to SCADE model		& {\footnotesize \url{https://github.com/openETCS/modeling/tree/master/model/Scade/System/ObuFunctions/manageData/manageTIU}} \\
\midrule
SCADE designer			& Bernd Hekele, DB Netz AG \\
\midrule
Description				&  This component manages the outgoing messages to the Train Interface Unit (TIU). \\
\midrule
Input documents	& 
Alstom API\newline
Subset-034\\
\midrule
Safety integrity level	& 4 \\
\midrule
Time constraints		& n/a \\
\midrule
API requirements 		& n/a \\
\bottomrule
\end{longtable}


\subsection{Interface}

An overview of the interface of component manageTIU\_output is shown in Figure~\ref{f:manageTIUOutput}. For the description of inputs and outputs we refer to the SCADE Suite model (cf. link above)  respectively the SCADE generated documentation.
%The inputs and outputs are described in detail in Section~\ref{s:manageTIUOutput_inputs} respectively \ref{s:manageTIUOutput_outputs}. 
Subcomponents are described in Section~\ref{s:manageTIUOutput_subcomponents}.

\begin{figure}
\center
\includegraphics[width=\textwidth]{images/F2_13_manageTIU_output.pdf}
\caption{manageTIU\_output SysML diagram.}\label{f:manageTIUOutput}
\end{figure}


%\subsubsection{Inputs}\label{s:manageTIUOutput_inputs}
%
%\paragraph{[Input 1 name]}
%
%\begin{longtable}{p{.25\textwidth}p{.7\textwidth}}
%\toprule
%Input name				& [Name of the input] \\
%\midrule
%Description				& [Brief description of the input] \\
%\midrule
%Source					& [Name of the source component] \\ 
%\midrule
%Type					& [Type of the input] \\
%\midrule
%Valid range of values	& [Complete list of valid values] \\
%\midrule
%Behaviour when value is at boundary	& [Description of components behaviour when input value is at boundary] \\
%\midrule
%Behaviour for values out of valid range	& [Description of components behaviour when input value is out of valid range] \\
%\midrule
%Behaviour when value is erroneous, absent or unwanted (i.e. spurious) & [Description of components behaviour when value is erroneous, absent or unwanted (i.e. spurious)] \\
%\bottomrule
%\end{longtable}
%
%
%\paragraph{[Input 2 name]}
%
%\begin{longtable}{p{.25\textwidth}p{.7\textwidth}}
%\toprule
%Input name				& [Name of the input] \\
%\midrule
%Description				& [Brief description of the input] \\
%\midrule
%Source					& [Name of the source component] \\ 
%\midrule
%Type					& [Type of the input] \\
%\midrule
%Valid range of values	& [Complete list of valid values] \\
%\midrule
%Behaviour when value is at boundary	& [Description of components behaviour when input value is at boundary] \\
%\midrule
%Behaviour for values out of valid range	& [Description of components behaviour when input value is out of valid range] \\
%\midrule
%Behaviour when value is erroneous, absent or unwanted (i.e. spurious) & [Description of components behaviour when value is erroneous, absent or unwanted (i.e. spurious)] \\
%\bottomrule
%\end{longtable}
%
%
%\subsubsection{Outputs}\label{s:manageTIUOutput_outputs}
%
%\paragraph{[Output 1 name]}
%
%\begin{longtable}{p{.25\textwidth}p{.7\textwidth}}
%\toprule
%Output name				& [Name of the output] \\
%\midrule
%Description				& [Brief description of the output] \\
%\midrule
%Destination				& [Name of the destination component(s)] \\ 
%\midrule
%Type					& [Type of the output] \\
%\midrule
%Valid range of values	& [Complete list of valid values] \\
%\midrule
%Behaviour when value is at boundary	& [Description of components behaviour when output value is at boundary] \\
%\midrule
%Behaviour for values out of valid range	& [Description of components behaviour when output value is out of valid range] \\
%\midrule
%Behaviour when value is erroneous, absent or unwanted (i.e. spurious) & [Description of components behaviour when value is erroneous, absent or unwanted (i.e. spurious)] \\
%\bottomrule
%\end{longtable}
%
%
%\paragraph{[Output 2 name]}
%
%\begin{longtable}{p{.25\textwidth}p{.7\textwidth}}
%\toprule
%Output name				& [Name of the output] \\
%\midrule
%Description				& [Brief description of the output] \\
%\midrule
%Destination				& [Name of the destination component(s)] \\ 
%\midrule
%Type					& [Type of the output] \\
%\midrule
%Valid range of values	& [Complete list of valid values] \\
%\midrule
%Behaviour when value is at boundary	& [Description of components behaviour when output value is at boundary] \\
%\midrule
%Behaviour for values out of valid range	& [Description of components behaviour when output value is out of valid range] \\
%\midrule
%Behaviour when value is erroneous, absent or unwanted (i.e. spurious) & [Description of components behaviour when value is erroneous, absent or unwanted (i.e. spurious)] \\
%\bottomrule
%\end{longtable}

\subsection{Subcomponents}\label{s:manageTIUOutput_subcomponents}

The subcomponents of ManageTIUInput are not documented in this version of the architecture and design description document.

%\subsubsection{handleTraction}
%%set the master document for easy compilation
%!TEX root = ../D3_5_3.tex

\paragraph{Component Requirements}

\begin{longtable}{p{.25\textwidth}p{.7\textwidth}}
\toprule
Component name			& handleTraction \\
\midrule
Link to SCADE model		& {\footnotesize \url{https://github.com/openETCS/modeling/tree/master/model/Scade/System/ObuFunctions/manageData/manageTIU}} \\
\midrule
SCADE designer			& Bernd Hekele, DB Netz AG \\
\midrule
Description				&   ETCS is able to command the change of traction, the raising / lowering of an overhead pantograph and command the air tightness function to activate/deactivate where certain trackside data has been received. \\
\midrule
Input documents	& 
Alstom API\newline
Subset-034\\
\midrule
Safety integrity level	& 4 \\
\midrule
Time constraints		& \todo[inline]{section and corresponding subsections have to be completed} \\
\midrule
API requirements 		& \todo[inline]{section and corresponding subsections have to be completed} \\
\bottomrule
\end{longtable}


\paragraph{Interface}

For an overview of the interface of this internal component we refer to the SCADE model (cf.~link above) respectively the SCADE generated documentation.

%\subsubsection{manageTIU\_output}
%\input{sections/manageTIU_output.tex}