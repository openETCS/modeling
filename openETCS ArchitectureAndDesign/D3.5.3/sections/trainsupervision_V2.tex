%set the master document for easy compilation
%!TEX root = ../D3_5_3.tex

\section{Speed and Distance Monitoring}

\subsection{Component Requirements}

\begin{longtable}{p{.25\textwidth}p{.7\textwidth}}
\toprule
Component name			& SpeedSupervision_Integration (``SDM'')\\
\midrule
Link to SCADE model		& {\footnotesize \url{https://github.com/openETCS/modeling/tree/master/model/Scade/System/ObuFunctions/SpeedSupervison}} \\
\midrule
SCADE designer			& Benjamin Beichler, University of Rostock; Christian Stahl, TWT; Thorsten Schulz, University of Rostock \\
\midrule
Description				& The task of SDM is to monitor the speed of the train and the train location and as such to ensure that the speed remains within the given speed and distance limits. This block is based on \cite[Chapt.~3.13]{subset-026}.

The integration node ``SpeedSupervision\_Integration'' takes as input (1) movement related information such as train speed, train position and acceleration, (2) train related information such as brake information and train length, and (3) track related information such as speed and distance limits and national values.

Based on this information a speed profile is calculated. Speed restrictions create target speeds (targets) that have to be followed. For each such target braking curves are generated to supervise at which location of the track the train must apply the brake. In case of no target restrictions the train may accelerate to the supervised maximum speed of the speed profile. These calculations lead to commands being sent to the driver and the brake system.

The functionality is modeled using eight operators, as shown in Figure~\ref{f:ssv}, which are explained below.

The current status of the analysis of ``SDM'' and a functional breakdown can be found in a separate document, \verb+SpeedSupervision_analysis.pdf+.\\
\midrule
Input documents	& 
Subset-026, Chapter 3.13: Speed and distance monitoring \\
\midrule
Safety integrity level		& 4 \\
\midrule
Time constraints		& n/a \\
\midrule
API requirements 		& n/a \\
\bottomrule
\end{longtable}


\subsection{Interface}

An overview of the interface of component SpeedSupervision\_integration is shown in Figure~\ref{f:ssv}. The inputs and outputs are described in detail in Section~\ref{s:SDM_inputs} respectively \ref{s:SDM_outputs}. Sub components are described in Section~\ref{s:SDM_subcomponents}.

\begin{figure}
\centering
\includegraphics[width=0.95\textheight, angle=90]{../images/speedsupervision.png}
\caption{Structure of component SpeedSupervision\_Integration.}\label{f:ssv}
\end{figure}



\subsubsection{Inputs}\label{s:SDM_inputs}

\paragraph{NationalValues}

\begin{longtable}{p{.25\textwidth}p{.7\textwidth}}
\toprule
Input name				& NationalValues \\
\midrule
Description				& This input is packet 3 of \cite[Chapt.~8]{subset-026}, describing the national values.  \\
\midrule
Source					& Track Atlas Data \\ 
\midrule
Type					& P3\_NationalValues\_T \\
\midrule
Valid range of values	& P3\_NationalValues\_T is a complex data type \\
\midrule
Behaviour when value is at boundary	& as specified in SRS \\
\midrule
Behaviour for values out of valid range	& currently not checked \\
\midrule
Behaviour when value is erroneous, absent or unwanted (i.e. spurious) & not checked; node must not be called without reasonable National Value \\
\bottomrule
\end{longtable}


\paragraph{TrainPosition}

\begin{longtable}{p{.25\textwidth}p{.7\textwidth}}
\toprule
Input name				& TrainPosition \\
\midrule
Description				& This input is the current train position. \\
\midrule
Source					& Manage Track Data \\ 
\midrule
Type					& trainPosition\_T \\
\midrule
Valid range of values	& complex data type \\
\midrule
Behaviour when value is at boundary	& not checked, may overflow \\
\midrule
Behaviour for values out of valid range	& currently not checked \\
\midrule
Behaviour when value is erroneous, absent or unwanted (i.e. spurious) & not checked; node must not be called without reasonable position data \\
\bottomrule
\end{longtable}


\paragraph{odometry}

\begin{longtable}{p{.25\textwidth}p{.7\textwidth}}
\toprule
Input name				& odometry \\
\midrule
Description				& This input is the odometry data. \\
\midrule
Source					& Odometry \\ 
\midrule
Type					& odometry\_T \\
\midrule
Valid range of values	& complex data type \\
 & used fields are: \\
 & - acceleration: Obu\_BasicTypes\_Pkg::A\_internal\_Type. No valid range defined, neither checked. \\
 & - motionState: [noMotion | Motion] (enum type) \\
\midrule
Behaviour when value is at boundary	& possible overflow not evaluated \\
\midrule
Behaviour for values out of valid range	& not checked \\
\midrule
Behaviour when value is erroneous, absent or unwanted (i.e. spurious) & not handled, valid data is expected for valid function \\
\bottomrule
\end{longtable}


\paragraph{m\_level}

\begin{longtable}{p{.25\textwidth}p{.7\textwidth}}
\toprule
Input name				& m\_level \\
\midrule
Description				& This input is the current level of the train. (will be removed in next release!)\\
\midrule
Source					& Mode and Level \\ 
\midrule
Type					& M\_LEVEL \\
\midrule
Valid range of values	& enum type, valid range is ensured at compile time \\
\midrule
Behaviour when value is at boundary	& n/a \\
\midrule
Behaviour for values out of valid range	& n/a \\
\midrule
Behaviour when value is erroneous, absent or unwanted (i.e. spurious) & n/a \\
\bottomrule
\end{longtable}


\paragraph{trainProps}

\begin{longtable}{p{.25\textwidth}p{.7\textwidth}}
\toprule
Input name				& trainProps \\
\midrule
Description				& This input is a set of train related properties. \\
\midrule
Source					& Database \\ 
\midrule
Type					& trainProperties\_T \\
\midrule
Valid range of values	& complex type \\
 & used fields are: \\
 & d_baliseAntenna_2_frontend.nominal: Obu\_BasicTypes\_Pkg::L\_internal\_Type No valid range defined, neither checked. \\
\midrule
Behaviour when value is at boundary	& n/a \\
\midrule
Behaviour for values out of valid range	& n/a \\
\midrule
Behaviour when value is erroneous, absent or unwanted (i.e. spurious) & Value is only evaluated in Level 1. Low values (e.g. invalid-default 0) will lead to early trip, brake and alike. Larger values will lead to late braking, possibly numeric overflow. \\
\bottomrule
\end{longtable}


\paragraph{MRSP}

\begin{longtable}{p{.25\textwidth}p{.7\textwidth}}
\toprule
Input name				& MRSP \\
\midrule
Description				& This input is the most restrictive speed profile. \\
\midrule
Source					& Track Atlas \\ 
\midrule
Type					& MRSP\_Profile\_t \\
\midrule
Valid range of values	& complex type \\
\midrule
Behaviour when value is at boundary	& n/a \\
\midrule
Behaviour for values out of valid range	& n/a \\
\midrule
Behaviour when value is erroneous, absent or unwanted (i.e. spurious) & at least one valid entry is expected (the maximum vehicle speed), else a trip shall be commanded. \\
\bottomrule
\end{longtable}


\paragraph{MA}

\begin{longtable}{p{.25\textwidth}p{.7\textwidth}}
\toprule
Input name				& MA \\
\midrule
Description				& This input is a movement authority. \\
\midrule
Source					& ??? \\ 
\midrule
Type					& MAs\_t \\
\midrule
Valid range of values	& complex type \\
\midrule
Behaviour when value is at boundary	& n/a \\
\midrule
Behaviour for values out of valid range	& n/a \\
\midrule
Behaviour when value is erroneous, absent or unwanted (i.e. spurious) & on .valid = false, brake/trip should be commanded. \\
\bottomrule
\end{longtable}


\paragraph{MA\_updated}

\begin{longtable}{p{.25\textwidth}p{.7\textwidth}}
\toprule
Input name				& MA\_updated \\
\midrule
Description				& This flag is true if the movement authority has been updated in this clock cycle and false otherwise. \\
\midrule
Source					& internal \\ 
\midrule
Type					& bool \\
\midrule
Valid range of values	& \[false, true\] \\
\midrule
Behaviour when value is at boundary	& limited range, check at compile time \\
\midrule
Behaviour for values out of valid range	& limited range, check at compile time \\
\midrule
Behaviour when value is erroneous, absent or unwanted (i.e. spurious) & n/a \\
\bottomrule
\end{longtable}


\paragraph{MRSP\_updated}

\begin{longtable}{p{.25\textwidth}p{.7\textwidth}}
\toprule
Input name				& MRSP\_updated \\
\midrule
Description				& This flag is true if the most restrictive speed profile has been updated in this clock cycle and false otherwise. \\
\midrule
Source					& internal \\ 
\midrule
Type					& bool \\
\midrule
Valid range of values	& \[false, true\] \\
\midrule
Behaviour when value is at boundary	& limited range, check at compile time \\
\midrule
Behaviour for values out of valid range	& limited range, check at compile time \\
\midrule
Behaviour when value is erroneous, absent or unwanted (i.e. spurious) & n/a \\
\bottomrule
\end{longtable}


\subsubsection{Outputs}\label{s:SDM_outputs}

\paragraph{sdmToDMI}

\begin{longtable}{p{.25\textwidth}p{.7\textwidth}}
\toprule
Output name				& sdmToDMI \\
\midrule
Description				& This output contains information about different speeds and positions, on the one hand and the current supervision status, on the other hand. This information shall be displayed to the driver. \\
\midrule
Destination				& DMI \\ 
\midrule
Type					& speedSupervisionForDMI\_T \\
\midrule
Valid range of values	& n/a \\
\midrule
Behaviour when value is at boundary	& n/a \\
\midrule
Behaviour for values out of valid range	& n/a \\
\midrule
Behaviour when value is erroneous, absent or unwanted (i.e. spurious) & .valid can be false \\
\bottomrule
\end{longtable}


\paragraph{target}

\begin{longtable}{p{.25\textwidth}p{.7\textwidth}}
\toprule
Output name				& target \\
\midrule
Description				& This output is the most restrictive displayed target (MRDT). \\
\midrule
Destination				& [Name of the destination component(s)] \\ 
\midrule
Type					& Target\_T \\
\midrule
Valid range of values	& n/a \\
\midrule
Behaviour when value is at boundary	& n/a \\
\midrule
Behaviour for values out of valid range	& n/a \\
\midrule
Behaviour when value is erroneous, absent or unwanted (i.e. spurious) & .valid may be false if no target is supervised or known, other values of this output must be ignored then. \\
\bottomrule
\end{longtable}


\paragraph{sdmCommands}

\begin{longtable}{p{.25\textwidth}p{.7\textwidth}}
\toprule
Output name				& sdmCommands \\
\midrule
Description				& This output gives some intermediate results of operator SDM\_Commands. It is currently used for test purposes only. \\
\midrule
Destination				& [Name of the destination component(s)] \\ 
\midrule
Type					& SDM\_Commands\_T \\
\midrule
Valid range of values	& n/a \\
\midrule
Behaviour when value is at boundary	& n/a \\
\midrule
Behaviour for values out of valid range	& n/a \\
\midrule
Behaviour when value is erroneous, absent or unwanted (i.e. spurious) & overall .valid is always set, individual speeds have their corresponding valid flag. \\
\bottomrule
\end{longtable}

\paragraph{brakeCmd}

\begin{longtable}{p{.25\textwidth}p{.7\textwidth}}
\toprule
Output name				& brakeCmd \\
\midrule
Description				& This output is the brake command, indicating whether performing the service brake or the emergency brake have been commanded. \\
\midrule
Destination				& [Name of the destination component(s)] \\ 
\midrule
Type					& Brake\_command\_T (enum)\\
\midrule
Valid range of values	& [brake_signal_command_not_defined, apply_brake, release_brake] \\
\midrule
Behaviour when value is at boundary	& n/a \\
\midrule
Behaviour for values out of valid range	& n/a \\
\midrule
Behaviour when value is erroneous, absent or unwanted (i.e. spurious) & n/a \\
\bottomrule
\end{longtable}


\paragraph{EOA\_overpassed}

\begin{longtable}{p{.25\textwidth}p{.7\textwidth}}
\toprule
Output name				& EOA\_overpassed \\
\midrule
Description				& This output is true if the end of authority has been overpassed and false otherwise. \\
\midrule
Destination				& [Name of the destination component(s)] \\ 
\midrule
Type					& bool \\
\midrule
Valid range of values	& [false, true] \\
\midrule
Behaviour when value is at boundary	& n/a \\
\midrule
Behaviour for values out of valid range	& n/a \\
\midrule
Behaviour when value is erroneous, absent or unwanted (i.e. spurious) & n/a \\
\bottomrule
\end{longtable}


\paragraph{Target\_Speed\_Reached}

\begin{longtable}{p{.25\textwidth}p{.7\textwidth}}
\toprule
Output name				& Target\_Speed\_Reached \\
\midrule
Description				& This output is true if the current speed is greater than or equal the target speed and false otherwise. \\
\midrule
Destination				& [Name of the destination component(s)] \\ 
\midrule
Type					& bool \\
\midrule
Valid range of values	& [false/true] \\
\midrule
Behaviour when value is at boundary	& n/a \\
\midrule
Behaviour for values out of valid range	& n/a \\
\midrule
Behaviour when value is erroneous, absent or unwanted (i.e. spurious) & n/a \\
\bottomrule
\end{longtable}


\subsection{Sub Components}\label{s:SDM_subcomponents}

\subsubsection{Receive\_TrackSide\_Msg}
%set the master document for easy compilation
%!TEX root = ../D3_5_3.tex

\paragraph{Component Requirements}

\begin{longtable}{p{.25\textwidth}p{.7\textwidth}}
\toprule
Component name			& SDM\_InputWrapper \\
\midrule
Link to SCADE model		& {\footnotesize \url{https://github.com/openETCS/modeling/tree/master/model/Scade/System/ObuFunctions/SpeedSupervison/SpeedSupervision_Integration}} \\
\midrule
SCADE designer			& Benjamin Beichler, University of Rostock\newline
Thorsten Schulz, University of Rostock \\
\midrule
Description				& The motivation for this operator is to convert all inputs of SDM that contain information about length, speed, distance, and acceleration defined as integer into \texttt{real} to allow automatically the highest precision in the calculations by the meaning of floating point operations. In addition, to ease the modeling, inside block ``Speed Supervision'' only units meters ($[m]$), seconds($[s]$), meters per second($[\frac{m}{s}]$), and meters per square second($[\frac{m}{s^{2}}]$) are used.

This operator forwards input messages, takes data from complex data types or transforms inputs messages into an internal type thereby converting int to real. \\
\midrule
Input documents	& 
Subset-026, Chapter 3.13, (not specific, helper function)\\
\midrule
Safety integrity level		& 4 \\
\midrule
Time constraints		& n/a \\
\midrule
API requirements 		& n/a \\
\bottomrule
\end{longtable}


\paragraph{Interface}

For an overview of the interface of this internal component we refer to the SCADE model (c.f.~link above) respectively the SCADE generated documentation.

\subsubsection{TargetManagement}
%set the master document for easy compilation
%!TEX root = ../D3_5_3.tex

\paragraph{Component Requirements}

\begin{longtable}{p{.25\textwidth}p{.7\textwidth}}
\toprule
Component name			& TargetManagement \\
\midrule
Link to SCADE model		& {\footnotesize \url{https://github.com/openETCS/modeling/tree/master/model/Scade/System/ObuFunctions/SpeedSupervison/TargetManagement}} \\
\midrule
SCADE designer			&  Benjamin Beichler, University of Rostock\newline
Thorsten Schulz, University of Rostock \\
\midrule
Description				& This operator calculates/updates the list of targets to be supervised by SDM. Taking the current movement authority, the most restrictive speed profile and the current maximum safe front end position as an input, the operator outputs a list of MRSP-Targets and a Limit-of-Authority-Target (LoA) or, if an End of Authority is known, the End-of-Authority-Target (EoA) and the Supervised Location (SvL). Since LoA and SvL are mutually exclusive but both result in a trip-target, they use the same flow.

\subparagraph*{Derivation of Targets from Movement Authority}
The \emph{Movement Authority} creates three types of targets:
\begin{description}
\item[Limit of Authority (LoA)] if the End of Authority is not yet known to the RBC and is accompanied by a speed limit for the given location.
\item[End Of Authority (EoA)] requires the train to stop and creates a Service Brake Target
\item[Supervised Location (SvL)] is derived from the EoA but results in an emergency brake target and on passing in a trip. The SvL may be offset from the EoA to the Overlap (OL) or the Dangerpoint (DP).
\end{description}

\subparagraph*{Derivation of Targets from MRSP}
According to \cite[Chapt.~3.13.8.2]{subset-026}, every speed decrease of the MRSP is used to derive a target. Therefore in every cycle in which the MRSP is updated, the operator iterates through the entire MRSP searching for all MRSP targets. For this purpose, every element of the MRSP is compared with its successor.

\subparagraph*{Update of Targets}
In every cycle the operator monitors whether all targets are already passed. To this end, it iterates over the list of targets comparing the current front end position with the target's location. \\
\midrule
Input documents	& 
Subset-026, Chapter 3.13.8.2: Determination of the supervised targets \\
\midrule
Safety integrity level		& 4 \\
\midrule
Time constraints		& n/a \\
\midrule
API requirements 		& n/a \\
\bottomrule
\end{longtable}


\paragraph{Interface}

For an overview of the interface of this internal component we refer to the SCADE model (c.f.~link above) respectively the SCADE generated documentation.

\subsubsection{CalcBrakingCurves\_Integration}
%set the master document for easy compilation
%!TEX root = ../D3_5_3.tex

\paragraph{Component Requirements}

\begin{longtable}{p{.25\textwidth}p{.7\textwidth}}
\toprule
Component name			& CalcBrakingCurves\_Integration \\
\midrule
Link to SCADE model		& {\footnotesize \url{???}} \\
\midrule
SCADE designer			& Christian Stahl, TWT \\
\midrule
Description				& For each type of target a certain braking curve has to be calculated. This curve enables proactive monitoring of the train's speed. A reverse lookup on this braking curve indicates, where the train has to start braking given the current speed. The braking curve does not depend on the actual train status. As a consequence the braking curve stays constant over time. As a legitimate simplification the calculation of the braking curve is not extended after the estimated front end position of the train has been passed. \\
\midrule
Input documents	& 
Subset-026, Chapter 3.13.8.3: Emergency Brake Deceleration curves (EBD)\newline
Subset-026, Chapter 3.13.8.4: Service Brake Deceleration curves (SBD)\newline
Subset-026, Chapter 3.13.8.5: Guidance curves (GUI) \\
\midrule
Safety integrity level		& 4 \\
\midrule
Time constraints		& [If applicable description of time constraints, otherwise n/a] \\
\midrule
API requirements 		& [If applicable description of API requirements, otherwise n/a] \\
\bottomrule
\end{longtable}


\paragraph{Interface}

For an overview of the interface of this internal component we refer to the SCADE model (c.f.~link above) respectively the SCADE generated documentation.

\subsubsection{SDMLimitLocations}
%set the master document for easy compilation
%!TEX root = ../D3_5_3.tex

\paragraph{Component Requirements}

\begin{longtable}{p{.25\textwidth}p{.7\textwidth}}
\toprule
Component name			& SDMLimitLocations \\
\midrule
Link to SCADE model		& {\footnotesize \url{???}} \\
\midrule
SCADE designer			& ??? \\
\midrule
Description				& This operator calculates the various locations needed to determine the speed and distance monitoring commands. The current implementation of functionality is stateless and requires a complete recalculation each cycle. 

This operator gathers all necessary input values and computes some frequently used intermediate values in the operators \texttt{surplusTractionDeltas} and \texttt{$v_{\mathit{bec}}$}. The other input preparation operator is the \texttt{TargetSelector} whose main task is to dissect the list of targets to find the Most Restrictive Target. The accompanying braking curves are extracted and promoted to trailing location calculations. Also the special values of the EOA are exposed.

The operator creates the requested values for the commands package. These are in particular the preindication locations for EBD and SBD based targets, the release speed monitoring start locations, the locations for target speed monitoring of the I-, W-, P- and FLOI-curve, the related FLOI speed and the location of the permitted speed supervision limit. Included in the output are also certain flags for the validity of linked values.\\
\midrule
Input documents	& 
Subset-026, Chapter 3.13.9: Supervision Limits \newline
Subset-026, Chapter 5.3.1.2: $f_{41}$ -- accuracy of speed known on-board\newline
Subset-026, Chapter 3.13.10: Monitoring Commands as reference for required outputs of this module \\
\midrule
Safety integrity level		& 4 \\
\midrule
Time constraints		& [If applicable description of time constraints, otherwise n/a] \\
\midrule
API requirements 		& [If applicable description of API requirements, otherwise n/a] \\
\bottomrule
\end{longtable}


\paragraph{Interface}

For an overview of the interface of this internal component we refer to the SCADE model (c.f.~link above) respectively the SCADE generated documentation.

\subsubsection{CalcSpeeds}
\input{sections/CalcSpeeds.tex}

\subsubsection{ReleaseSpeed\_Selection}
%set the master document for easy compilation
%!TEX root = ../D3_5_2.tex

\paragraph{Component Requirements}

\begin{longtable}{p{.25\textwidth}p{.7\textwidth}}
\toprule
Component name			& ReleaseSpeed\_Selection \\
\midrule
Link to SCADE model		& {\footnotesize \url{???}} \\
\midrule
SCADE designer			& ??? \\
\midrule
Description				&This operator outputs the release speed which can be given either by the national values or the movement authority. This operator will be integrated into other operators in the next iteration.\\
\midrule
Input documents	& 
Subset-026, Chapter 3.8: Movement authority \\
\midrule
Safety integrity level		& 4 \\
\midrule
Time constraints		& [If applicable description of time constraints, otherwise n/a] \\
\midrule
API requirements 		& [If applicable description of API requirements, otherwise n/a] \\
\bottomrule
\end{longtable}


\paragraph{Interface}

For an overview of the interface of this internal component we refer to the SCADE model (c.f.~link above) respectively the SCADE generated documentation.

\subsubsection{SDM\_Commands}
%set the master document for easy compilation
%!TEX root = ../D3_5_3.tex

\paragraph{Component Requirements}

\begin{longtable}{p{.25\textwidth}p{.7\textwidth}}
\toprule
Component name			& SDM\_Commands \\
\midrule
Link to SCADE model		& {\footnotesize \url{https://github.com/openETCS/modeling/tree/master/model/Scade/System/ObuFunctions/SpeedSupervison/SDM_Commands}} \\
\midrule
SCADE designer			& Christian Stahl, TWT\newline
Thorsten Schulz, University of Rostock \\
\midrule
Description				& This operator models the speed and distance monitoring commands. More precisely, it triggers the service or emergency brake and outputs the current supervision status of the OBU together with information on speeds and locations to the driver.

The OBU can be in any of three types of speed and distance monitoring modes: ceiling speed monitoring, release speed monitoring and target speed monitoring. We use a state machine to model the switching between the three modes: each state models a mode and a transition between to states is enabled if the condition two switch between the two corresponding modes is evaluated to true. In each mode, the OBU can be in up to five different supervision stati. The behavior of changing from one status to another is also modeled as a state machine. As a result, the model is a hierarchical state machine.\\
\midrule
Input documents	& 
Subset-026, Chapter 3.13.10: Speed and distance monitoring commands \\
\midrule
Safety integrity level		& 4 \\
\midrule
Time constraints		& n/a \\
\midrule
API requirements 		& n/a \\
\bottomrule
\end{longtable}


\paragraph{Interface}

For an overview of the interface of this internal component we refer to the SCADE model (cf.~link above) respectively the SCADE generated documentation.

\subsubsection{SDM\_OutputWrapper}
%set the master document for easy compilation
%!TEX root = ../D3_5_3.tex

\paragraph{Component Requirements}

\begin{longtable}{p{.25\textwidth}p{.7\textwidth}}
\toprule
Component name			& SDM\_OutputWrapper \\
\midrule
Link to SCADE model		& {\footnotesize \url{https://github.com/openETCS/modeling/tree/master/model/Scade/System/ObuFunctions/SpeedSupervison/SpeedSupervision_Integration}} \\
\midrule
SCADE designer			& Benjamin Beichler,University of Rostock\newline
Thorsten Schulz, University of Rostock \\
\midrule
Description				& This operator is the counterpart to operator SDM\_OutputWrapper---that is, it converts all internal outputs of SDM that contain information about length, speed, distance, and acceleration defined as real into int, such that all other blocks can stick to their types and also performs the calculation into units used by the environment.

This operator forwards input messages and transforms inputs messages into an internal type thereby converting real to int.\\
\midrule
Input documents	& 
Subset-026, Chapter 3.13.10: Speed and distance monitoring commands\\
\midrule
Safety integrity level	& 4 \\
\midrule
Time constraints		& n/a \\
\midrule
API requirements 		& n/a \\
\bottomrule
\end{longtable}



\paragraph{Interface}

For an overview of the interface of this internal component we refer to the SCADE model (cf.~link above) respectively the SCADE generated documentation.

