%set the master document for easy compilation
%!TEX root = ../D3_5_2.tex

\section{DMI}

\subsection{Component Requirements}

\begin{longtable}{p{.25\textwidth}p{.7\textwidth}}
\toprule
Component name			& DMI \\
\midrule
Link to SCADE model		& {\footnotesize \url{https://github.com/openETCS/modeling/tree/master/model/Scade/System/DMI_Control}} \\
\midrule
SCADE designer			& Valerio D'Angelo, DB Netz AG \\
\midrule
Description				& The DMI controller interact with the DMI display and is responsible for alls procedures between the DMI display and Driver. Furthermore, the DMI controller will interact with the DMI Management to compute the received information (e.g. driver number request, ...) and send, if necessary, data or reports to the DMI Management (acknowledge, text messages...). The DMI Controller is a passive module, this means that all the processing are performed EVC-side, therefore the DMI Controller simply responds to the requests of the EVC or Driver and performs some checks according with the information received from EVC. \\
\midrule
Input documents	& 
ERA\_ERTMS\_015560 \\
\midrule
Safety integrity level	& 4 \\
\midrule
Time constraints		& [If applicable description of time constraints, otherwise n/a] \\
\midrule
API requirements 		& [If applicable description of API requirements, otherwise n/a] \\
\bottomrule
\end{longtable}


\subsection{Interface}

An overview of the interface of component Mode\_and\_Level is shown in Figure~\ref{f:DMI_interface}. The inputs and outputs are described in detail in Section~\ref{s:DMI_inputs} respectively \ref{s:DMI_outputs}.

\begin{figure}
\center
\includegraphics[width=.8\textwidth]{images/DMI_Interfaces}
\caption{Component SysML diagram}\label{f:DMI_interface}
\end{figure}


\subsubsection{Inputs}\label{s:DMI_inputs}

\paragraph{[Input 1 name]}

\begin{longtable}{p{.25\textwidth}p{.7\textwidth}}
\toprule
Input name				& [Name of the input] \\
\midrule
Description				& [Brief description of the input] \\
\midrule
Source					& [Name of the source component] \\ 
\midrule
Type					& [Type of the input] \\
\midrule
Valid range of values	& [Complete list of valid values] \\
\midrule
Behaviour when value is at boundary	& [Description of components behaviour when input value is at boundary] \\
\midrule
Behaviour for values out of valid range	& [Description of components behaviour when input value is out of valid range] \\
\bottomrule
\end{longtable}


\paragraph{[Input 2 name]}

\begin{longtable}{p{.25\textwidth}p{.7\textwidth}}
\toprule
Input name				& [Name of the input] \\
\midrule
Description				& [Brief description of the input] \\
\midrule
Source					& [Name of the source component] \\ 
\midrule
Type					& [Type of the input] \\
\midrule
Valid range of values	& [Complete list of valid values] \\
\midrule
Behaviour when value is at boundary	& [Description of components behaviour when input value is at boundary] \\
\midrule
Behaviour for values out of valid range	& [Description of components behaviour when input value is out of valid range] \\
\bottomrule
\end{longtable}


\subsubsection{Outputs}\label{s:DMI_outputs}

\paragraph{[Output 1 name]}

\begin{longtable}{p{.25\textwidth}p{.7\textwidth}}
\toprule
Output name				& [Name of the output] \\
\midrule
Description				& [Brief description of the output] \\
\midrule
Destination				& [Name of the destination component(s)] \\ 
\midrule
Type					& [Type of the output] \\
\midrule
Valid range of values	& [Complete list of valid values] \\
\midrule
Behaviour when value is at boundary	& [Description of components behaviour when output value is at boundary] \\
\midrule
Behaviour for values out of valid range	& [Description of components behaviour when output value is out of valid range] \\
\bottomrule
\end{longtable}


\paragraph{[Output 2 name]}

\begin{longtable}{p{.25\textwidth}p{.7\textwidth}}
\toprule
Output name				& [Name of the output] \\
\midrule
Description				& [Brief description of the output] \\
\midrule
Destination				& [Name of the destination component(s)] \\ 
\midrule
Type					& [Type of the output] \\
\midrule
Valid range of values	& [Complete list of valid values] \\
\midrule
Behaviour when value is at boundary	& [Description of components behaviour when output value is at boundary] \\
\midrule
Behaviour for values out of valid range	& [Description of components behaviour when output value is out of valid range] \\
\bottomrule
\end{longtable}


\subsection{Sub Components}

%\subsubsection{Management\_of\_Radio\_Communication}
%%set the master document for easy compilation
%!TEX root = ../D3_5_3.tex

\paragraph{Component Requirements}

\begin{longtable}{p{.25\textwidth}p{.7\textwidth}}
\toprule
Component name			& Management\_of\_Radio\_Communication \\
\midrule
Link to SCADE model		& {\footnotesize \url{https://github.com/openETCS/modeling/tree/master/model/Scade/System/ObuFunctions/Radio/MoRC}} \\
\midrule
SCADE designer			& Uwe Steinke, Siemens \\
\midrule
Description				& 
\todo[inline]{to be checked}
The function \emph{managementOfRadioCommunication} implements the session states establishing, maintaining and termination as described in Subset-026, chap. 3.5. A SCADE state machine reflects this state model (Figure ???) accurately. Within each of the states, the activities needed as long as the state is active, are performed. When there is no communication session (state \emph{NoSession}) currently, the state machine waits for events that initiate a session (subfunction \emph{initiate\_a\_Session}). When the appropriate conditions are fulfilled, the state machine moves to the \textit{Establishing} state. Here in, it runs through the sequence required fore establishing a session (subfunction \emph{establish\_a\_Session}. Dependent on the results, the state machine changes over to the \emph{Maintaining} or \emph{Terminating} state. While in \emph{Maintaining}, the communication connection is monitored. When an event triggering the session termination occurs, the state machine switches to the state \emph{Terminating} with the subfunction \emph{terminating\_a\_CommunicationSession} and performs the session termination sequence. 

In parallel to the main state machine, \emph{managementOfRadioCommunication} monitors all the time whether the session has to be terminated (subfunction \emph{initiateTerminatingASession}) or if the session has the be terminated and subsequently established (subfunction \emph{terminateAndEstablishSession}). \emph{registeringToTheRadioNetwork} is responsible for connection to the radio network. \emph{safeRadioConnectionIndication} controls the radio connection indication for the driver.\\
\midrule
Input documents	& 
Subset-026, Chapter 3.5 \\
\midrule
Safety integrity level		& 4 \\
\midrule
Time constraints		& [If applicable description of time constraints, otherwise n/a] \\
\midrule
API requirements 		& [If applicable description of API requirements, otherwise n/a] \\
\bottomrule
\end{longtable}


\paragraph{Interface}

For an overview of the interface of this internal component we refer to the SCADE model (cf.~link above) respectively the SCADE generated documentation.


