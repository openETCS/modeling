\documentclass{template/openetcs_report}
% Use the option "nocc" if the document is not licensed under Creative Commons
%\documentclass[nocc]{template/openetcs_article}
\usepackage{lipsum,url}
\usepackage{supertabular}
\usepackage{multirow}
\usepackage{color, colortbl}
\usepackage{hyperref}
\usepackage{listings}
\usepackage{makeidx}
\definecolor{gray}{rgb}{0.8,0.8,0.8}
\usepackage[modulo]{lineno}
\usepackage{float}
\usepackage{fixme}
\usepackage{pdflscape}
\usepackage[acronym, % list of acronyms
  %section, % add the glossary to the table of content
            %description,% acronyms have a user-supplied description,
 style=longheader, % table style
 nonumberlist % no page number
  ]{glossaries}

\graphicspath{{./template/}{.}{./images/}}

\renewcommand*{\glspostdescription}{} %Deactivate point at the end of every description
\renewcommand*{\glossaryname}{Glossary}

%create glossary
% \makeglossaries
 %Glossary terms
% \loadglsentries{glossary}

\begin{document}
\frontmatter
\project{openETCS}

\newcommand{\define}[1]{\index{#1}\emph{#1}}



%Please do not change anything above this line
%============================

% The document metadata is defined below

%assign a report number here
\reportnum{OETCS/WP3/D3.5.0}

%define your workpackage here
\wp{Work Package 3: ``Modeling''}

%set a title here
\title{openETCS System Architecture and Design Specification}

%set the date of the report here
\date{May 2015}


%document approval
%define the name and affiliation of the people involved in the documents approbation here
\creatorname{Baseliyos Jacob}
\creatoraffil{DB Netz AG}

\techassessorname{Jan Welte}
\techassessoraffil{Technische Universität Braunschweig}

\qualityassessorname{Izaskun de la Torre}
\qualityassessoraffil{SQS}

\approvalname{Klaus-R\"udiger Hase}
\approvalaffil{DB Netz}


%define a list of authors and their affiliation here

\author{Baseliyos Jacob, Peter Mahlmann}
\affiliation{DB Netz AG}




% define the coverart
\coverart[width=350pt]{openETCS_EUPL}

\newpage
%define the type of report
\reporttype{Architecture and Design Specification}


\begin{abstract}
%define an abstract here
This document gives an introduction to the architecture of openETCS. The functional scope is tailored to cover the functionality required for the openETCS demonstration as an objective of the ITEA2 project. The goal is to develop a formal model and to demonstrate the functionality during a proof of concept on the ETCS Level 2 Utrecht Amsterdam track with real scenarios. It has to be read as a complement to the models in SysML and Scade languages. 
\end{abstract}

%=============================
\maketitle

%Modification history
%if you do not need a modification history table for your document simply comment out the eight lines below
%=============================


\chapter*{Modification History}
\tablefirsthead{
\hline 
\rowcolor{gray} 
Version & Section & Modification / Description & Author & Date \\\hline}
\begin{supertabular}{| m{1.2cm} | m{1.5cm} | m{4.0cm} | m{3.5cm} | m{3.5cm} |}
0.1 & Document & Initial document providing structure & Peter Mahlmann& \\\hline

\end{supertabular}

% list subsubsections in table of contents
\setcounter{tocdepth}{3}


\tableofcontents
\listoffiguresandtables
\newpage
%=============================

%Uncomment the next line if you need line numbers for tracebility when the document is in review
%\linenumbers
%=============================


% The actual document starts below this line
%=============================

\mainmatter

\chapter{Introduction}

%set the master document for easy compilation
%!TEX root = ../D3_5_3.tex

\chapter{Purpose of the document}

This document is managed as a deliverable of the modeling work package with denomination ~D3.7.x, and contains advices and recommendation for the design of a physical system architecture.  

The development of the functional model is done iteratively increasing the scope in steps, the last digit of the deliverable identifier, i.e.~x, denotes the release of the model to which it applies. If the functional model requires to update the system architecture a consistent version number will be applied to this document as required by the Model release version.

This document complements the indications contained in the API requirements specification and the documentation derived from this as the generic openETCS Application Programming Interface (API), available at \url{https://github.com/openETCS/modeling/blob/master/API/description/api-description.pdf}. \cite{alstom-api}

\section{Input Documents}

The following documents provide a context for the system perspective.

\begin{itemize}
	\item ERA Subset-026 \cite{subset-026}, V3.3.0
	\item ERA TSI CCS Documents
	\item openETCS API documentation, available at \url{https://github.com/openETCS/modeling/blob/master/API/description/api-description.pdf} \cite{alstom-api}\cite{alstom-api-app-layer}\cite{alstom-api-data-dict}
	%\item openETCS requirements, i.e.~D2.1, D2.2,$\ldots$, %D2.9, available at %\url{https://github.com/openETCS/requirements/tree/maste%r/Reference}
\end{itemize}




\chapter{Introduction}

Designing a sub system integrable with the train borne system is a complex task. The designer faces a large variety of serious challenges and design complexities. 

Before the functions are actually implemented, a system architect will have to select an appropriate hardware-software concept out of the large number of available boards, controllers, network and  bus constraints.  He  will as well include robustness criteria against environmental influences. 

Memories, operating systems, drivers, generic and application software segregation as well as selection criteria for sensors and actuators need to be correctly assessed. 

The target architecture has to meet a large variety of requirements. Criteria of timing, Bus bandwidth, processor and peripheral performance, memory size, safety principles and possible processing or data transfer bottlenecks. Environmental conditions, timing constraints, robustness against specific interferences shall constantly be tracked.

Power requirement as well as allocation of availability, maintainability figures to enumerate only the most relevant items accompany all the design phases.

On top of this a specific vital architecture has to be selected and the required integrity level has to be granted. The relative safety constraints have to be assured and maybe exported.

Selecting the components matching these is a critical phase. Over-dimensioning the architecture may impact on cost factors relevant for the market access of the system. Under-dimensioning the architecture design could result in not achieving performance constraints, thus compromising system quality and suitability. Early architectural choices have a dominant impact on the success of the new system. 

The system architects will commit to efficient design choices according to the target project margins and all this within the frame of a defined project delivery time schedule. 

Due to the fact that the design verification phase, requiring to have completed all the integration steps, may be very late in the release process a high precision during the system architecture design is mandatory. 

Therefore highly experienced System Designer are considered as the key factor for a reliable achievement of expected design result.

A primary goal of the openETCS ITEA2 project is to provide a formal specification and a model of an ETCS onboard functionality according to the specification defined in Subset-026 \cite{subset-026} by the European Railway Agency (ERA). 

The Model-Based Development process is an approach that allows engineers to specify the behavior of a system and to simulate and execute it in a very early development stage.

Once a model-based development process has been established, engineers should be able to apply new technologies and tools to enhance and shorten product development cycles,
e.g. by introducing generation of Model Validation test cases and target Code directly from the model. This enables to improve the V based development process to save development time and effort while preserving or improving the dependability of the developed systems. 


\begin{figure}[H]
	\center
	\includegraphics[width= 0.9\textwidth]{Y_process.pdf}
	\caption{SRS modeling cycle}\label{Y_process}
\end{figure}


The methodology makes it easier to understand requirements and increases the correctness of the requirements, the correctness of the design and the code with respect to the requirements. An integration of system-level and design-level modeling tools allows a virtually integrated V-process that is sharpened up to a Y-based process with the required steps at the bottom of the former V being considerably automated (see figure\ref{Y_process} )

Nevertheless when specifying the overall software architecture, the designer should be aware of the implications of software design decisions on the target end system.

\section{Safety Integrity and Functional Safety according CENELEC}
The Railway Industry currently relies on the international standard group of coordinated standards: EN 50126 “Railway
applications – The specification and
demonstration of Reliability, Availability,
Maintainability and Safety (RAMS)” 
the EN 50129 “Railway applications – Safety
related electronic systems for signalling” and
the EN 50128 “Railway applications -  Communications, signalling and processing systems – Software for railway control and  protection systems” to provide a rational and consistent approach for the development of safety-related systems.

This group of standards owes much of its direction and contents to the IEC 61508 standard that is a generic safety standard for electrical/electronic/programmable electronics safety-related systems.

Both of these IEC and EN standards share the same philosophy in the sense that they:

\begin{itemize}
\item consider all relevant product and software safety life-cycle phases, from an initial concept phase to maintenance and decommissioning when these systems are used to perform safety functions;
\item intend to shape a safety awareness; \item have been conceived with a rapidly developing technology in mind;
\item provide methods and rules for defining safety requirements necessary to achieve defined functional safety.
\item use Safety Integrity Levels (SIL) for specifying the target level of safety integrity for the safety functions to be implemented.
\item adopt a statistical risk-based approach for the determination of the SIL requirements;
\item distinguish between safe and unsafe failure modes and requires precautions against undetected failures. 
\end{itemize}

According the Cenelec norms the product is subject to a certification process. The definition of the equipment under control (EUC) depends on the scope of the certification. It can be, for example the complete ERTMS/ETCS subsystem or a  module of it.

The term safety-related is used to describe systems that are required to perform a specific function to ensure that risks are kept at an acceptable level. Such functions are, by definition, safety functions. Two types of requirements are necessary to achieve functional safety: 

\begin{itemize}
\item Safety function requirements (what the function does),
\item Safety integrity requirements (the required likelihood of a safety function being performed satisfactorily).
\end{itemize}

The safety function requirements are derived from a risk analysis phase, in the scope of EN 50126, where significant risks for equipment and any associated control system in its intended environment have to be identified. This analysis determines whether functional safety is necessary to ensure adequate protection from unacceptable risks. Functional safety is therefore
a method of dealing with risks to eliminate them or reduce them to an acceptable level. EN 50128 specifies four levels of safety
performance for a safety function. These are called Software Safety Integrity Levels (SwSIL).

\section{Reference to the openETCS functional Model}
The openETCS OBU partial model has been developed according to the specification given in ERA Subset-026 \cite{subset-026}, Version 3.3.0. The software release is publicly available on a repository at 
\begin{quotation}
\centering
\url{https://github.com/openETCS/modeling/tree/v0.3-D3.6.3}
\end{quotation}




\chapter{Input documents}
%-----------------------------------------------------------------------
%\subsection{Mode and Level}
%-----------------------------------------------------------------------
%\tbc
%Baseliyos Jacob
This section gives an overview about the input documents used for the
\begin{itemize}
\item analysis of the OBU functions,
\item functional decomposition and allocation of functional blocks, functions and libraries,
\item design of the OBU functions, and
\item determination of "use cases" and scenarios for the different iterations of the Architecture and Design Document
\end{itemize} 

List of main documents that are being used as reference or input for analysis and design:
\begin{itemize}
\item ERA TSI CCS documents
\item openETCS API speccification
\item openETCS requirements WP2
\item Railway operator documents
\item Industry documents
\item ERSA simulator documents
\item Other project partner data, information and documents
\item Utrecht - Amsterdam track documents
\end{itemize}

Furthermore, relevant ETCS know how from industry and operators is used for the design and the analysis.

While the list above serves as a high-level reference, detailed information, links to the actual documents and additional remarks are being maintained at
\url{https://github.com/openETCS/modeling/wiki/Input-Documents-Repository}
while the documents describing the standard are referenced at \url{https://github.com/openETCS/SSRS/wiki/SSRS-Documents}

The figure below illustrates  the relationships among the input used documents:

\begin{figure}
\includegraphics[scale=0.5]{images/AnalysisDocuments}
\caption{Analysis of input documents.}
\label{Analyising of input document}
\end{figure}

For a detailed discussion of the actual work process, please refer to the previous chapters. 



\chapter{Use case description - proof of concept Utrecht - Amsterdam}
%-----------------------------------------------------------------------
%\subsection{Mode and Level}
%-----------------------------------------------------------------------
%\tbc
%Baseliyos Jacob

\section{Proof of concept on the Track Utrecht Amsterdam User Stories 1 - 4}

The goal of the openETCS@ITEA2 project is to deliver at the and a proof of concept in a lab on a real ETCS Track. Since the Level 2 Utrecht - Amsterdam track was evaluated as the most approriate reference track for this concept due the maturity and representative of the track, it will be use for the mentioned simulation.\\

To start with the realisation of the concept in a iterative way in the same pattern the industry is proceeding and an regarding the "classical" state of the art of sytemanalisiys we started in this third iteration with the following Use Cases and Scenarios:\\

\subsection{Use Case and Scenario 1}
\textbf{Start of Mission - Awakening of the Train:}
This use case according to the procedure in chapter 5 will demonstrate the start of a train from a no power modus to the state that train will be ready for level and mode change according to the chapter 5.\\ 
Link on Git-Hub \url{https://github.com/openETCS/modeling/issues/66}

The following Subsystems needs to be realised for Scenario 1:\\
\begin{itemize}
\item Procedure
\item TIU Management
\item DMI Management and Controller
\item Position Report
\item Management of Radio Communication
\item Manage Track Data
\item Manage Mode and Level
\item Train Supervision
\end{itemize}

\subsection{Use Case and Scenario 2}
\textbf{Start of Mission - Start in Level 2 Mode FS:}
This use case according to the procedure in chapter 5 will demonstrate the start of a train from the awakening of the train in mode stand by to the state that train will receive a movement authority in level 2 and change into the mode full supervision to start running under real supervision according to the chapter 5.\\ 
link on Git-Hub \url{https://github.com/openETCS/modeling/issues/67}

The following Subsystems needs to be realised for Scenario 2:\\
\begin{itemize}
\item Procedure
\item TIU Management
\item DMI Management and Controller
\item Position Report
\item Management of Radio Communication
\item Manage Track Data
\item Manage Mode and Level
\item Train Supervision
\end{itemize}

\subsection{Use Case and Scenario 3}
\textbf{Brake intervention - Revocation of a Movement Authority and Overrun Permitted Speed:}
This use case according to the subset 26 chapter 3 principles will demonstrate the brake intervention that will cause by a revocation of a movement authority due to a occupied section or track an due simple overrun of a permitted speed  according to the chapter 3.\\ 
Link on Git-Hub \url{https://github.com/openETCS/modeling/issues/68}

The following Subsystems needs to be realised for Scenario 3:\\
\begin{itemize}
\item Procedure
\item TIU Management
\item DMI Management and Controller
\item Position Report
\item Management of Radio Communication
\item Manage Track Data
\item Manage Mode and Level
\item Train Supervision
\end{itemize}

\subsection{Use Case and Scenario 4}
\textbf{ETCS Onboard Unit is reading and sending track information:}
This use case according to the subset 26 chapter 3 principles will demonstrate the full completeness and checking the reading and sending of track information in interaction with the ETCS Onboard Unit and the track that will be separated in radio and balise messages. Messages and packages are defined in chapter 7 and 8 of the subset 26.\\ 
Link on Git-Hub \url{https://github.com/openETCS/modeling/issues/69}

The following Subsystems needs to be realised for Scenario 4:\\
\begin{itemize}
\item Procedure
\item TIU Management
\item DMI Management and Controller
\item Position Report
\item Management of Radio Communication
\item Manage Track Data
\item Manage Mode and Level
\item Train Supervision
\item Buildiung of coordinate system
\end{itemize}





\chapter{Architecture Description}
%-----------------------------------------------------------------------
%\subsection{Mode and Level}
%-----------------------------------------------------------------------
%\tbc
%Baseliyos Jacob

\section{System Architecture}


\section{Interfaces}


\chapter{Runtime API}

\section{Introduction to the Architecture}

\subsection{Abstract Hardware Architecture}

For proper understanding of openETCS \gls{API} and of constraints imposed on
both sides of the \gls{API}, we need to define a \define{reference abstract hardware architecture}. This hardware architecture is ``abstract''
is the sense that the actual vendor specific hardware architecture
might be totally different of the abstract architecture described in
this chapter. For example, several units might be grouped together on
the same processor.

However the actual vendor specific architecture shall fulfil all the
requirements and constraints of this reference abstract hardware
architecture and shall not request additional constraints.

\subsection{Definition of the reference abstract hardware architecture}

\begin{figure}
  \centering
  \includegraphics[width=\linewidth]{abstract-hardware-architecture.pdf}
  \caption{Reference abstract hardware architecture}
  \label{fig:hardware-arch}
\end{figure}

The reference abstract hardware architecture is shown in figure
\ref{fig:hardware-arch}.

The reference abstract hardware architecture is made of a bus on which
are connected \define{units} defining the \gls{OBU}:

\begin{itemize}
\item \gls{EVC};
\item \gls{TIU};
\item \gls{ODO};
\item \gls{DMI};
\item \gls{STM};
\item \gls{BTM};
\item \gls{LTM}: Not part of this openETCS implementation;
\item EURORADIO;
\item \gls{JRU}: Not part of this openETCS implementation;
\end{itemize}

Elements not being part of this implementation are marked. 

Those units shall working concurrently. They shall exchange
information with other units through asynchronous message passing.

\subsection{Reference abstract software architecture}
\label{software-arch}

\begin{figure}[htbp]
  \centering
  \includegraphics[width=\linewidth]{software-architecture.pdf}
  \caption{Reference abstract software architecture}
  \label{fig:software-arch}
\end{figure}

The \define{reference abstract software architecture} is shown in figure
\ref{fig:software-arch}. This architecture is made of following
elements:
\begin{itemize}
\item \define{openETCS executable model} produced by the
  \cite{scade-model} Scade Model. It shall contain the program implementing core
  ETCS functions;
\item\define{openETCS model run-time system} shall help the execution
  of the openETCS executable model by providing additional functions
  like encode/decode messages, proper execution of the model through
  appropriate scheduling, re-order or prioritize messages, etc. 
\item \define{Vendor specific \gls{API} adapter} shall make the link between
  the Vendor specific platform and the openETCS model run-time system.
  It can buffer message parts, encode/decode messages, route messages
  to other \gls{EVC} components, etc.
\item All above three elements shall be included in the \gls{EVC};
\item \define{Vendor specific platform} shall be all other elements of
  the system, bus and other units, as shown in figure
  \ref{fig:hardware-arch}.
\end{itemize}

We have thus three interfaces:
\begin{itemize}
\item \define{model interface}
 is the interface between openETCS
  executable model and openETCS model run-time system. 
\item \define{openETCS \gls{API}}
 is the interface between openETCS model
  run-time system and Vendor specific \gls{API} adapter.
\item \define{Vendor specific \gls{API}}
 is the interface between Vendor
  specific \gls{API} adapter and Vendor specific platform. This interface is
  not publicly described for all vendors. You can find the Alstom imnplementation as an example.
\end{itemize}

The two blocks openETCS executable model and openETCS model run-time
system are making the \define{Application software} part. This Application software might be either openETCS reference software or
vendor specific software.

The Vendor specific \gls{API} adapter is making the \define{Basic software} part.


\section{Functional breakdown}


\subsection{F1: openETCS \gls{API} Runtime System and Input to the EVC)}
\label{chp_openETCS_API}
%Authors: Bernd Hekele (DB)

\begin{figure}[hbtp]
\centering
\includegraphics[width=\linewidth]{openETCSAPI.png}
\caption{openETCS API Highlevel View}
\label{fig:apiHighLevel}
\end{figure}

Figure \ref{fig:apiHighLevel} shows the structure of API with respect of the software architecture. Input boxes and output boxes not implemented in this stage are marked as red, other interfaces are marked as green. The System covers functions for processing Inputs from other Units, functions for processing Outputs to other functions and a basic runtime system. Inputs are used to feed the input to the executable model before calling it, outputs are used for collecting information provided by the executable model to be passed to the relevant interfaces after the execution cycle has finished.

\subsubsection{Principles for Interfaces (openETCS \gls{API})}


Information  is exchanged \define{messages} in an asynchronous way. A message is a set
of information corresponding to an event of a particular unit, e.g. a
balise received from the \gls{BTM}. The possible kind of messages are
described in chapter \ref{information-flows}.

The information is passed to the executable model as parmeters to the snychronous call of a procedure (Interface to the executable model). Since the availability of input messages to the application is not guaranteed the parts of the interfaces are defined with a "present" flag. In addition, fields of input arraysquite often is of variable size. Implementation in the concrete interface in this use-case is the use of a "size" parameter and a "valid"-flag.


\subsubsection{openETCS Model Runtime System}
The openETCS model runtime system also provides:

\begin{itemize}
\item Input Functions From other Units\\
In this entity messages from other connected units are received.
\item Output Functions to other Units\\
The entity writes messages to other connected units.
\item Conversation Functions for Messages (Bitwalker)\\
The conversion function are triggered by Input and Ouput Functions. The main task is to convert input messages from an bit-packed format into logical ETCS messages (the ETCS language) and Output messages from Logical into a bit-packed format. The logical format of the messages is defined for all used types in the openETCS data dictonary. \\
Variable size elements in the Messages are converted to fixed length arrays with an used elements indicator.\\
Optional elements are indicated with an valid flag.
The conversion routines are responsible for checking the data received is valid. If  faults are detected the information is passed to the openETCS executable model for further reaction. 
\item Model Cycle\\

The version management function is part of the message handling.This implies, conversions from other physical or logical layouts of messages are mapped onto a generic format used in the EVC. Information about the origin version of the message is part of the messages.
 
The executable model is called in cycles. In the cycle 
\begin{itemize}
\item First the received input messages are decoded
\item The input data is passed to the executable model in a predefined order. \textbf{(Details for the interface to be defined)}.
\item Output is encoded according to the \gls{SRS} and passed to the  buffers to the units.
\end{itemize}
\end{itemize}


\subsubsection{Input Interfaces of the openETCS API From other Units of the OBU}
Interfaces are defined in the Scade project APITypes (package API\_Msg\_Pkg.xscade).

In the interfaces the following principles for indicating the quality of the information is used:


\tablefirsthead{
\hline 
\rowcolor{gray} 
Indicator & Type & Purpose \\\hline}
\begin{supertabular}{| p{2 cm} | p{2 cm} | p{8 cm} |}
present & bool & True indicates the component has been changed compared to the previous call of the routine
\\\hline 
valid & bool & True indicates the component is valid to be used. 
\\\hline 
\end{supertabular}

In the next table we can see the interfaces being used in the openETCS system. Details on the interfaces are defined further down.

\tablefirsthead{
\hline 
\rowcolor{gray} 
Unit & Name &  Processing Function  \\\hline}
%\begin{itemize}{| m{1.2cm} | m{1.5cm} | m{1.2cm} | m{3.7cm}  | m{3.7cm} |}
\begin{supertabular}{| c | c | c |}
\gls{BTM} & Balise Telegram & Receive Messages  \\\hline
\gls{DMI} & Driver Machine Interface & DMI Manager  \\\hline
EURORADIO & Communication Management & Communication Management  \\\hline
EURORADIO & Radio Messages & Receive Messages  \\\hline
\gls{ODO} & Odometer & All Parts \\\hline
System TIME & Time system of the OBU & All Parts \\\hline
TIU & Train Data & All Parts \\\hline
\end{supertabular}

Information in the following sections gives an more detailed overview of the structure of the interfaces.


\subsubsection{Message based interface (BTM, RTM)}


Balise Message (Track to Train)\\

\tablefirsthead{
\hline 
\rowcolor{gray} 
Message Name & Optional Packets & Restrictions in the current scope \\\hline}
\begin{supertabular}{| p{4 cm} | p{6 cm} | p{4,5 cm} |}
Balise Telegram &
3: National Values \newline
41: Level Transition Order \newline
42: Session Management  \newline
45: Radio Network registration \newline
46: Conditional Level Transition Order \newline
65: Temporary Speed Restriction \newline
66: Revoke Temporary Speed Restriction \newline
72: Packet for sending plain text messages \newline
137: Stop if in Staff Responsible \newline
255: End of Information \newline
& Used in Scenario
\\\hline
Balise Telegram &
0, 2, 3, 5, 6, 12, 16, 21, 27, 39,
40, 41, 42, 44, 45, 46, 49, 51, 52, 65,
66, 67, 68, 69, 70, 71, 72, 76, 79, 80,
88, 90, 131, 132, 133, 134, 135, 136, 137, 138,
139, 141, 145, 180, 181, 254
&  Not Used in Scenario\\\hline
\end{supertabular}

Radio Messages (Track to Train)

\tablefirsthead{
\hline 
\rowcolor{gray} 
Message Name & Optional Packets & Restrictions in the current scope \\\hline}
\begin{supertabular}{| p{4 cm} | p{6 cm} | p{4,5 cm} |}
2: SR Authorisation & 63:\ List\ of\ Balises\ in\ SR Authority & Message Not Supported \\\hline
3: Movement Authority &
 21:\ Gradient\ Profile\newline
 27: International Static Speed Profile\newline
 49: List of balises for SH Area\newline
 80: Mode profile\newline
 plus common optional packets\newline
 & a \\\hline
9: Request To Shorten MA &
 49: List of balises for SH Area\newline
 80: Mode profile\newline 
& \\\hline
24: General Message &
From RBC:\newline
 21:\ Gradient\ Profile\newline
 27: International Static Speed Profile\newline
 plus common optional packets\newline
From RIU:\newline 44, 45, 143, 180, 254
& Messages from RIU are not supported \\\hline
28: SH authorised & 3, 44, 49
& \\\hline
33: MA with Shifted Location Reference &
 21:\ Gradient\ Profile\newline
 27: International Static Speed Profile\newline
 49: List of balises for SH Area\newline
 80: Mode profile\newline
 plus common optional packets\newline
& \\\hline
37: Infill MA &
5, 21, 27, 39, 40, 41, 44, 49, 51, 52, 65, 66, 68, 69, 70, 71, 80, 88, 138, 139 
 & Message Not Supported \\\hline
List of common optional parameters &
3, 5, 39, 40, 51, 41, 42, 44, 45, 52, 57, 58, 64, 65, 66, 68, 69, 70, 71, 72, 76, 79, 88, 131, 138, 139, 140, 180
& \\\hline
\end{supertabular}

The runtime system is in charge to transfer the messages from its stream mode first to  compressed message format. 

\subsubsection{Interfaces to the Time System}
The interface types are defined in the OBU\_Basic\_Types\_Pkg Package. The system time is defined in the basic software.

The system TIME is provided to the executable model at the begin of the cycle. It is not refreshed during the cycle. The time provided to the application is equal to 0 at power-up of the EVC (it is not a “UTC time” nor a “Local
Time”), then must increase at each cycle (unit = 1 msec), until it reaches its maximum value (i.e current EVC
limitation = 24 hours)

\begin{itemize}
\item TIME (T\_internal\_Type, 32-bit INT)\\
Standardized system time type used for all internal time calculations: in ms. The time is defined as a cyclic counter: When the maximum is exceeded the time starts from 0 again. 

\item CLOCK (to be implemented)\\
The clocking system is provided by the JRU. A GPS based clock is assumed to provide the local time.

\end{itemize}

\subsubsection{Interfaces to the Odometry System}
The interface types are defined in the OBU\_Basic\_Types\_Pkg Package. 
The odometer gives the current information of the positing system of the train. In this section the structure of the interfaces are only highlighted. Details, including the internal definitions for distances, locations speed and time are implemented in the package. 

\begin{itemize}
\item Odometer (odometry\_T)
\begin{itemize}
\item valid (bool)\\
valid flag, i.e., the information is provided by the ODO system and can be used.
\item timestamp (T\_internal\_Type)\\
of the system when the odometer information was collected. Please, see also general remarks on the time system. 
\item Coordinate (odometryLocation\_T)
\begin{itemize}
\item nominal (L\_internal\_Type) [cm]
\item min (L\_internal\_Type) [cm]
\item max (L\_internal\_Type) [cm]
\end{itemize}
The type used for length values is a 32 bit integer. 
Min and max value give the interval where the train is to be expected. The bounderies are determined by the inaccuracy of the positioning system. All values are set to 0 when the train starts.

\item speed (OdometrySpeeds\_T) [km/h]
\begin{itemize}
\item v\_safeNominal (speed internal type) [km/h]\\
The safe nominal estimation of the speed which will
be bounded between 98\% and 100\% of the upper
estimation
\item v\_rawNominal (speed internal type) [km/h]\\
The raw nominal estimation of the speed which will
be bounded between the lower and the upper
estimations
\item v\_lower (speed internal type) [km/h]\\
The lower estimation of the speed
\item v\_upper (speed internal type) [km/h]\\
The upper estimation of the speed
\end{itemize}
The type used for speed values is a 32 bit integer. 
Min and max value give the interval where the train is to be expected. The bounderies are determined by the inaccuracy of the positioning system. All values are set to 0 when the train starts.
\item acceleration (A\_internal\_Type)[0.01 m/s2],\\
Standardized acceleration type for all internal calculations : in 
\item motionState (Enumeration)\\
indicates whether the train is in motion or in no motion
\item motionDirection (Enumeration)\\
indicates the direction of the train, i.e., CAB-A first, CAB-B first or unknown.
\end{itemize}
\end{itemize}

\subsubsection{Interfaces to the Train Interfaces (TIU)}
The following infomration is based on the implementation of the Alstom API. The interface is organised in packets. The packets of the Alstom implementation are listed in the appendix to this document.

The description of interfaces needed for the current scope will be added according to the use.

\subsubsection{Output Interfaces of the openETCS API TO other Units of the OBU}

\tablefirsthead{
\hline 
\rowcolor{gray} 
From Function & Name &  To Unit & Description \\\hline}
%\begin{supertabular}{| m{1.2cm} | m{1.5cm} | m{1.2cm} | m{3.7cm}  | m{3.7cm} |}
\begin{supertabular}{| c | c | c | c  | c |}
 & Radio Output Message & \ EURORADIO & \\\hline
 & Communication Management  &  EURORADIO  & \\\hline
 & Driver Information & \gls{DMI} & \\\hline
 & Train Data  & TIU &  
\\\hline
\end{supertabular}

Packets:
to be completed

Radio Messages
to be completed


%-----------------------------------------------------------------------
%\subsection{Runtime- APIl}
%-----------------------------------------------------------------------
%\tbc
%JakobGärtner




\bibliographystyle{unsrt}
\bibliography{architecture}


\addcontentsline{toc}{chapter}{Index}
\printindex
%===================================================
%Do NOT change anything below this line

\end{document}
