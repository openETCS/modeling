\documentclass{template/openetcs_report}
% Use the option "nocc" if the document is not licensed under Creative Commons
%\documentclass[nocc]{template/openetcs_article}
\usepackage{lipsum,url}
\usepackage{supertabular}
\usepackage{multirow}
\usepackage{color, colortbl}
\usepackage{hyperref}
\usepackage{listings}
\usepackage{makeidx}
\definecolor{gray}{rgb}{0.8,0.8,0.8}
\usepackage[modulo]{lineno}
\usepackage{float}
\usepackage{fixme}
\usepackage[acronym, % list of acronyms
  %section, % add the glossary to the table of content
            %description,% acronyms have a user-supplied description,
 style=longheader, % table style
 nonumberlist % no page number
  ]{glossaries}

\graphicspath{{./template/}{.}{./images/}}

\renewcommand*{\glspostdescription}{} %Deactivate point at the end of every description
\renewcommand*{\glossaryname}{Glossary}

%create glossary
 \makeglossaries
 %Glossary terms
 \loadglsentries{glossary}

\begin{document}
\frontmatter
\project{openETCS}

\newcommand{\define}[1]{\index{#1}\emph{#1}}






%Please do not change anything above this line
%============================

%user specified macros
%\newenvironment{activity}[2][planned]
	{\begin{tabular}{p{0.25\textwidth}@{\hspace{0.05\textwidth}}p{0.7\textwidth}}
			\multicolumn{2}{p{\textwidth}}{\colorbox{black}{\begin{minipage}{1.1cm}\begin{center}\textsc{\footnotesize \textcolor{white}{#1}}\end{center}\end{minipage}}~~\textbf{#2}}\\
	}
	{\end{tabular}}

\newcommand{\entry}[2]{#1:&#2\\}
\newcommand{\website}[1]{Website:&\url{#1}\\}
\newcommand{\desc}[1]{\multicolumn{2}{p{\textwidth}}{#1}\\}

\newcommand{\VV}{Verification \& Validation\xspace}
\newcommand{\vv}{verification \& validation\xspace}

\newcommand{\tbd}{\colorbox{cyan}{\%\%To Be Defined\%\%}}
\newcommand{\tbc}{\colorbox{cyan}{\%\%To Be Confirmed\%\%}}
\newcommand{\todo}[1]{\colorbox{cyan}{\%\%{#1}\%\%}}
\newcommand{\nthng}[1]{}

% The document metadata is defined below

%assign a report number here
%\reportnum{OETCS/WP3/D3.5.1.3}

%define your workpackage here
\wp{Work-Package 3: ``Modeling''}

%set a title here
\title{openETCS System Architecture and Design Specification}

%set a subtitle here
\subtitle{Third iteration: Scope of openETCS ITEA2 Functions}

%set the date of the report here
\date{November 2014}


%document approval
%define the name and affiliation of the people involved in the documents approbation here
\creatorname{Baseliyos Jacob}
\creatoraffil{LEA Engineering / DB Netz}

\techassessorname{[assessor name]}
\techassessoraffil{[affiliation]}

\qualityassessorname{Izaskun de la Torre}
\qualityassessoraffil{SQS}

\approvalname{Klaus-R\"udiger Hase}
\approvalaffil{DB Netz}


%define a list of authors and their affiliation here

\author{Baseliyos Jacob, Bernd Hekele, Peyman Farhangi, Stefan Karg}

\affiliation{DB Netz AG\\
  V\"olckerstrasse 5\\
  D-80959 M\"unchen Freimann, Germany}

\author{Uwe Steinke}

\affiliation{Siemens AG}

\author{Christian Stahl}

\affiliation{TWT-GmbH}

\author{David Mentré}
\affiliation{Mitsubishi Electric R\&D Centre Europe}

\author{David Mentre}
\affiliation{Mitsubishi Electric R\&D Centre Europe}

\author{Jos Holtzer, Jan Welvaarts, Vincent Nuhaan}
\affiliation{NS}

\author{Jacob G\"artner}
\affiliation{LEA Engineering}

% define the coverart
\coverart[width=350pt]{openETCS_EUPL}

%define the type of report
\reporttype{Architecture and Functional Specification}


\begin{abstract}
%define an abstract here
This document gives an introduction to the architecture of openETCS. The functional scope is tailored to cover the functionality required for the openETCS demonstration as a target of the ITEA2 project: the Utrecht Amsterdam use-case. It has to be read as an add-on to the models in SysML, Scade and to additional reading referenced from the document.
\end{abstract}

%=============================
\maketitle

%Modification history
%if you do not need a modification history table for your document simply comment out the eight lines below
%=============================


\chapter*{Modification History}
\tablefirsthead{
\hline 
\rowcolor{gray} 
Version & Section & Modification / Description & Author \\\hline}
\begin{supertabular}{| m{1.2cm} | m{1.5cm} | m{6.6cm} | m{3.7cm} |}
0.1 & Document & Initial document providing the structure & Baseliyos Jacob \\\hline
0.2 & Document & Workshop Results included and some pretty-printing & Bernd Hekele \\\hline

\end{supertabular}

% list subsubsections in table of contents
\setcounter{tocdepth}{3}


\tableofcontents
\listoffiguresandtables
\newpage
%=============================

%Uncomment the next line if you need line numbers for tracebility when the document is in review
%\linenumbers
%=============================


% The actual document starts below this line
%=============================

\mainmatter

\chapter{Introduction}

%set the master document for easy compilation
%!TEX root = ../D3_5_3.tex

\chapter{Purpose of the document}

This document is managed as a deliverable of the modeling work package with denomination ~D3.7.x, and contains advices and recommendation for the design of a physical system architecture.  

The development of the functional model is done iteratively increasing the scope in steps, the last digit of the deliverable identifier, i.e.~x, denotes the release of the model to which it applies. If the functional model requires to update the system architecture a consistent version number will be applied to this document as required by the Model release version.

This document complements the indications contained in the API requirements specification and the documentation derived from this as the generic openETCS Application Programming Interface (API), available at \url{https://github.com/openETCS/modeling/blob/master/API/description/api-description.pdf}. \cite{alstom-api}

\section{Input Documents}

The following documents provide a context for the system perspective.

\begin{itemize}
	\item ERA Subset-026 \cite{subset-026}, V3.3.0
	\item ERA TSI CCS Documents
	\item openETCS API documentation, available at \url{https://github.com/openETCS/modeling/blob/master/API/description/api-description.pdf} \cite{alstom-api}\cite{alstom-api-app-layer}\cite{alstom-api-data-dict}
	%\item openETCS requirements, i.e.~D2.1, D2.2,$\ldots$, %D2.9, available at %\url{https://github.com/openETCS/requirements/tree/maste%r/Reference}
\end{itemize}




\chapter{Introduction}

Designing a sub system integrable with the train borne system is a complex task. The designer faces a large variety of serious challenges and design complexities. 

Before the functions are actually implemented, a system architect will have to select an appropriate hardware-software concept out of the large number of available boards, controllers, network and  bus constraints.  He  will as well include robustness criteria against environmental influences. 

Memories, operating systems, drivers, generic and application software segregation as well as selection criteria for sensors and actuators need to be correctly assessed. 

The target architecture has to meet a large variety of requirements. Criteria of timing, Bus bandwidth, processor and peripheral performance, memory size, safety principles and possible processing or data transfer bottlenecks. Environmental conditions, timing constraints, robustness against specific interferences shall constantly be tracked.

Power requirement as well as allocation of availability, maintainability figures to enumerate only the most relevant items accompany all the design phases.

On top of this a specific vital architecture has to be selected and the required integrity level has to be granted. The relative safety constraints have to be assured and maybe exported.

Selecting the components matching these is a critical phase. Over-dimensioning the architecture may impact on cost factors relevant for the market access of the system. Under-dimensioning the architecture design could result in not achieving performance constraints, thus compromising system quality and suitability. Early architectural choices have a dominant impact on the success of the new system. 

The system architects will commit to efficient design choices according to the target project margins and all this within the frame of a defined project delivery time schedule. 

Due to the fact that the design verification phase, requiring to have completed all the integration steps, may be very late in the release process a high precision during the system architecture design is mandatory. 

Therefore highly experienced System Designer are considered as the key factor for a reliable achievement of expected design result.

A primary goal of the openETCS ITEA2 project is to provide a formal specification and a model of an ETCS onboard functionality according to the specification defined in Subset-026 \cite{subset-026} by the European Railway Agency (ERA). 

The Model-Based Development process is an approach that allows engineers to specify the behavior of a system and to simulate and execute it in a very early development stage.

Once a model-based development process has been established, engineers should be able to apply new technologies and tools to enhance and shorten product development cycles,
e.g. by introducing generation of Model Validation test cases and target Code directly from the model. This enables to improve the V based development process to save development time and effort while preserving or improving the dependability of the developed systems. 


\begin{figure}[H]
	\center
	\includegraphics[width= 0.9\textwidth]{Y_process.pdf}
	\caption{SRS modeling cycle}\label{Y_process}
\end{figure}


The methodology makes it easier to understand requirements and increases the correctness of the requirements, the correctness of the design and the code with respect to the requirements. An integration of system-level and design-level modeling tools allows a virtually integrated V-process that is sharpened up to a Y-based process with the required steps at the bottom of the former V being considerably automated (see figure\ref{Y_process} )

Nevertheless when specifying the overall software architecture, the designer should be aware of the implications of software design decisions on the target end system.

\section{Safety Integrity and Functional Safety according CENELEC}
The Railway Industry currently relies on the international standard group of coordinated standards: EN 50126 “Railway
applications – The specification and
demonstration of Reliability, Availability,
Maintainability and Safety (RAMS)” 
the EN 50129 “Railway applications – Safety
related electronic systems for signalling” and
the EN 50128 “Railway applications -  Communications, signalling and processing systems – Software for railway control and  protection systems” to provide a rational and consistent approach for the development of safety-related systems.

This group of standards owes much of its direction and contents to the IEC 61508 standard that is a generic safety standard for electrical/electronic/programmable electronics safety-related systems.

Both of these IEC and EN standards share the same philosophy in the sense that they:

\begin{itemize}
\item consider all relevant product and software safety life-cycle phases, from an initial concept phase to maintenance and decommissioning when these systems are used to perform safety functions;
\item intend to shape a safety awareness; \item have been conceived with a rapidly developing technology in mind;
\item provide methods and rules for defining safety requirements necessary to achieve defined functional safety.
\item use Safety Integrity Levels (SIL) for specifying the target level of safety integrity for the safety functions to be implemented.
\item adopt a statistical risk-based approach for the determination of the SIL requirements;
\item distinguish between safe and unsafe failure modes and requires precautions against undetected failures. 
\end{itemize}

According the Cenelec norms the product is subject to a certification process. The definition of the equipment under control (EUC) depends on the scope of the certification. It can be, for example the complete ERTMS/ETCS subsystem or a  module of it.

The term safety-related is used to describe systems that are required to perform a specific function to ensure that risks are kept at an acceptable level. Such functions are, by definition, safety functions. Two types of requirements are necessary to achieve functional safety: 

\begin{itemize}
\item Safety function requirements (what the function does),
\item Safety integrity requirements (the required likelihood of a safety function being performed satisfactorily).
\end{itemize}

The safety function requirements are derived from a risk analysis phase, in the scope of EN 50126, where significant risks for equipment and any associated control system in its intended environment have to be identified. This analysis determines whether functional safety is necessary to ensure adequate protection from unacceptable risks. Functional safety is therefore
a method of dealing with risks to eliminate them or reduce them to an acceptable level. EN 50128 specifies four levels of safety
performance for a safety function. These are called Software Safety Integrity Levels (SwSIL).

\section{Reference to the openETCS functional Model}
The openETCS OBU partial model has been developed according to the specification given in ERA Subset-026 \cite{subset-026}, Version 3.3.0. The software release is publicly available on a repository at 
\begin{quotation}
\centering
\url{https://github.com/openETCS/modeling/tree/v0.3-D3.6.3}
\end{quotation}




\glsaddall
\printglossaries

\chapter{Input documents}
%-----------------------------------------------------------------------
%\subsection{Mode and Level}
%-----------------------------------------------------------------------
%\tbc
%Baseliyos Jacob
This section gives an overview about the input documents used for the
\begin{itemize}
\item analysis of the OBU functions,
\item functional decomposition and allocation of functional blocks, functions and libraries,
\item design of the OBU functions, and
\item determination of "use cases" and scenarios for the different iterations of the Architecture and Design Document
\end{itemize} 

List of main documents that are being used as reference or input for analysis and design:
\begin{itemize}
\item ERA TSI CCS documents
\item openETCS API speccification
\item openETCS requirements WP2
\item Railway operator documents
\item Industry documents
\item ERSA simulator documents
\item Other project partner data, information and documents
\item Utrecht - Amsterdam track documents
\end{itemize}

Furthermore, relevant ETCS know how from industry and operators is used for the design and the analysis.

While the list above serves as a high-level reference, detailed information, links to the actual documents and additional remarks are being maintained at
\url{https://github.com/openETCS/modeling/wiki/Input-Documents-Repository}
while the documents describing the standard are referenced at \url{https://github.com/openETCS/SSRS/wiki/SSRS-Documents}

The figure below illustrates  the relationships among the input used documents:

\begin{figure}
\includegraphics[scale=0.5]{images/AnalysisDocuments}
\caption{Analysis of input documents.}
\label{Analyising of input document}
\end{figure}

For a detailed discussion of the actual work process, please refer to the previous chapters. 



\chapter{Use case description - proof of concept Utrecht - Amsterdam}
%-----------------------------------------------------------------------
%\subsection{Mode and Level}
%-----------------------------------------------------------------------
%\tbc
%Baseliyos Jacob

\section{Proof of concept on the Track Utrecht Amsterdam User Stories 1 - 4}

The goal of the openETCS@ITEA2 project is to deliver at the and a proof of concept in a lab on a real ETCS Track. Since the Level 2 Utrecht - Amsterdam track was evaluated as the most approriate reference track for this concept due the maturity and representative of the track, it will be use for the mentioned simulation.\\

To start with the realisation of the concept in a iterative way in the same pattern the industry is proceeding and an regarding the "classical" state of the art of sytemanalisiys we started in this third iteration with the following Use Cases and Scenarios:\\

\subsection{Use Case and Scenario 1}
\textbf{Start of Mission - Awakening of the Train:}
This use case according to the procedure in chapter 5 will demonstrate the start of a train from a no power modus to the state that train will be ready for level and mode change according to the chapter 5.\\ 
Link on Git-Hub \url{https://github.com/openETCS/modeling/issues/66}

The following Subsystems needs to be realised for Scenario 1:\\
\begin{itemize}
\item Procedure
\item TIU Management
\item DMI Management and Controller
\item Position Report
\item Management of Radio Communication
\item Manage Track Data
\item Manage Mode and Level
\item Train Supervision
\end{itemize}

\subsection{Use Case and Scenario 2}
\textbf{Start of Mission - Start in Level 2 Mode FS:}
This use case according to the procedure in chapter 5 will demonstrate the start of a train from the awakening of the train in mode stand by to the state that train will receive a movement authority in level 2 and change into the mode full supervision to start running under real supervision according to the chapter 5.\\ 
link on Git-Hub \url{https://github.com/openETCS/modeling/issues/67}

The following Subsystems needs to be realised for Scenario 2:\\
\begin{itemize}
\item Procedure
\item TIU Management
\item DMI Management and Controller
\item Position Report
\item Management of Radio Communication
\item Manage Track Data
\item Manage Mode and Level
\item Train Supervision
\end{itemize}

\subsection{Use Case and Scenario 3}
\textbf{Brake intervention - Revocation of a Movement Authority and Overrun Permitted Speed:}
This use case according to the subset 26 chapter 3 principles will demonstrate the brake intervention that will cause by a revocation of a movement authority due to a occupied section or track an due simple overrun of a permitted speed  according to the chapter 3.\\ 
Link on Git-Hub \url{https://github.com/openETCS/modeling/issues/68}

The following Subsystems needs to be realised for Scenario 3:\\
\begin{itemize}
\item Procedure
\item TIU Management
\item DMI Management and Controller
\item Position Report
\item Management of Radio Communication
\item Manage Track Data
\item Manage Mode and Level
\item Train Supervision
\end{itemize}

\subsection{Use Case and Scenario 4}
\textbf{ETCS Onboard Unit is reading and sending track information:}
This use case according to the subset 26 chapter 3 principles will demonstrate the full completeness and checking the reading and sending of track information in interaction with the ETCS Onboard Unit and the track that will be separated in radio and balise messages. Messages and packages are defined in chapter 7 and 8 of the subset 26.\\ 
Link on Git-Hub \url{https://github.com/openETCS/modeling/issues/69}

The following Subsystems needs to be realised for Scenario 4:\\
\begin{itemize}
\item Procedure
\item TIU Management
\item DMI Management and Controller
\item Position Report
\item Management of Radio Communication
\item Manage Track Data
\item Manage Mode and Level
\item Train Supervision
\item Buildiung of coordinate system
\end{itemize}





\chapter{Architecture Description}
%-----------------------------------------------------------------------
%\subsection{Mode and Level}
%-----------------------------------------------------------------------
%\tbc
%Baseliyos Jacob

\section{System Architecture}


\section{Interfaces}


\chapter{Runtime API}

\section{Introduction to the Architecture}

\subsection{Abstract Hardware Architecture}

For proper understanding of openETCS \gls{API} and of constraints imposed on
both sides of the \gls{API}, we need to define a \define{reference abstract hardware architecture}. This hardware architecture is ``abstract''
is the sense that the actual vendor specific hardware architecture
might be totally different of the abstract architecture described in
this chapter. For example, several units might be grouped together on
the same processor.

However the actual vendor specific architecture shall fulfil all the
requirements and constraints of this reference abstract hardware
architecture and shall not request additional constraints.

\subsection{Definition of the reference abstract hardware architecture}

\begin{figure}
  \centering
  \includegraphics[width=\linewidth]{abstract-hardware-architecture.pdf}
  \caption{Reference abstract hardware architecture}
  \label{fig:hardware-arch}
\end{figure}

The reference abstract hardware architecture is shown in figure
\ref{fig:hardware-arch}.

The reference abstract hardware architecture is made of a bus on which
are connected \define{units} defining the \gls{OBU}:

\begin{itemize}
\item \gls{EVC};
\item \gls{TIU};
\item \gls{ODO};
\item \gls{DMI};
\item \gls{STM};
\item \gls{BTM};
\item \gls{LTM}: Not part of this openETCS implementation;
\item EURORADIO;
\item \gls{JRU}: Not part of this openETCS implementation;
\end{itemize}

Elements not being part of this implementation are marked. 

Those units shall working concurrently. They shall exchange
information with other units through asynchronous message passing.

\subsection{Reference abstract software architecture}
\label{software-arch}

\begin{figure}[htbp]
  \centering
  \includegraphics[width=\linewidth]{software-architecture.pdf}
  \caption{Reference abstract software architecture}
  \label{fig:software-arch}
\end{figure}

The \define{reference abstract software architecture} is shown in figure
\ref{fig:software-arch}. This architecture is made of following
elements:
\begin{itemize}
\item \define{openETCS executable model} produced by the
  \cite{scade-model} Scade Model. It shall contain the program implementing core
  ETCS functions;
\item\define{openETCS model run-time system} shall help the execution
  of the openETCS executable model by providing additional functions
  like encode/decode messages, proper execution of the model through
  appropriate scheduling, re-order or prioritize messages, etc. 
\item \define{Vendor specific \gls{API} adapter} shall make the link between
  the Vendor specific platform and the openETCS model run-time system.
  It can buffer message parts, encode/decode messages, route messages
  to other \gls{EVC} components, etc.
\item All above three elements shall be included in the \gls{EVC};
\item \define{Vendor specific platform} shall be all other elements of
  the system, bus and other units, as shown in figure
  \ref{fig:hardware-arch}.
\end{itemize}

We have thus three interfaces:
\begin{itemize}
\item \define{model interface}
 is the interface between openETCS
  executable model and openETCS model run-time system. 
\item \define{openETCS \gls{API}}
 is the interface between openETCS model
  run-time system and Vendor specific \gls{API} adapter.
\item \define{Vendor specific \gls{API}}
 is the interface between Vendor
  specific \gls{API} adapter and Vendor specific platform. This interface is
  not publicly described for all vendors. You can find the Alstom imnplementation as an example.
\end{itemize}

The two blocks openETCS executable model and openETCS model run-time
system are making the \define{Application software} part. This Application software might be either openETCS reference software or
vendor specific software.

The Vendor specific \gls{API} adapter is making the \define{Basic software} part.


\section{Functional breakdown}


\subsection{F1: openETCS \gls{API} Runtime System and Input to the EVC)}
\label{chp_openETCS_API}
%Authors: Bernd Hekele (DB)

\begin{figure}[hbtp]
\centering
\includegraphics[width=\linewidth]{openETCSAPI.png}
\caption{openETCS API Highlevel View}
\label{fig:apiHighLevel}
\end{figure}

Figure \ref{fig:apiHighLevel} shows the structure of API with respect of the software architecture. Input boxes and output boxes not implemented in this stage are marked as red, other interfaces are marked as green. The System covers functions for processing Inputs from other Units, functions for processing Outputs to other functions and a basic runtime system. Inputs are used to feed the input to the executable model before calling it, outputs are used for collecting information provided by the executable model to be passed to the relevant interfaces after the execution cycle has finished.

\subsubsection{Principles for Interfaces (openETCS \gls{API})}


Information  is exchanged \define{messages} in an asynchronous way. A message is a set
of information corresponding to an event of a particular unit, e.g. a
balise received from the \gls{BTM}. The possible kind of messages are
described in chapter \ref{information-flows}.

The information is passed to the executable model as parmeters to the snychronous call of a procedure (Interface to the executable model). Since the availability of input messages to the application is not guaranteed the parts of the interfaces are defined with a "present" flag. In addition, fields of input arraysquite often is of variable size. Implementation in the concrete interface in this use-case is the use of a "size" parameter and a "valid"-flag.


\subsubsection{openETCS Model Runtime System}
The openETCS model runtime system also provides:

\begin{itemize}
\item Input Functions From other Units\\
In this entity messages from other connected units are received.
\item Output Functions to other Units\\
The entity writes messages to other connected units.
\item Conversation Functions for Messages (Bitwalker)\\
The conversion function are triggered by Input and Ouput Functions. The main task is to convert input messages from an bit-packed format into logical ETCS messages (the ETCS language) and Output messages from Logical into a bit-packed format. The logical format of the messages is defined for all used types in the openETCS data dictonary. \\
Variable size elements in the Messages are converted to fixed length arrays with an used elements indicator.\\
Optional elements are indicated with an valid flag.
The conversion routines are responsible for checking the data received is valid. If  faults are detected the information is passed to the openETCS executable model for further reaction. 
\item Model Cycle\\

The version management function is part of the message handling.This implies, conversions from other physical or logical layouts of messages are mapped onto a generic format used in the EVC. Information about the origin version of the message is part of the messages.
 
The executable model is called in cycles. In the cycle 
\begin{itemize}
\item First the received input messages are decoded
\item The input data is passed to the executable model in a predefined order. \textbf{(Details for the interface to be defined)}.
\item Output is encoded according to the \gls{SRS} and passed to the  buffers to the units.
\end{itemize}
\end{itemize}


\subsubsection{Input Interfaces of the openETCS API From other Units of the OBU}
Interfaces are defined in the Scade project APITypes (package API\_Msg\_Pkg.xscade).

In the interfaces the following principles for indicating the quality of the information is used:


\tablefirsthead{
\hline 
\rowcolor{gray} 
Indicator & Type & Purpose \\\hline}
\begin{supertabular}{| p{2 cm} | p{2 cm} | p{8 cm} |}
present & bool & True indicates the component has been changed compared to the previous call of the routine
\\\hline 
valid & bool & True indicates the component is valid to be used. 
\\\hline 
\end{supertabular}

In the next table we can see the interfaces being used in the openETCS system. Details on the interfaces are defined further down.

\tablefirsthead{
\hline 
\rowcolor{gray} 
Unit & Name &  Processing Function  \\\hline}
%\begin{itemize}{| m{1.2cm} | m{1.5cm} | m{1.2cm} | m{3.7cm}  | m{3.7cm} |}
\begin{supertabular}{| c | c | c |}
\gls{BTM} & Balise Telegram & Receive Messages  \\\hline
\gls{DMI} & Driver Machine Interface & DMI Manager  \\\hline
EURORADIO & Communication Management & Communication Management  \\\hline
EURORADIO & Radio Messages & Receive Messages  \\\hline
\gls{ODO} & Odometer & All Parts \\\hline
System TIME & Time system of the OBU & All Parts \\\hline
TIU & Train Data & All Parts \\\hline
\end{supertabular}

Information in the following sections gives an more detailed overview of the structure of the interfaces.


\subsubsection{Message based interface (BTM, RTM)}


Balise Message (Track to Train)\\

\tablefirsthead{
\hline 
\rowcolor{gray} 
Message Name & Optional Packets & Restrictions in the current scope \\\hline}
\begin{supertabular}{| p{4 cm} | p{6 cm} | p{4,5 cm} |}
Balise Telegram &
3: National Values \newline
41: Level Transition Order \newline
42: Session Management  \newline
45: Radio Network registration \newline
46: Conditional Level Transition Order \newline
65: Temporary Speed Restriction \newline
66: Revoke Temporary Speed Restriction \newline
72: Packet for sending plain text messages \newline
137: Stop if in Staff Responsible \newline
255: End of Information \newline
& Used in Scenario
\\\hline
Balise Telegram &
0, 2, 3, 5, 6, 12, 16, 21, 27, 39,
40, 41, 42, 44, 45, 46, 49, 51, 52, 65,
66, 67, 68, 69, 70, 71, 72, 76, 79, 80,
88, 90, 131, 132, 133, 134, 135, 136, 137, 138,
139, 141, 145, 180, 181, 254
&  Not Used in Scenario\\\hline
\end{supertabular}

Radio Messages (Track to Train)

\tablefirsthead{
\hline 
\rowcolor{gray} 
Message Name & Optional Packets & Restrictions in the current scope \\\hline}
\begin{supertabular}{| p{4 cm} | p{6 cm} | p{4,5 cm} |}
2: SR Authorisation & 63:\ List\ of\ Balises\ in\ SR Authority & Message Not Supported \\\hline
3: Movement Authority &
 21:\ Gradient\ Profile\newline
 27: International Static Speed Profile\newline
 49: List of balises for SH Area\newline
 80: Mode profile\newline
 plus common optional packets\newline
 & a \\\hline
9: Request To Shorten MA &
 49: List of balises for SH Area\newline
 80: Mode profile\newline 
& \\\hline
24: General Message &
From RBC:\newline
 21:\ Gradient\ Profile\newline
 27: International Static Speed Profile\newline
 plus common optional packets\newline
From RIU:\newline 44, 45, 143, 180, 254
& Messages from RIU are not supported \\\hline
28: SH authorised & 3, 44, 49
& \\\hline
33: MA with Shifted Location Reference &
 21:\ Gradient\ Profile\newline
 27: International Static Speed Profile\newline
 49: List of balises for SH Area\newline
 80: Mode profile\newline
 plus common optional packets\newline
& \\\hline
37: Infill MA &
5, 21, 27, 39, 40, 41, 44, 49, 51, 52, 65, 66, 68, 69, 70, 71, 80, 88, 138, 139 
 & Message Not Supported \\\hline
List of common optional parameters &
3, 5, 39, 40, 51, 41, 42, 44, 45, 52, 57, 58, 64, 65, 66, 68, 69, 70, 71, 72, 76, 79, 88, 131, 138, 139, 140, 180
& \\\hline
\end{supertabular}

The runtime system is in charge to transfer the messages from its stream mode first to  compressed message format. 

\subsubsection{Interfaces to the Time System}
The interface types are defined in the OBU\_Basic\_Types\_Pkg Package. The system time is defined in the basic software.

The system TIME is provided to the executable model at the begin of the cycle. It is not refreshed during the cycle. The time provided to the application is equal to 0 at power-up of the EVC (it is not a “UTC time” nor a “Local
Time”), then must increase at each cycle (unit = 1 msec), until it reaches its maximum value (i.e current EVC
limitation = 24 hours)

\begin{itemize}
\item TIME (T\_internal\_Type, 32-bit INT)\\
Standardized system time type used for all internal time calculations: in ms. The time is defined as a cyclic counter: When the maximum is exceeded the time starts from 0 again. 

\item CLOCK (to be implemented)\\
The clocking system is provided by the JRU. A GPS based clock is assumed to provide the local time.

\end{itemize}

\subsubsection{Interfaces to the Odometry System}
The interface types are defined in the OBU\_Basic\_Types\_Pkg Package. 
The odometer gives the current information of the positing system of the train. In this section the structure of the interfaces are only highlighted. Details, including the internal definitions for distances, locations speed and time are implemented in the package. 

\begin{itemize}
\item Odometer (odometry\_T)
\begin{itemize}
\item valid (bool)\\
valid flag, i.e., the information is provided by the ODO system and can be used.
\item timestamp (T\_internal\_Type)\\
of the system when the odometer information was collected. Please, see also general remarks on the time system. 
\item Coordinate (odometryLocation\_T)
\begin{itemize}
\item nominal (L\_internal\_Type) [cm]
\item min (L\_internal\_Type) [cm]
\item max (L\_internal\_Type) [cm]
\end{itemize}
The type used for length values is a 32 bit integer. 
Min and max value give the interval where the train is to be expected. The bounderies are determined by the inaccuracy of the positioning system. All values are set to 0 when the train starts.

\item speed (OdometrySpeeds\_T) [km/h]
\begin{itemize}
\item v\_safeNominal (speed internal type) [km/h]\\
The safe nominal estimation of the speed which will
be bounded between 98\% and 100\% of the upper
estimation
\item v\_rawNominal (speed internal type) [km/h]\\
The raw nominal estimation of the speed which will
be bounded between the lower and the upper
estimations
\item v\_lower (speed internal type) [km/h]\\
The lower estimation of the speed
\item v\_upper (speed internal type) [km/h]\\
The upper estimation of the speed
\end{itemize}
The type used for speed values is a 32 bit integer. 
Min and max value give the interval where the train is to be expected. The bounderies are determined by the inaccuracy of the positioning system. All values are set to 0 when the train starts.
\item acceleration (A\_internal\_Type)[0.01 m/s2],\\
Standardized acceleration type for all internal calculations : in 
\item motionState (Enumeration)\\
indicates whether the train is in motion or in no motion
\item motionDirection (Enumeration)\\
indicates the direction of the train, i.e., CAB-A first, CAB-B first or unknown.
\end{itemize}
\end{itemize}

\subsubsection{Interfaces to the Train Interfaces (TIU)}
The following infomration is based on the implementation of the Alstom API. The interface is organised in packets. The packets of the Alstom implementation are listed in the appendix to this document.

The description of interfaces needed for the current scope will be added according to the use.

\subsubsection{Output Interfaces of the openETCS API TO other Units of the OBU}

\tablefirsthead{
\hline 
\rowcolor{gray} 
From Function & Name &  To Unit & Description \\\hline}
%\begin{supertabular}{| m{1.2cm} | m{1.5cm} | m{1.2cm} | m{3.7cm}  | m{3.7cm} |}
\begin{supertabular}{| c | c | c | c  | c |}
 & Radio Output Message & \ EURORADIO & \\\hline
 & Communication Management  &  EURORADIO  & \\\hline
 & Driver Information & \gls{DMI} & \\\hline
 & Train Data  & TIU &  
\\\hline
\end{supertabular}

Packets:
to be completed

Radio Messages
to be completed


%-----------------------------------------------------------------------
%\subsection{Runtime- APIl}
%-----------------------------------------------------------------------
%\tbc
%JakobGärtner


\chapter{Design Description}

\section{F1: Receive information from Trackside}
\section{F2: ETCS Kernel}
%-----------------------------------------------------------------------
\subsection{Manage\_TrackSideInformation\_Integration}
%-----------------------------------------------------------------------
%\tbc
%Bernd Hekele

The block ``Manage\_TrackSideInformation\_Integration'' is responsible for receiving Eurobalise-telegrams and Euroradio-messages from the API and perform several consistency checks on the input.

The block collects the telegrams of balises in order to build balise group messages. Euroradio messages are always delivered as a whole message. 

On each message, a consistency check is performed, before the data is validated according to the driving direction of the train. In general, messages not designated for the current driving direction of the train are not forwarded to the further processing.

After applying consistency checks, the data direction is validated.

%Information of the odometer is used to control for the train leaving the expectation window of the balises. % TODO makes not much sense here.

\begin{figure}[H]
 \centering
 \includegraphics[width=\textwidth]{./images/Input-Messages4.PNG}
 % Input-Messages4.PNG: 0x0 pixel, 0dpi, nanxnan cm, bb=
 \caption{Structure of the Receive message and check consistency module}
 \label{fig:receiveAndCheckConsistencyArch}
\end{figure}


\subsubsection{Input}
For providing the output, the module needs different input data flows. An overview is provided in table \ref{tbl:ReceiveMessageAndCheckConsistencyInput}
\begin{table}[H]
  \scriptsize
  \begin{tabular}{| c | l | l | l | l |}
    \hline
    \textbf{Index} & \textbf{Input name} & \textbf{Input type} & \textbf{Source}\\ \hline
    0 & \texttt{fullChecks} & \texttt{bool} & Configuration \\
    1 & \texttt{API\_trackSide\_Message} & \texttt{API\_Msg\_Pkg::API\_TrackSideInput\_T} & API\\
    2 & \texttt{ActualOdometry} & \texttt{Obu\_BasicTypes\_Pkg::odometry\_T} & Odometer\\
    3 & \texttt{reset} & \texttt{bool} & Environment\\
    4 & \texttt{trainPosition} & \texttt{TrainPosition\_Types\_Pck::trainPosition\_T} & Calculate Train Position\\
    5 & \texttt{modeAndLevel} & \texttt{BG\_Types\_Pkg::ModeAndLevelStatus\_T} & Mode and Level\\
    6 & \texttt{tNvContact} & \texttt{Obu\_BasicTypes\_Pkg::T\_internal\_Type} & Database\\
    7 & \texttt{lastRelevantEventTimestamp} & \texttt{Obu\_BasicTypes\_Pkg::T\_internal\_Type} & Database\\
    8 & \texttt{connectionStatus} & \texttt{Radio\_Types\_Pkg::sessionStatus\_Type} & Manage Radio Communication\\
    9 & \texttt{inSupervisingRbcId} & \texttt{int} & Database\\
    10 & \texttt{inAnnouncedBGs} & \texttt{TrainPosition\_Types\_Pck::positionedBGs\_T} & Calculate Train Position\\
    11 & \texttt{q\_nvlocacc} & \texttt{Q\_NVLOCACC} & Database\\
    \hline
  \end{tabular} 
  \caption{Overview over input}
  \label{tbl:ReceiveMessageAndCheckConsistencyInput}
\end{table}

\paragraph{Input 0: \texttt{fullChecks}}
The boolean indicates, if all checks on the message should be performed. The possible values are given in table \ref{tbl:fullChecks}.

\begin{table}[H]
  \begin{tabular}{| l | p{9cm} |}
    \hline
    \textbf{Value} & \textbf{Interpretation}\\ \hline
    true & All checks are performed.\\
    false & The module \texttt{Information Filter} is deactivated.\\
    \hline
  \end{tabular} 
  \caption{Possible values for the input \texttt{fullChecks}}
  \label{tbl:fullChecks}
\end{table}
\paragraph{Input 1: \texttt{API\_trackSide\_Message}}

The \texttt{API\_trackSide\_Message} is the message received from the API. The API performs preprocessing of RTM and BTM messages and deliveres a maximum of a single message per cycle to the SCADE model.

\paragraph{Input 2: \texttt{ActualOdometry}}
The input \texttt{ActualOdometry} is provided by the external odometry module of the train. It contains location information with inaccuracies.

\paragraph{Input 3: \texttt{reset}}
To delete all data stored in the module (e.g. collected balise-telegrams, which do not yet form a complete message), a reset input can be used. If the input is set to \texttt{true}, all data kept in the module is deleted and no input is accepted.

\begin{table}[H]
  \begin{tabular}{| l | p{9cm} |}
    \hline
    \textbf{Value} & \textbf{Interpretation}\\ \hline
    true & All data kept in the module is deleted and no input is accepted.\\
    false & No action. Data at input is accepted.\\
    \hline
  \end{tabular} 
  \caption{Possible values for the input \texttt{reset}}
  \label{tbl:reset}
\end{table}

\paragraph{Input 4: \texttt{trainPosition}}
The input \texttt{trainPosition} is generated by the ``Calculate Train Position'' module and contains the current position of the train.

\paragraph{Input 5: \texttt{modeAndLevel}}
The input is generated by the ``Mode and level management'' module. It provides the current level and mode of the EVC.

\paragraph{Input 6: \texttt{tNvContact}}

For monitoring the safe radio connection, the national value \texttt{T\_NVCONTACT} is needed as an input.

\paragraph{Input 7: \texttt{lastRelevantEventTimestamp}}

For monitoring the safe radio connection, it's necessary, that the time between two packets is less than the value of \texttt{T\_NVCONTACT}.

In situations like level-changes or announced radioholes, not the timestamp of the last message is relevant for comparison, but the timestamp of the last relevant event. This can be e.g. the timestamp of the level change or the timestamp of the timestamp of the moment, when the train was passing the end of the radiohole. 

For performing this check, the timestamp of the last relevant event is provided to the model as an \texttt{T\_internal\_Type}-type.

\paragraph{Input 8: \texttt{connectionStatus}}
The input \texttt{connectionStatus} will give information about the radio connection. This input is delivered by the session management module, not from the API. The information is needed to perform the timing check, which is depending on the connection state.

\begin{table}[H]
  \begin{tabular}{| l | p{9cm} |}
    \hline
    \textbf{Value} & \textbf{Interpretation}\\ \hline
    DISCONNECTED & The OBU is currently not connected to a RBC.\\
    CONNECTING & The OBU is currently connecting to the RBC. Received messages belong to the process of establishing a connection.\\
    CONNECTION\_ESTABLISHED &  The connection to RBC is established.\\
    \hline
  \end{tabular} 
  \caption{Possible values for the input \texttt{connectionStatus}}
  \label{tbl:connectionStatus}
\end{table}

\paragraph{Input 9: \texttt{inSupervisingRbcId}}
For the submodule ``Information Filter'', the information is needed, which radio messages are sent by the supervising RBC. To recognize these messages, the identifier of the supervising RBC is needed.

\paragraph{Input 10: \texttt{inAnnouncedBGs}}
This input provides information about balise groups which will be passed by the train soon. This information is generated by ``Calculate Train Position'' based on the linking information received from trackside.

\paragraph{Input 11: \texttt{q\_nvlocacc}}
The national value determines the location accuracy and is delivered by the database.



\subsubsection{Output}
The output of the module provides the received and processed Euroradio and Eurobalise messages. The module combines messages both from Eurobalises and from Euroradio to one common dataflow.

An overview over the output dataflows is provided in table \ref{tbl:ReceiveMessageAndCheckConsistencyOutput}.

\begin{table}[H]
 \footnotesize
  \begin{tabular}{| c | l | l | l |}
    \hline
    \textbf{Index} & \textbf{Output name} & \textbf{Output type}\\ \hline
    0 & \texttt{outputMessage} & \texttt{Common\_Types\_Pkg::ReceivedMessage\_T}\\
    1 & \texttt{ApplyServiceBrake} & \texttt{bool}\\
    2 & \texttt{BadBAliseMessageToDMI} & \texttt{bool}\\
    3 & \texttt{errorLinkedBG} & \texttt{bool}\\
    4 & \texttt{errorUnlinkedBG} & \texttt{bool}\\
    5 & \texttt{passedBG} & \texttt{BG\_Types\_Pkg::passedBG\_T} \\
    6 & \texttt{outPositionParams} & \texttt{Common\_Types\_Pkg::PositionReportParameter\_T} \\
    7 & \texttt{outRadioManagement} & \texttt{Common\_Types\_Pkg::radioManagementMessage\_T} \\
    8 & \texttt{radioSequenceError} & \texttt{bool} \\
    9 & \texttt{radioMessageConsistencyError} & \texttt{bool} \\
    \hline
  \end{tabular} 
  \caption{Dataflow at output}
  \label{tbl:ReceiveMessageAndCheckConsistencyOutput}
\end{table}

\subparagraph{Output 0: \texttt{outputMessage}}
The element \texttt{outputMessage} consists of the type \texttt{ReceivedMessage\_T} combines both balise and radio messages to one common datatype. This datatype contains all variables and packets, which are possible for the given scenario.

\begin{table}[H]
  \scriptsize
  \begin{tabular}{| l | l | p{5.5cm} |}
  \hline
  \textbf{Name} & \textbf{Datatype} & \textbf{Description}\\ \hline
  \texttt{valid} & \texttt{bool} & true, if no consistency errors were detected.\\
  \texttt{source} & \texttt{Common\_Types\_Pkg::MsgSource\_T} & Defines, if this is a Euroradio or Eurobalise message.\\
  \texttt{packetMetadata} & \texttt{Common\_Types\_Pkg::Metadata\_T} & contains the metadata of the packets\\
  \texttt{radioMetadata} & \texttt{Common\_Types\_Pkg::RadioMetadata\_T} & contains the metadata of the radio specific header variables\\
  \texttt{BG\_Common\_Header} & \texttt{BG\_Types\_Pkg::BG\_Header\_T} & Header of Eurobalise message\\
  \texttt{Radio\_Common\_Header} & \texttt{Radio\_Types\_Pkg::Radio\_TrackTrain\_Header\_T} & Header of Euroradio message\\
  \texttt{packets} & Common\_Types\_Pkg::Packets\_T & Structure of packets in messages\\
  \hline
\end{tabular}
  \caption{Structure of \texttt{ReceivedMessage\_T}}
  \label{tbl:receivedMessage_structure}
\end{table}

The Eurobalise-common-header \texttt{BG\_Header\_T} consists of the fields visible in the SCADE-declaration. The structure corresponds to the structure defined in the SRS chapter 8.4.2.1. Some fields were removed since they are not needed anymore for further processing after building messages from separate telegrams.

The Euroradio-common-header \texttt{Radio\_TrackTrain\_Header\_T} consists of the fields visible in the SCADE declaration. The structure corresponds to the structure defined in the SRS chapter 8.4.4.6.1. The structure contains all variables required by possible \texttt{NID\_MESSAGE} values for the given scenario. Which values are valid is defined in the field \texttt{radioMetadata}.

%\textbf{TODO:} Different definition of Radio-header than in SCADE!

%\textbf{TODO:} Note on packet type definitions and implementation details (which values were not used).

%\textbf{Note:} Packet 44 not used (applications outside the ERTMS/ETCS system are not supported by this implementation).

%\textbf{TODO:} Define packets 136, 12 in SCADE.

\subparagraph{Output 1: \texttt{ApplyServiceBreak}}
The flag indicates the balise group the train just passed could not be processed correctly. The check results in the request for a service break.

\subparagraph{Output 2: \texttt{BadBaliseMessageToDMI}}
Information to be passed to the DMI to indicate the reception of a ``bad balise'' to the driver.

\subparagraph{Output 3: \texttt{errorLinkedBG}}

\begin{table}[H]
  \begin{tabular}{| l | p{9cm} |}
    \hline
    \textbf{Value} & \textbf{Interpretation}\\ \hline
    true & A error in a linked balise group was detected.\\
    false & No error in a linked balise group was detected.\\
    \hline
  \end{tabular} 
  \caption{Possible values for the input \texttt{errorLinkedBG}}
  \label{tbl:errorLinkedBG}
\end{table}

\subparagraph{Output 4: \texttt{errorUnlinkedBG}}

\begin{table}[H]
  \begin{tabular}{| l | p{9cm} |}
    \hline
    \textbf{Value} & \textbf{Interpretation}\\ \hline
    true & A error in an unlinked balise group was detected.\\
    false & No error in an unlinked balise group was detected.\\
    \hline
  \end{tabular} 
  \caption{Possible values for the input \texttt{errorUnlinkedBG}}
  \label{tbl:errorUnlinkedBG}
\end{table}

\subparagraph{Output 5: \texttt{passedBG}}
The output \texttt{passedBG} provides the received balise group message in a special format needed by the module ``Calculate train position''.

\subparagraph{Output 6: \texttt{outPositionParams}}
The output \texttt{outPositionParams} provides the parameters for the position report in a special format needed by the module ``Provide Position Report''.

\subparagraph{Output 7: \texttt{outRadioManagement}}
The output \texttt{outRadioManagement} provides the messages for radio session management in a special format needed by the module ``Management of Radio Communication''.

\subparagraph{Output 8: \texttt{radioSequenceError}}

\begin{table}[H]
  \begin{tabular}{| l | p{9cm} |}
    \hline
    \textbf{Value} & \textbf{Interpretation}\\ \hline
    true & A sequence error or a timeout has been detected in the radio message.\\
    false & No error in the radio message sequence was detected.\\
    \hline
  \end{tabular} 
  \caption{Possible values for the input \texttt{radioSequenceError}}
  \label{tbl:radioSequenceError}
\end{table}

\subparagraph{Output 9: \texttt{radioMessageConsistencyError}}

\begin{table}[H]
  \begin{tabular}{| l | p{9cm} |}
    \hline
    \textbf{Value} & \textbf{Interpretation}\\ \hline
    true & A consistency error has been detected in the radio message.\\
    false & No consistency error in the radio message was detected.\\
    \hline
  \end{tabular} 
  \caption{Possible values for the input \texttt{radioMessageConsistencyError}}
  \label{tbl:radioMessageConsistencyError}
\end{table}


\subsubsection{Receive\_TrackSide\_Msg in Manage\_TrackSideInformation\_Integration}
\paragraph{Reference to the SRS (or other requirements)}
\begin{itemize}
  \item Definition of the Balise Telegram: subset 26 section 7 and 8
  \item Interface to the BTM: Subset 36, section  4.2.2, 4.2.4, 4.2.9
  \item Handling of Balise Telegrams: Subset 26, sections 3.4.1 - 3.4.3, 3.16.2
  \item Check of the balise group Subset 26, section 3.16.2
  \item Determining the Orientation: 3.4.2
  \item Active Functions Table: 4.5.2
  \item Rules for Euroradio messages: Subset 26, chapter 8.4.4

\end{itemize}
\paragraph{Short description of the functionality}
This function defines the interface of the OBU model to the openETCS generic API for Eurobalise  and Euroradio messages. On the interface, either a valid telegram/message is provided or a telegram/message is indicated which could not be received correct when passing the balise or receiving the radio message. The function passes a balise telegram without major changes of the information to the next entity for collecting the balise group information. This entity collects telegrams received via the interface into Balise Group Information. In case of a radio message, the message is converted to an internal format for further processing and passed without changing the information contained.

\paragraph{Interface}
\paragraph{Functional Design Description}
\textbf{Design Constraints and Choices}
\begin{enumerate}
\item The decoding of balises is done at the API. Also, packets received via the interface are already transformed into a usable shape.
\item Only packets used inside the current model are passed via the interface.\\
\item Treatment of Packet 5: Linking Information.\\
Linking Information is added to the linking array starting from index 0 without gaps. Used elements are marked as valid. Elements are sorted according to the order given by the telegram sequence.
\item Telegrams received as invalid are passed to the ``Check-Function'' to process errors in communication with the track side according to the requirements and in a single place.
Telegrams are added to the telegram array starting from index 0 without gaps. Used elements are marked as valid. Elements are stored according to the order given by the telegram sequence.
\item This function does not process information from the packets. The information is passed to the check without further processing of the values. 
\end{enumerate}
\paragraph{Reference to the Scade Model}
The SCADE model can be found on github under the following path:

\tiny\url{https://github.com/openETCS/modeling/tree/master/model/Scade/System/ObuFunctions/ManageLocationRelatedInformation/BaliseGroup/Receive_TrackSide_Msg}
\normalsize
\subsubsection{CheckBGConsistency in Manage\_TrackSideInformation\_Integration}%Mainfunction receive track data. Name should be be defined and substituded by the designer of the function. 
\paragraph{Reference to the SRS or other Requirements (or other requirements)}
\begin{itemize}
  \item Definition of the Balise Telegram: subset 26 section 7 and 8
  \item Handling of Balise Telegrams: Subset 26, sections 3.4.1 - 3.4.3, 3.16.2
  \item Check of the balise group Subset 26, section 3.16.2
  \item Active Functions Table: 4.5.2
\end{itemize}
\paragraph{Short description of the functionality}
This function has the task  to verify the completeness and correctness of the received messages from balise groups.\\
A message consists of at least a telegram and a maximum of 8 telegrams.\\

\begin{itemize}
\item A message is still complete and correct, if a telegram is missing (or not decoded or incomplete decoded ), and this telegram is duplicated within the balise group and the duplicating one is correctly read.
\item By more than one telegram, the order of the telegrams must be either ascending (nominal) or descending(reverse).\\
\item A message is correct, if  all message counters (M MCUNT) do not equal 254 (that means: The telegram never fits any message of the group).\\ A message counter can be equal 255 (that means: The telegram fits with all telegrams of the same balise group) and all other values must be the same.\\
\end{itemize}

\paragraph{Interface}

\paragraph{Functional Design Description}
This function is active in certain modes and the output and reactions are dependent on if the linking information is used.\\
The orientation of the BG will also be calculated in this block.\
The check, if the message has been received in due time and the right at the right expected location, will be performed in "Calculate Train Position".\\
The checks on the validity of the data in the packets and the validity with respect to the direction of motion will be performed in other modules, e.g. "Validate Data Direction" .

\paragraph{Reference to the Scade Model}
The SCADE model can be found on github under the following path:

\tiny\url{https://github.com/openETCS/modeling/tree/master/model/Scade/System/ObuFunctions/ManageLocationRelatedInformation/BaliseGroup/CheckBGConsistency}
\normalsize

\subsubsection{CheckEuroradioMessage in Manage\_TrackSideInformation\_Integration}%Mainfunction receive track data. Name should be be defined and substituded by the designer of the function. 
\paragraph{Reference to the SRS or other Requirements (or other requirements)}
\begin{itemize}
 \item SRS subset 26, chapter 8.4.4: Rules for Euroradio messages
 \item SRS subset 26, chapter 3.16: Data consistency
\end{itemize}
\paragraph{Short description of the functionality}

The function ``CheckEuroradioMessage'' has to perform several checks on the received radio message. These checks include checking of the message sequence, completeness of messages. Invalid messages are marked as invalid in the header.

The bitwalker is responsible for checking the validity of the values in fields. If a consistency error is detected by the bitwalker, it is signalled to the model. If the bitwalker marks a packet as valid, all variables are expected to contain a valid value.

\paragraph{Interface}

\paragraph{Functional Design Description}
\begin{itemize}
 \item Content checks
 \begin{itemize}
    %\item The computed length of the message must be equal to the value in \texttt{L\_MESSAGE}. (SRS 8.4.4.2.1)
    \item The whole message must be complete and contains all necessary fields. (SRS 3.16.1.1)
    \item The message must respect the ETCS language. (SRS 3.16.1.1)
    \item The variables of the message does not contain invalid values. (SRS 3.16.1.1) % already done by API?
    \item Check if the specified priority of message is equal to the priority with which the message was received. (SRS 3.16.3.1.3.1) 
  \end{itemize}
  \item Timing checks
  \begin{itemize}
    \item Check if the timestamp of a message is greater than the timestamp of the former message (SRS 3.16.3.3.3)
    \item If a message contains the timestamp ``Unknown'', check if this message is part of the initiation of the communication session. (SRS 3.16.3.3.4)
    \item Perform the check with the current packet $n$:  $T\_TRAIN_{n} <= T\_TRAIN_{n-1} + T\_NVCONTACT$ (SRS 3.16.1.1). This ensures, that the packet was received in due time.
  \end{itemize}
\end{itemize}

For inconsistent messages, the following actions need to be performed by the module:

\begin{itemize}
  \item If a message is not consistent, it shall be rejected (SRS 3.16.3.1.1.1). For this purpose, the message is marked as invalid.
  \item The RBC shall be informed, when a message was rejected (SRS 3.16.3.1.1.2). Therefore the necessary information for creating an error report is provided as an output. 
  %\item If the RBC requested an ACK for a received message, message will be marked for the module to send a report to the RBC. (SRS 3.16.3.5)
  \item This module will not trigger the reaction for an interrupted radio connection to the RBC. The reaction sepcified by \texttt{M\_NVCONTACT} will be triggered by the RBC session management module.
\end{itemize}

The check by the Euroradio-protocol (SRS 3.16.3.1.1) will not be performed by the model, but on a lower level (RTM or openETCS-API).

Safe connection supervision is not in the scope of this module. This functionality will be implemented by the ``Manage Radio communication'' module.

\paragraph{Reference to the Scade Model}
The SCADE model can be found on github under the following path:

\tiny\url{https://github.com/openETCS/modeling/tree/master/model/Scade/System/ObuFunctions/ManageLocationRelatedInformation/BaliseGroup/CheckEuroRadioMessage}
\normalsize

\subsubsection{ValidateDataDirection in Manage\_TrackSideInformation\_Integration}

\paragraph{Reference to the SRS or other Requirements (or other requirements)}
\begin{itemize}
 \item The functionality is mainly described in \cite[Chapter~3.6.3]{subset-026}.
\end{itemize}
\paragraph{Short description of the functionality}
This function determines for direction information of the LRBG, an (ordinary) balise group or a radio message, whether this information is valid or not. The function takes as an input the LRBG, the balise groups passed and the train position and outputs the input extended with validity information.

\paragraph{Interface}
\paragraph{Functional Design Description}
\begin{itemize}
 \item The module contains two processing paths for either a radio message or for a balise message.
\end{itemize}

\paragraph{Reference to the Scade Model}
The SCADE model can be found on github under the following path:

\tiny\url{https://github.com/openETCS/modeling/tree/master/model/Scade/System/ObuFunctions/ManageLocationRelatedInformation/BaliseGroup/ValidateDataDirection}
\normalsize

\subsubsection{InformationFilter in Manage\_TrackSideInformation\_Integration}%Mainfunction Train Supervision. Name should be be defined and substituded by the designer of the function. 
\paragraph{Reference to the SRS or other Requirements (or other requirements)}
\begin{itemize}
 \item The functionality of Select Usable Info is described in Chapter 4.8 of subset-026 \cite{subset-026}. The following list gives an overview of the most important sections for each of the blocks in the model.
 \item § 4.8.2, § 4.8.2, § 4.8.3, § 4.8.4
\end{itemize}

\paragraph{Short description of the functionality}
The function Select Usable Info filters information received from balises that have been passed, radio messages, and EUROLOOP messages. Filtering is done depending on the mode of the train, the current ETCS level, the type/content of the information, and the transition media of the information. As neither radio messages nor EUROLOOP are part of the first iteration of work, not all functionality of the filter described in the specification is currently implemented.
\paragraph{Interface}
\textbf{Input from:} Receive MSG Check Consistency/Coordinate System - track messages and package\\
Level and Mode Management - Mode and Level State\\

\textbf{Output to:} Build Data structure and Location Based/ Build Data Structures Drivers- forwarded packages, messages and variables\\

\begin{figure}[hbtp]
\centering
\includegraphics[scale=0.7]{images/FilterInandOUt}
\caption{Filter In and out}
\end{figure}

\subsubsection{SysML Model}
\begin{figure}[hbtp]
\centering
\includegraphics [scale=0.5]{images/SysMLFilter}
\caption{SysML Filter}
\end{figure}

\paragraph{Functional Design Description}
The fillter receives track information (balise an radio) and will filter them in dependency of the mode and level.
Therefore the filter needs the input from level and mode management. The filtered information will be forwarded to the data strcuture.

\begin{description}
\item[First filter] The first filter, i.e.~the filter on the level, is described in \cite[Chapter~4.8.3]{subset-026}.
\item[Second filter] The second filter, i.e.~the filter on the transition media, is described in\cite[Chapter~4.8.3]{subset-026}.
\item[Third filter]
 The third filter, i.e.~the filter on the modes, is described in \cite[Chapter~4.8.4]{subset-026}.
\item[Transition buffers] Details on the handling of the transition buffers used in the first and the second filter are described in \cite[Chapter~4.8.5]{subset-026}.
\end{description}

%%%% 
\subparagraph{Documentation of design}
From § 4.8.1.2 The following sections have to be interpreted by applying the filters and the assigned packets/messages as shown in Figure a and 2. The first filter is detailed in section § 4.8.3 (figure 1) “Accepted information depending on the level and transmission media”, the third filter in section § 4.8.4 (figure 2) “Accepted information depending on the modes”.\\

From § 4.8.1.3 If a message contains level transition information, any other information in that message shall be evaluated considering the level transition information. Explanation: If a message contains level transition information, all other information in that message shall be buffered and level transition shall be read first. Then the remained balise information shall be read from the buffer in the level that was announced to the balise.\\

From § 4.8.1.3.1 Information received in the same message as an immediate level transition order or a conditional level transition order that causes a level transition shall be evaluated first considering the on-board currently operated level, as if a level transition order for further location had been received (i.e. conditions [1], [2] or [6] of Figure 1, if applied, shall be automatically fulfilled). Then, if relevant, it shall be immediately extracted from the buffer and re-evaluated according to the new selected level.\\
\textbf{Explanation:} As described in Explanation of § 4.8.1.3 and figure 1 – First Filter conditions [1], [2] and [6])\\

From § 4.8.1.4 Note: As shown in Figure 1, information stored following an announcement of a change of level, is re-checked for acceptance when the level has changed. This implies that, when the level changes, the mode is - for a short moment – still unchanged, until the stored information has been processed. The consequence for the Third Filter is that information needs to be accepted for this short period also in modes in which this information is otherwise useless.\\
\textbf{Explanation:} when a level announced the level the mode change will be unchanged until the buffered information has been processed. The model change is the third filter (§ 4.8.3 figure 3).\\

\subparagraph{table for the filter rules}
\textbf{Assumptions from § 4.8.2 need to be considered}\\
\textbf{Explanation}: See figure 1 and 2 – announced packets/messages/variables to the filter. Exception and explanation of the meaning of R and A please read § 4.8.3.\\. 

\textbf {Filter rules}: Filter will filter messages, packages and variables. Therefole a rule must be defined to cover all these inputs.\\

\textbf {Explanation figure 1}: will filtering the different inputs in dependency of the level\\
\textbf {Explanation figure 2}: will filtering the different inputs in dependency of thel mode\\

\subparagraph{Filter on Level}
\begin{figure}[hbtp]
\centering
\includegraphics [scale=0.6]{images/LevelFilter1}
\end{figure}
\begin{figure}[hbtp]
\centering
\includegraphics [scale=0.6]{images/LevelFilter2}
\end{figure}
\begin{figure}[hbtp]
\centering
\includegraphics [scale=0.6]{images/LevelFilter3}
\end{figure}
\begin{figure}[hbtp]
\centering
\includegraphics [scale=0.6]{images/LevelFilter4}
\caption{Level Filter}
\end{figure}
\newpage

\subparagraph{Filter on Modes}
\begin{figure}[hbtp]
\centering
\includegraphics [angle=90, scale=0.8]{images/FilterMode1}
\end{figure}
\begin{figure}[hbtp]
\centering
\includegraphics [angle=90, scale=0.8]{images/FilterMode2}
\end{figure}
\begin{figure}[hbtp]
\centering
\caption{Mode Filter}
\end{figure}
\newpage

\subparagraph{Filtering (Mode/Level) - One packet per type}
\textbf{ISSUE: HOW MANY PKT 44, 65 AND 66 PER MESSAGE ARE MAXIMALLY SUPPORTED? (BH: who made this comment??)}\\

- Check on announced and immediate level transition orders in the messages to be filtered (needed for further criteria for filtering, to decide if the data shall be stored in the transition buffer).\\
- Filter data stored in the transition buffer according to the current level (what to do if similar information is available in the new message??). Data can be rejected, accepted or kept in the transition buffer.
(Filtering according to new level will be done directly afterwards in the next cycle)\\
- Filter new received messages according to the current level (new level will be done in the next cycle as according to \gls{SRS} data first has to be filtered according to old level and afterwards to new level). Data can be rejected, accepted or stored in the transition buffer.\\
- Filter (level) accepted data according to originating RBC (supervising or other). Information from \gls{BG}'s, loops or RIU is not filtered with this filter.\\
- Filter (level and RBC) accepted data according to the current mode (only reject or accept)\\
\paragraph{Reference to the Scade Model}
The SCADE model can be found on github under the following path:

\tiny\url{https://github.com/openETCS/modeling/tree/master/model/Scade/System/ObuFunctions/ManageLocationRelatedInformation/BaliseGroup/InformationFilter}
\normalsize%Mainfunction receive track data. Name should be be defined and substituded by the designer of the function. 

%-----------------------------------------------------------------------
\subsection{Train Supervision}
%-----------------------------------------------------------------------
%\tbc
%Group 1 (Christian Stahl)

\begin{figure}[!h]
\centering
\includegraphics[width=0.95\textheight, angle=90]{../images/speedsupervision.PNG}
\caption{Structure of component ProvidePositionReport}\label{fig:ssv}
\end{figure}

The task of block ``Train Supervision'' is to monitor the speed of the train and the train location and as such to ensure that the speed remains within the given speed and distance limits. This block is mainly based on \cite[Chapt.~3.13]{subset-026}.

The block ``Train Supervision'' takes as input (1) movement related information such as train speed, train position and acceleration, (2) train related information such as brake information and train length, and (3) track related information such as speed and distance limits and national values.

Based on this information a speed profile is calculated. Speed restrictions create target speeds (targets) that have to be followed. For each such target braking curves are generated to supervise at which location of the track the train must perform the brake. In case of no target restrictions the train may accelerate to the supervised maximum speed of the speed profile. These calculations lead to commands being sent to the driver and the brake system.

The functionality is modeled using four operators, as shown in Figure~\ref{fig:ssv}, which are explained below.

The current status of the analysis of ``Train Supervision'' and a functional breakdown can be found in a separate document, \verb+SpeedSupervision_analysis.pdf+.



\subsubsection{Input}
For providing the output, the module needs different input data flows. Table \ref{tbl:speedsupervisionInput} gives an overview.

\begin{table}[H]
  \begin{tabular}{| c | l | l | l | l |}
    \hline
    \textbf{Index} & \textbf{Input name} & \textbf{Input type} & \textbf{Source}\\ \hline
    0 & \texttt{MRSP} & \texttt{MRSP\_Profile\_t} & ?? \\
    1 & \texttt{MA} & \texttt{MAs\_t} & ??\\
    2 & \texttt{NationalValues} & \texttt{P3\_NationalValues\_T} & ???\\
    3 & \texttt{TrainPosition} & \texttt{trainPosition\_T} & Manage Track Data\\
    4 & \texttt{odometry} & \texttt{odometry\_T} & Odometry\\
    5 & \texttt{m\_level} & \texttt{M\_LEVEL} & Mode and Level\\
    6 & \texttt{trainProps} & \texttt{trainProperties\_T} & Database\\
    7 & \texttt{MA\_updated} & \texttt{bool} & internal\\
    8 & \texttt{MRSP\_updated} & \texttt{bool} & internal\\
    \hline
  \end{tabular} 
  \caption{Overview of input}
  \label{tbl:speedsupervisionInput}
\end{table}

\paragraph{Input 0: \texttt{MRSP}}
This input is the most restrictive speed profile.
\paragraph{Input 1: \texttt{MA}}
This input is a movement authority.
\paragraph{Input 2: \texttt{NationalValues}}
This input is packet 3 of \cite[Chapt.~8]{subset-026}, describing the national values. 
\paragraph{Input 3: \texttt{TrainPosition}}
This input is the current train position.
\paragraph{Input 4: \texttt{odometry}}
This input is the odometry data.
\paragraph{Input 5: \texttt{m\_level}}
This input is the current level of the train.
\paragraph{Input 6: \texttt{trainProps}}
This input is a set of train related properties.
\paragraph{Input 7: \texttt{MA\_updated}}
This flag is true if the movement authority has been updated in this clock cycle and false otherwise.
\paragraph{Input 8: \texttt{MRSP\_updated}}
This flag is true if the most restrictive speed profile has been updated in this clock cycle and false otherwise.



\subsubsection{Output}
Based on the input the block produces the following output. Table~\ref{tbl:speedsupervisionOutput} gives an overview.

\begin{table}[H]
  \begin{tabular}{| c | l | l | l |}
    \hline
    \textbf{Index} & \textbf{Output name} & \textbf{Output type}\\ \hline
    0 & \texttt{sdmToDMI} & \texttt{speedSupervisionForDMI\_T}\\
    1 & \texttt{target} & \texttt{Target\_T}\\
    2 & \texttt{sdmCommands} & \texttt{SDM\_Commands\_T}\\
    3 & \texttt{brakeCmd} & \texttt{Brake\_command\_T}\\
    4 & \texttt{EOA\_overpassed} & \texttt{bool}\\
    5 & \texttt{Target\_Speed\_Reached} & \texttt{bool}\\
    \hline
  \end{tabular} 
  \caption{Overview of output}
  \label{tbl:speedsupervisionOutput}
\end{table}

\paragraph{Output 0: \texttt{sdmToDMI}}
This output contains information about different speeds and positions, on the one hand and the current supervision status, on the other hand. This information shall be displayed to the driver.
\paragraph{Output 1: \texttt{target}}
This output is the most restrictive displayed target (MRDT).
\paragraph{Output 2: \texttt{sdmCommands}}
This output gives some intermediate results of operator SDM\_Commands. It is currently used for test purposes only.
\paragraph{Output 3: \texttt{brakeCmd}}
This output is the brake command, indicating whether performing the service brake or the emergency brake have been commanded.
\paragraph{Output 4: \texttt{EOA\_overpassed}}
This output is true if the end of authority has been overpassed and false otherwise.
\paragraph{Output 5: \texttt{Target\_Speed\_Reached}}
This output is true if the current speed is greater than or equal the target speed and false otherwise.



\subsubsection{SDM\_InputWrapper in Train Supervision}

\paragraph{Reference to the SRS or other Requirements (or other requirements)}
\begin{itemize}
	\item \cite[Chapt.~3.13]{subset-026}: Speed and distance monitoring 
\end{itemize}

\paragraph{Short description of the functionality}
The motivation for this operator is to convert all inputs of block ``Speed Supervision'' that contain information about length, speed, distance, and acceleration defined as integer into \texttt{real} to allow automatically the highest precision in the calculations by the meaning of floating point operations. In addition, to ease the modeling, inside block ``Speed Supervision'' only units meters ($[m]$), seconds($[s]$), meters per second($[\frac{m}{s}]$), and meters per square second($[\frac{m}{s^{2}}]$) are used.

\paragraph{Interface}

\paragraph{Functional Design Description}
This operator forwards input messages, takes data from complex data types or transforms inputs messages into an internal type thereby converting int to real.
  
\paragraph{Reference to the Scade Model}
\textbf{only in special case or link to the Scade model}

\subsubsection{TargetManagement in Train Supervision} \marginpar{Ben}
\paragraph{Reference to the SRS or other Requirements (or other requirements)}
\begin{itemize}
	\item \cite[Chapt.~3.13.8.2]{subset-026}: Determination of the supervised targets 
\end{itemize}

\paragraph{Short description of the functionality}
This operator calculates/updates the list of targets to be supervised by the block ``Train Supervision''. Taking the current movement authority and the most restrictive speed profile as an input, the operator outputs a list of locations corresponding to the most restrictive speed profile, a list of locations corresponding to a limit of authority, the location of an end of authority, or the location of supervised location.

The operator takes as Inputs the Movement Authority, Most Restrictive Speed Profile and the current maximum safe front end position. 

The Output is a single End of Authority Target, a list of all MRSP-Targets and a list of all LoA-Targets.
  
\paragraph{Interface}

\paragraph{Functional Design Description}
\subparagraph{Derivation of Targets from Movement Authority Sections}
The sections of the \emph{Movement Authority} could cause 2 types of Targets:
\begin{itemize}
\item End Of Authority(EoA) - only one could exist and this is only in the \emph{end section} of the \emph{MA}
\item Limit of Authority (LoA) - is possibly in every section of the \emph{MA} except the end section
\end{itemize}
In every Cycle the MA is updated, the operator iterates over the whole MA and put all speed limitations by \emph{LoA}s into a list of targets. The end section is used to derived the \emph{EoA}-Target. All LoA targets are sorted by location.

\subparagraph{Derivation of Targets from MRSP}
As the \cite[Chapt.~3.13.8.2]{subset-026} says, every speed decrease of the MRSP is used to derive a target. Therefore in every cycle the MRSP is updated, the whole MRSP is iterated to search for all MRSP Targets. For this purpose, every element of the MRSP is compared with its successor. When the calculation arrives the last valid element, the calculation is exited.

\subparagraph{Updating of Targets}
In every cycle the operator monitors all targets, whether they are already passed by iterating over the list of targets comparing the current max safe front end position with the target position.

 


\paragraph{Reference to the Scade Model}
\textbf{only in special case or link to the Scade model}

\subsubsection{CalcBrakingCurves\_Integration in Train Supervision} \marginpar{Ben}
\paragraph{Reference to the SRS or other Requirements (or other requirements)}
\begin{itemize}
	\item \cite[Chapt.~3.13.8.3]{subset-026}: Emergency Brake Deceleration curves (EBD)
	%
	\item \cite[Chapt.~3.13.8.4]{subset-026}: Service Brake Deceleration curves (SBD)
	%
	\item \cite[Chapt.~3.13.8.5]{subset-026}: Guidance curves (GUI)
\end{itemize}
\paragraph{Short description of the functionality}
For each type of target a certain braking curve has to be calculated. This curve enables us, to proactively monitor the speed of the train. From this braking curve could be derived, where the train has to start braking for a given actual speed. The braking curve is not depended on actual train status, as a consequence the braking curve stay constant over a certain time. Only to stop the calculation of the braking curve the estimated front end position is used, but this is only to have an always valid exit condition.
\paragraph{Interface}
\paragraph{Functional Design Description}
The braking curve calculation takes the complex input function $A_{safe}$, which describes the overall braking performance of the train in a speed and position depended meaning. This two dimensional function needs to be simplified for every target to get an function of position to speed. The individual target has a position and a speed (a point in the distance speed plain) and is the starting point of the braking curve. The initial step is to getting the deceleration of the train at the target point from the $A_{safe}$-function. Afterwards the calculation iterates through the $A_{safe}$-function until the current estimated front end position is reached. While this two cases could appear:

\begin{itemize}
\item a new distance step of $A_{safe}$ is reached
\item a new speed step of $A_{safe}$ is reached
\end{itemize}

Both cases are checked and the applicable one is used to calculate a new Arc. Every Arc of the braking curve consists of:
\begin{itemize}
\item distance where the arc begins
\item speed at the point, where the arc begins
\item deceleration for the whole arc
\end{itemize}
An abstract overview of the calculation could be seen in picture \ref{fig:bc_calc}.
\begin{figure}[!h]
\centering
\includegraphics[width=0.95\textwidth]{../images/EBD_CalcAlgorithm.png}
\caption{Calculation of Braking Curves}\label{fig:bc_calc}
\end{figure}

Currently, the model supports the calculation of the following braking curves:
\begin{itemize}
	\item the Emergency Brake Deceleration curve for the most restrictive speed profile,
	%
	\item the Emergency Brake Deceleration curve for the limit of authority,
	%
	\item the Emergency Brake Deceleration curve for the end of authority, and
	%
	%\item the Service Brake Deceleration curve for the end of authority
\end{itemize}
\paragraph{Reference to the Scade Model}
\textbf{only in special case or link to the Scade model}

\subsubsection{SDMLimitLocations in Train Supervision} \marginpar{Tho}
\paragraph{Reference to the SRS or other Requirements (or other requirements)}
\begin{itemize}
	\item \cite[Chapt.~3.13.9]{subset-026}: Supervision Limits 
	\item \cite[Chapt.~5.3.1.2]{subset-041}: $f_{41}$ -- accuracy of speed known on-board
	\item \cite[Chapt.~3.13.10]{subset-026}: Monitoring Commands as reference for required outputs of this module
\end{itemize}

\paragraph{Short description of the functionality}
This operator calculates the various locations needed to determine the speed and distance monitoring commands. The current implementation of functionality is stateless and requires a complete recalculation each cycle.

\paragraph{Interface}
\subparagraph{Input}
\begin{enumerate}
  \item \texttt{MRSPProfile} Speed profile related to current track under train.
  \item \texttt{odometry}, \texttt{trainLocations} External state of train provided by odometry.
  \item \texttt{targetCollection} The different target (list) types wrapped in a structure.
  \item \texttt{curveCollection} The related braking curves correlated to above targets.
  \item \texttt{$v_{release}$} Release speed as defined by external sources.
  \item \texttt{$v_{ura}$} Speed under reading amount.
  \item \texttt{inhibitUnderReadingCompensation} A flag defined by National Value, relating to above item.
  \item \texttt{$T_{bs}$, $T_{be}$, $T_{tractionCutOff}$} Time constants defined externally or in other modules.
\end{enumerate}
\subparagraph{Output}
\begin{enumerate}
  \item \texttt{locations} Internal type to wrap the locations calculated herein and pass it on directly to SDM-Commands-Operator.
  \item \texttt{MostRestrictiveTarget} An internal structure to contain the information the target based locations are linked to.
  \item \texttt{FLOIisSBI1} Flag is true if First Line of Intervention uses the service brake curve (SBI1) or false if it uses deceleration values based on the emergency brake curve (SBI2).
  \item \texttt{$v_{target}$} The designated speed of the Most Restrictive Target. This is a convienience reference into the above data structure. 
  \item \texttt{$v_{MRSP}$} The current Most Restrictive Speed at the Max Save Frontend of the train.
\end{enumerate}

\paragraph{Functional Design Description}
This operator gathers all necessary input values and computes some frequently used intermediate values in the suboperators \texttt{surplusTractionDeltas} and \texttt{$v_{bec}$}. The other input prepartion operator is the \texttt{TargetSelector} whose main task is to dissect the list of targets to find the Most Restrictive Target. The accompanying braking curves are extracted and promoted to trailing location calculations. Also the special values of the EOA are exposed.

The other half of the operator creates the requested values for the commands package. These ar in particular the preindication locations for EBD and SBD based targets, the release speed monitoring start locations, the locations for target speed monitoring of the I-, W-, P- and FLOI-curve, the related FLOI speed and the location of the permitted speed supervision limit. Included in the output are also certain flags for the validity of linked values.

%%\paragraph{Reference to the Scade Model}
%%\textbf{only in special case or link to the Scade model}

\subsubsection{CalcSpeeds in Train Supervision}
\paragraph{Reference to the SRS or other Requirements (or other requirements)}
\begin{itemize}
	\item \cite[Chapt.~3.13.9]{subset-026}: Supervision Limits 
\end{itemize}
\paragraph{Short description of the functionality}
This operator calculates the various speeds needed to determine the speed and distance monitoring commands.
\paragraph{Interface}
\paragraph{Functional Design Description}
This operator will be integrated into other operators in the next iteration.
\paragraph{Reference to the Scade Model}
\textbf{only in special case or link to the Scade model}

\subsubsection{SDM\_Commands in Train Supervision}
\paragraph{Reference to the SRS or other Requirements (or other requirements)}
\begin{itemize}
	\item \cite[Chapt.~3.13.10]{subset-026}: Speed and distance monitoring commands 
\end{itemize}
\paragraph{Short description of the functionality}
This operator models the speed and distance monitoring commands. More precisely, it triggers the service or emergency brake and outputs the current supervision status of the OBU together with information on speeds and locations to the driver.
\paragraph{Interface}
\paragraph{Functional Design Description}
The OBU can be in any of three types of speed and distance monitoring modes: ceiling speed monitoring, release speed monitoring and target speed monitoring. We use a state machine to model the switching between the three modes: each state models a mode and a transition between to states is enabled if the condition two switch between the two corresponding modes is evaluated to true. In each mode, the OBU can be in up to five different supervision stati. The behavior of changing from one status to another is also modeled as a state machine. As a result, the model is a hierarchical state machine.
\paragraph{Reference to the Scade Model}
\textbf{only in special case or link to the Scade model}

\subsubsection{SDM\_OutputWrapper in Train Supervision}
\paragraph{Reference to the SRS or other Requirements (or other requirements)}
\begin{itemize}
	\item \cite[Chapt.~3.13]{subset-026}: Speed and distance monitoring 
\end{itemize}

\paragraph{Short description of the functionality}
This operator is the counterpart to operator SDM\_OutputWrapper---that is, it converts all internal outputs of block ``Speed Supervision'' that contain information about length, speed, distance, and acceleration defined as real into int, such that all other blocks can stick to their types and also performs the calculation into units used by the environment.

\paragraph{Interface}

\paragraph{Functional Design Description}
This operator forwards input messages and transforms inputs messages into an internal type thereby converting real to int.
  
\paragraph{Reference to the Scade Model}
\textbf{only in special case or link to the Scade model}%Mainfunction receive track data. Name should be be defined and substituded by the designer of the function. 

%-----------------------------------------------------------------------
\subsection{Manage ETCS Procedures}
%-----------------------------------------------------------------------
%\tbc
%Baseliyos Jacob

\subsubsection{Macrofunction x in Manage ETCS Procedures}%Mainfunction receive track data. Name should be be defined and substituded by the designer of the function. 
\paragraph{Reference to the SRS or other Requirements (or other requirements)}
\paragraph{Short descriptoiin of the functionality}
\paragraph{Interface}
\paragraph{Functional Design Description}
\paragraph{Refernce to the Scade Model}
\textbf{only in special case or link to the Scade model}

\subsubsection{Macrofunction x in Manage ETCS Procedures}%Mainfunction receive track data. Name should be be defined and substituded by the designer of the function. 
\paragraph{Reference to the SRS or other Requirements (or other requirements)}
\paragraph{Short descriptoiin of the functionality}
\paragraph{Interface}
\paragraph{Functional Design Description}
\paragraph{Refernce to the Scade Model}
\textbf{only in special case or link to the Scade model}

\subsubsection{Macrofunction x in Manage ETCS Procedures}%Mainfunction receive track data. Name should be be defined and substituded by the designer of the function. 
\paragraph{Reference to the SRS or other Requirements (or other requirements)}
\paragraph{Short descriptoiin of the functionality}
\paragraph{Interface}
\paragraph{Functional Design Description}
\paragraph{Refernce to the Scade Model}
\textbf{only in special case or link to the Scade model}

\subsubsection{Macrofunction x in Manage ETCS Procedures}%Mainfunction receive track data. Name should be be defined and substituded by the designer of the function. 
\paragraph{Reference to the SRS or other Requirements (or other requirements)}
\paragraph{Short descriptoiin of the functionality}
\paragraph{Interface}
\paragraph{Functional Design Description}
\paragraph{Refernce to the Scade Model}
\textbf{only in special case or link to the Scade model}

%Mainfunction receive track data. Name should be be defined and substituded by the designer of the function. 

%-----------------------------------------------------------------------
\subsection{Manage Track Data}
%-----------------------------------------------------------------------
%\tbc
%Jakob Gärtner

\subsection{F.2.2 Calculate Train Position}\label{sss:calctrainpos}

\subsubsection{Short Description of Functionality}
The main purpose of the function is to calculate the locations of linked and unlinked balise groups (BGs) and the current train position while the train is running along the track. 

\begin{figure}[hbtp]
\centering
\includegraphics[scale=1]{../images/CalculateTrainPosition.png}
\caption{Structure of calculateTrainPosition}
\end{figure}


\paragraph{Functional Structure in Stages}
The function calculateTrainPosition is divided into the following four functions, which are being performed sequentially: 
\begin{enumerate}
\item \textbf{\textit{calculateBGLocations}}: Calculate the balise group locations\\
The first stage is triggered each time the train passes a balise group (input \textit{passedBG}). It takes the balise group header with the BG identification, the linking information (Subset 26, packet 5) and the current odometry values as inputs and calculates the location of the the passed balise group. If the passed BG has been announced via linking information previously, it takes into account the linking as well as the odometry information. If the passed BG does not meet the tolerance window announced by linking, an error flag is set. If the passed BG is an unlinked BG, its location is determined by odometry only, but related to the next previously passed linked BG, if there is one.\\
Then, if the passed BG is a linked BG comprising linking information for BGs ahead, the linking information is evaluated by creating the announced BGs and computing their locations from the linking distances.\\
The passed and the announced BGs are stored in a list \textit{BGs}, ordered by their nominal location on the track.\\
Afterwards the locations of all BGs are further improved by re-adjusting their locations with reference to the just passed BG. This optimizes the BG location inaccuries around the current train position (= location of the passed BG). 

\item \textbf{\textit{delDispensableBGs}}: Delete dispensable balise groups\\
The second stage removes balise groups supposed not to be needed any longer from the list of \textit{BGs}.\\
If the number of stored passed linked BGs exceeds the maximum number of eight as specified in \cite[Chapter~3.6.2.2.2 c]{subset-026}, all BGs astern are deleted.
If only (passed) unlinked BGs are in the list and exceed the number of \textit{cNoOfAtLeast\_x\_unlinkedBGs}, all passed BGs astern to those are removed from the list. 

\item \textbf{\textit{calculateTrainPositionInfo}}: Calculate train position information.\\
This stage take the list of stored BGs and the current odometry values as inputs and steadily provides the current train position. 

\item \textbf{\textit{calculateTrainpositionAttributes}}: Calculate train position attribute information.\\
This stage provides several additional position related attributes that might conveniently be used by subsequent consumers in the architecture. It requires the actual LRBG and the previous LRBG to be assigned external from the list \textit{BGs}. 
\end{enumerate}

\subsubsection{Reference to the SRS (or other requirements)}
The component calculateTrainPosition determines the location of linked and unlinked balise groups and the current train position during the train trip as specified mainly in \cite[Chapter~3.6]{subset-026}.

\subsubsection{Design Constraints and Choices}
The following constraints and prerequisites apply:
\begin{enumerate}
\item The input data received from the balises groups must have been checked and filtered for validity, consistency and the appropriate train orientation before delivering them to calculateTrainPosition. 
\item The storage capacity for balise groups is finite. calculateTrainPosition will raise an error flag when a balise group cannot be stored due to capacity limitations.
\item calculateTrainPosition will raise an error flag if a just passed balise group is not found where announced by linking information. It will not (yet) detect when an announced balise group is missing. 
\item calculateTrainPosition is not yet prepared for train movement direction changes. 
\item calculateTrainPosition does not yet consider repositioning information.
\end{enumerate}



\subsection{Provide Position Report}\label{sss:provposrep}

\begin{figure}[ht]
\centering
\includegraphics[width=\textwidth]{../images/ProvidePositionReport.pdf}
\caption{Structure of component ProvidePositionReport}\label{fig:provideposrep}
\end{figure}

\subsubsection{Short Description of Functionality}
This function takes the current train position and generates a position report which is sent to the RBC. The point in time when such a report is sent is determined by events, on the one hand, and position report parameters---which are basically triggers---provided by the RBC or a balise group passed, on the other hand. The functionality is modeled using three operations, as shown in Figure~\ref{fig:provideposrep}, which are explained below.
\begin{description}
	\item[CalculateSafeTrainLength] Calculates the the safeTrainLength and the MinSafeRearEnd according to \cite[Chapter~3.6.5.2.4/5]{subset-026}. \\
\verb+safeTrainLength =  absolute(EstimatedFrontEndPosition - MinSafeRearEnd)+, where
\verb+MinSafeRearEnd = minSafeFrontEndPosition - L_TRAIN+.
	\item[EvaluateTriggerAndEvents] Returns a Boolean modelling whether the sending of the next position report is triggered or not. This value is the conjunction of the evaluation of all triggers (PositionReportParameters, i.e., Packet 58) and events (see \cite[Chapter~3.6.5.1.4]{subset-026}).
	\item[CollectData] This operation aggregates data of Packet 0, \dots, Packet 5 and the header to a position report.
\end{description}

\subsubsection{Reference to the SRS (or other requirements}
Most of the functionality is described in \cite[Chapter~3.6.5]{subset-026}.

\subsubsection{Design Constraints and Choices}
\begin{enumerate}
	\item The message length (i.e., attribute \verb+L_MESSAGE+) is by default set to 0; the actual value will be set by the Bitwalker/API.
	\item The attribute \verb+Q_SCALE+ is assumed to be constant; that is, all operations using this attribute do not convert between different values of that attribute.
	\item \textit{PositionReportHeader}: The time stamp (i.e., attribute \verb+T_TRAIN+) is not set; this should be done once the message is being sent by the API.
	\item \textit{Packet 4}: When aggregating data for this packet, an error message might be overwritten by a succeeding error message. Because the specification allows only to sent one error in one position report, errors are not being stored in a queue, for instance.
	\item \textit{Packet 44}: This packet is currently not contained in a position report as it is not part of the kernel functions.
	\item The usage of attributes \verb+D_CYCLOC+ and \verb+T_CYCLOC+ as part of the triggers specified by the position report parameters (i.e., Packet 58 sent by the RBC) may lead to unexpected results if a big clock cycle together with small values for the attributes is used. The cause is that at every clock cycle the current model increments the reference value for the distance and time by at most \verb+D_CYCLOC+ and \verb+T_CYCLOC+, respectively and not a factor of it.
\end{enumerate}

\subsubsection{Open Issues}
\begin{enumerate}
	\item The specification requires to store the last eight balise groups for which a position report has been sent (see \cite[Chapter~3.6.2.2.2.c]{subset-026}).
	\item For all reports that contain Packet 1 (i.e., report based on two balise groups), the RBC sends a coordinate system. It is unclear where this has to be stored (i.e., somehow the balise groups have to be stored in a database which has then to be updated), see \cite[Chapter~3.4.2.3.3.6]{subset-026}. Moreover, such a coordination system can be invalid and then has to be rejected (see \cite[Chapter~3.4.2.3.3.7-8]{subset-026}). On a more abstract level, we need to think about the interface between the RBC and the OBU or a proper abstraction thereof.
\end{enumerate}
 


\subsubsection{Macrofunction x in Manage Track Data}%MainfunctionManage Track Data.. Name should be be defined and substituded by the designer of the function. 
\paragraph{Reference to the SRS or other Requirements (or other requirements)}
\paragraph{Short descriptoiin of the functionality}
\paragraph{Interface}
\paragraph{Functional Design Description}
\paragraph{Refernce to the Scade Model}
\textbf{only in special case or link to the Scade model}

\subsubsection{Macrofunction x in Manage Track Data}%Mainfunction Manage Track Data. Name should be be defined and substituded by the designer of the function. 
\paragraph{Reference to the SRS or other Requirements (or other requirements)}
\paragraph{Short descriptoiin of the functionality}
\paragraph{Interface}
\paragraph{Functional Design Description}
\paragraph{Refernce to the Scade Model}
\textbf{only in special case or link to the Scade model}

\subsubsection{Macrofunction x in Manage Track Data}%Mainfunction Manage Track Data. Name should be be defined and substituded by the designer of the function. 
\paragraph{Reference to the SRS or other Requirements (or other requirements)}
\paragraph{Short descriptoiin of the functionality}
\paragraph{Interface}
\paragraph{Functional Design Description}
\paragraph{Refernce to the Scade Model}
\textbf{only in special case or link to the Scade model}

\subsubsection{Macrofunction x in Manage Track Data}%Mainfunction Manage Track Data. Name should be be defined and substituded by the designer of the function. 
\paragraph{Reference to the SRS or other Requirements (or other requirements)}
\paragraph{Short descriptoiin of the functionality}
\paragraph{Interface}
\paragraph{Functional Design Description}
\paragraph{Refernce to the Scade Model}
\textbf{only in special case or link to the Scade model}

%Mainfunction receive track data. Name should be be defined and substituded by the designer of the function. 

%-----------------------------------------------------------------------
\subsection{Manage Data}
%-----------------------------------------------------------------------
%\tbc
%Group2 (Bernd Hekele)

\subsubsection{functional block x in Manage Data}%Mainfunction Manage Data. Name should be be defined and substituded by the designer of the function. 
\paragraph{Reference to the SRS or other Requirements (or other requirements)}
\paragraph{Short descriptoiin of the functionality}
\paragraph{Interface}
\paragraph{Functional Design Description}
\paragraph{Refernce to the Scade Model}
\textbf{only in special case or link to the Scade model}

\subsubsection{functional block x in Manage Data}%Mainfunction Manage Data. Name should be be defined and substituded by the designer of the function. 
\paragraph{Reference to the SRS or other Requirements (or other requirements)}
\paragraph{Short descriptoiin of the functionality}
\paragraph{Interface}
\paragraph{Functional Design Description}
\paragraph{Refernce to the Scade Model}
\textbf{only in special case or link to the Scade model}

\subsubsection{functional block x in Manage Data}%Mainfunction Manage Data. Name should be be defined and substituded by the designer of the function. 
\paragraph{Reference to the SRS or other Requirements (or other requirements)}
\paragraph{Short descriptoiin of the functionality}
\paragraph{Interface}
\paragraph{Functional Design Description}
\paragraph{Refernce to the Scade Model}
\textbf{only in special case or link to the Scade model}

\subsubsection{functional block x in Manage Data}%Mainfunction Manage Data. Name should be be defined and substituded by the designer of the function. 
\paragraph{Reference to the SRS or other Requirements (or other requirements)}
\paragraph{Short descriptoiin of the functionality}
\paragraph{Interface}
\paragraph{Functional Design Description}
\paragraph{Refernce to the Scade Model}
\textbf{only in special case or link to the Scade model}

%Mainfunction receive track data. Name should be be defined and substituded by the designer of the function. 

%-----------------------------------------------------------------------
\subsection{Manage Outputs}
%-----------------------------------------------------------------------
%\tbc
%Bernd Hekele

\subsubsection{Macrofunction x in Manage Outputs}%Mainfunction Manage Outputs. Name should be be defined and substituded by the designer of the function. 
\paragraph{Reference to the SRS or other Requirements (or other requirements)}
\paragraph{Short descriptoiin of the functionality}
\paragraph{Interface}
\paragraph{Functional Design Description}
\paragraph{Refernce to the Scade Model}
\textbf{only in special case or link to the Scade model}

\subsubsection{Macrofunction x in Manage Outputs}%Mainfunction Manage Outputs. Name should be be defined and substituded by the designer of the function. 
\paragraph{Reference to the SRS or other Requirements (or other requirements)}
\paragraph{Short descriptoiin of the functionality}
\paragraph{Interface}
\paragraph{Functional Design Description}
\paragraph{Refernce to the Scade Model}
\textbf{only in special case or link to the Scade model}

\subsubsection{Macrofunction x in Manage Outputs}%Mainfunction Manage Outputs. Name should be be defined and substituded by the designer of the function. 
\paragraph{Reference to the SRS or other Requirements (or other requirements)}
\paragraph{Short descriptoiin of the functionality}
\paragraph{Interface}
\paragraph{Functional Design Description}
\paragraph{Refernce to the Scade Model}
\textbf{only in special case or link to the Scade model}

\subsubsection{Macrofunction x in Manage Outputs}%Mainfunction Manage Outputs. Name should be be defined and substituded by the designer of the function. 
\paragraph{Reference to the SRS or other Requirements (or other requirements)}
\paragraph{Short descriptoiin of the functionality}
\paragraph{Interface}
\paragraph{Functional Design Description}
\paragraph{Refernce to the Scade Model}
\textbf{only in special case or link to the Scade model}

%Mainfunction receive track data. Name should be be defined and substituded by the designer of the function. 

%-----------------------------------------------------------------------
\subsection{Mode and Level}
%-----------------------------------------------------------------------
%\tbc
%Marielle Petit
\subsubsection{Macrofunction x in Mode and Level}%Mainfunction receive track data. Name should be be defined and substituded by the designer of the function. 
\paragraph{Reference to the SRS or other Requirements (or other requirements)}
\paragraph{Short descriptoiin of the functionality}
\paragraph{Interface}
\paragraph{Functional Design Description}
\paragraph{Refernce to the Scade Model}
\textbf{only in special case or link to the Scade model}

\subsubsection{Macrofunction x in Mode and Level}%Mainfunction receive track data. Name should be be defined and substituded by the designer of the function. 
\paragraph{Reference to the SRS or other Requirements (or other requirements)}
\paragraph{Short descriptoiin of the functionality}
\paragraph{Interface}
\paragraph{Functional Design Description}
\paragraph{Refernce to the Scade Model}
\textbf{only in special case or link to the Scade model}

\subsubsection{Macrofunction x in Mode and Level}%Mainfunction receive track data. Name should be be defined and substituded by the designer of the function. 
\paragraph{Reference to the SRS or other Requirements (or other requirements)}
\paragraph{Short descriptoiin of the functionality}
\paragraph{Interface}
\paragraph{Functional Design Description}
\paragraph{Refernce to the Scade Model}
\textbf{only in special case or link to the Scade model}

\subsubsection{Macrofunction x in Mode and Level}%Mainfunction receive track data. Name should be be defined and substituded by the designer of the function. 
\paragraph{Reference to the SRS or other Requirements (or other requirements)}
\paragraph{Short descriptoiin of the functionality}
\paragraph{Interface}
\paragraph{Functional Design Description}
\paragraph{Refernce to the Scade Model}
\textbf{only in special case or link to the Scade model}

%Mainfunction receive track data. Name should be be defined and substituded by the designer of the function. 


%-----------------------------------------------------------------------
\subsection{Manage Radio Communication}
%-----------------------------------------------------------------------
%\tbc
%Group4 (Uwe Steinke)

\subsubsection{Management of Radio Communication (\emph{MoRC})} 

\paragraph{Reference to the SRS}
The management of radio communication is specified in Subset-026, chap. 3.5. 

\paragraph{Short description of the functionality}

The management of radio communication \textit{MoRC} implements the on board management part of a single communication session with the track, i.e. a single RBC. It controls the establishing, maintaining and termination process of a radio communication session and steers the underlying communication safety layer and the mobile device. Those and the data transfer itself are not part of the function. 

\paragraph{Interface}

\subparagraph{Inputs}
The MoRC function takes as inputs datagrams received from track, OBU internal phases and status information and configuration data: 

\begin{itemize}
 \item Datagrams received from track (\textit{inMessage}):
 \begin{itemize}
  \item Packet 42 (session management) received from balise group or RBC
  \item Packet 45 (radio network registration) received from balise group or RBC
  \item Message 32 (RBC/RIU System Version) received from RBC: \textit{MoRC} only needs to know if the system version received from track side is supported by the OBU.
  \item Message 38 (initiation of a communication session) received from RBC
  \item Message 39 (acknowledgement of termination of a communication session)
 \end{itemize}

 \item \textit{obuEventsAndPhases}: information about OBU internal events and OBU internal phases:

  \begin{itemize}
   \item \textit{atPowerDown}
   \item \textit{atPowerUp}
   \item \textit{atStartOfMission}
   \item \textit{startOfMissionProcedureIsGoingOn}
   \item \textit{startOfMissionProcedureCompleted}
   \item \textit{trainIsRejectedByRBC\_duringStartOfMission}
   \item \textit{endOfMissionIsExecuted}
   \item \textit{driverClosesTheDeskduringStartOfMission}
   \item \textit{driverHasManuallyChangedLevel}
   \item \textit{afterDriverEntryOfANewRadioNetworkID}
   \item \textit{triggerDecisionThatNoRadioNetworkIDAvailable}
   \item \textit{isPartOfAnOngoingStartOfMissionProcedure}
   \item \textit{trainPassesALevelTransitionBorder}
   \item \textit{trainPassesA\_RBC\_RBC\_border\_WithItsFrontEnd}
   \item \textit{trainExitedFromAnRBCArea}
   \item \textit{modeChangeHasToBeReportedToRBC}
   \item \textit{trainFrontInsideInAnAnnouncedRadioHole}
   \item \textit{trainFrontReachesEndOfAnnouncedRadioHole}
   \item \textit{OBU\_hasToEstablishANewSession}
   \item \textit{isInCommunicationSessionWithAnRIU}
   \item \textit{errorConditionRequiringTerminationDetected}
  \end{itemize}

  \item Current OBU internal states:
  
  \begin{itemize}
   \item \textit{currentTime}: current OBU system time
   \item \textit{t\_train}: current trainborne clock (T\_TRAIN) as specified in Subset-026, chap. 7
   \item \textit{mode}: current OBU mode
   \item \textit{level}: current OBU level
  \end{itemize}

  \item \textit{statusOfMobile}: status of the associated mobile device
  
  \item configuration parameters:
  
   \begin{itemize}
    \item \textit{onboardPhoneNumbers} (NID\_RADIO)
    \item \textit{radioNetworkIDs}: Identities of radio networks (NID\_MN): default, memorized or from driver  
    \item \textit{nid\_engine}: Onboard ETCS identity (NID\_ENGINE)
    \item \textit{connectionStatusTimerInterval}: Connection status timer period 
   \end{itemize} 

 \end{itemize} 

\subparagraph{Outputs}

MoRC generates a couple of outputs: 

\begin{itemize}
 \item \textit{MessageToRBC}: messages to be sent to the RBC: 

  \begin{itemize}
   \item Message 155 (initiation of a communication session)
   \item Message 156 (termination of a communication session)
   \item Message 159 (session established)
   \item Message 154 (no compatible version supported) 
  \end{itemize}

 \item Action triggers: 
 
  \begin{itemize}
   \item \textit{sendAPositionReport}: triggers a position report to be sent to the RBC
   \item \textit{memorizeTheLastRadioNetworkID}: triggers to store the last radio network ID for later use
   \item \textit{orderTheRegistrationOfItsConnectedMobiles}
   \item \textit{rejectOrderToContactRBC\_or\_RIU}
   \item \textit{InformTheDriverThatNoConnectionWasSetup}
   \item \textit{requestTheSetupOfASafeRadioConnection}: initiate the setup of a safe radio connection
   \item \textit{requestReleaseOfSafeRadioConnectionWithTrackside}: initiate the release of a safe radio connection
   \item \textit{ignoreMessagesFromRBC\_except\_m39\_AckOfTerminationOfCommunicationSession}
   \item \textit{sessionSuccessfullyEstablished}
  \end{itemize} 

 \item \textit{cmdsToMobile}: control commands to the mobile device
 
 \item Status information: 
 
  \begin{itemize}
   \item \textit{sessionStatus}: current session status
   \item \textit{mobileSWStatus}: connection status
   \item \textit{currentRadioNetworkID}: current radio network ID
  \end{itemize}
 
\end{itemize}

\begin{figure}
\centering
\includegraphics[scale=0.2]{../images/MoRC_Main.png}
\caption{Main function of \textit{MoRC}.}
\end{figure}

\begin{figure}
\centering
\includegraphics[width=\textwidth]{../images/MoRC_SessionManagement.png}
\caption{Implementation of session states.}
\end{figure}


\paragraph{Functional Design Description}

The kernel function of the \textit{MoRC} component is \emph{managementOfRadioCommunication} (figure ???). The implementation is kept close to the prose of Subset-026, chap. 3.5. Since chap. 3.5 rarely refers to terms, variable types, packets and messages of the ETCS language as specified in Subset-026, chap. 7 and 8, \emph{managementOfRadioCommunication} does neither. 

To be capable of being integrated with other OBU software components, \emph{MoRC} had to be wrapped with a transformer between the ETCS and the "chap. 3.5" language. This is the purpose of the main function of \emph{MoRC}, \emph{MoRC\_Main}. 



The function \emph{managementOfRadioCommunication} implements the session states establishing, maintaining and termination as described in Subset-026, chap. 3.5. A SCADE state machine reflects this state model (figure ???) accurately. Within each of the states, the activities needed as long as the state is active, are performed. When there is no communication session (state \emph{NoSession}) currently, the state machine waits for events that initiate a session (subfunction \emph{initiate\_a\_Session}). When the appropriate conditions are fulfilled, the state machine moves to the \textit{Establishing} state. Here in, it runs through the sequence required fore establishing a session (subfunction \emph{establish\_a\_Session}. Dependent on the results, the state machine changes over to the \emph{Maintaining} or \emph{Terminating} state. While in \emph{Maintaining}, the communication connection is monitored. When an event triggering the session termination occurs, the state machine switches to the state \emph{Terminating} with the subfunction \emph{terminating\_a\_CommunicationSession} and performs the session termination sequence. 

In parallel to the main state machine, \emph{managementOfRadioCommunication} monitors all the time whether the session has to be terminated (subfunction \emph{initiateTerminatingASession}) or if the session has the be terminated and subsequently established (subfunction \emph{terminateAndEstablishSession}). \emph{registeringToTheRadioNetwork} is responsible for connection to the radio network. \emph{safeRadioConnectionIndication} controls the radio connection indication for the driver.


\paragraph{Reference to the Scade Model}

The MoRC SCADE model resides at \url{https://github.com/openETCS/modeling/tree/master/model/Scade/System/ObuFunctions/Radio/MoRC} .






%Mainfunction receive track data. Name should be be defined and substituded by the designer of the function. 

\input{sections/managedmiprocedure.tex}%Mainfunction receive track data. Name should be be defined and substituded by the designer of the function. 

\section{F3: Measure Train Movement}

\section{F4: Manage Radio Communication}

\section{F5: Manage JRU}

\section{F6: DMI Controller}
  %-----------------------------------------------------------------------
  \subsection{DMI Controller}
  %-----------------------------------------------------------------------
  %\tbc
  %Valerio D´Angelo/Baseliyos Jacob

  \paragraph{Reference to the SRS or other Requirements (or other requirements)}

   ERA\_ERTMS\_015560

  \paragraph{Short description of the functionality}
  The DMI controller interact with the DMI display and is responsible for alls procedures between the DMI display and Driver. Furthermore, the DMI controller will interact with the DMI Management to compute the received information (e.g. driver number request, ...) and send, if necessary, data or reports to the DMI Management (acknowledge, text messages...). The DMI Controller is a passive module, this means that all the processing are performed EVC-side, therefore the DMI Controller simply responds to the requests of the EVC or Driver and performs some checks according with the information received from EVC.\\

  \begin{figure}[hbtp]
  \centering
  \includegraphics[scale=0.5]{images/DMI_Interfaces}
  \caption{DMI Interfaces}
  \end{figure}

  \paragraph{Interface}
  The DMI Controller has two interfaces. One between DMI Controller and DMI Display and one between DMI Controller and DMI Management. 
  The structure of the interface between DMI Controller and DMI Display is driven by the logic of SCADE Display therefore It doesn't follow any standard or constraints (It will not be described in this chapter).
  DMI Controller and DMI Management exchange packets. Each packet is a structured type with a valid flag (a boolean variable), the DMI controller takes into account the data inside the packet only when the valid flag is true.\\
  The interface between DMI Controller and DMI Management consist of three parts according with the direction of the information:
  
  \begin{itemize}
  \item From DMI Management to DMI Controller
  \item From DMI Controller to DMI Management 
  \item Both ways directions (You will find the same type both as input than as output)
  \end{itemize}
  


  \subparagraph{From DMI Management to DMI Controller}

  In the following table are listed the inputs coming from DMI Management with a brief description:\\
  %\begin{table}[H]
    \begin{tabular}{| c | l | l | l | l |}
      \hline
      \textbf{NAME} & \textbf{DESCRIPTION} \\ \hline
      \texttt{DMI\_entry\_request} & Request to input data (e.g. driver id, Train running number etc.)\\
      \texttt{DMI\_identifier\_request} & Request of the DMI informations\\
      \texttt{DMI\_menu\_request} & Request to enable or disable buttons\\
      \texttt{DMI\_dynamic} & Contains informations about current speed, current mode etc.\\
      \texttt{DMI\_text\_message} & Contains predefined or plain text messages\\
      \texttt{DMI\_icons} & Request to display one or more icons in any area\\

      \hline
    \end{tabular} 
  % \caption{Overview of input}
    \label{tbl:DMICtrToDMIMng}
  %\end{table}
  
      Please note: TIU\_trainStatus input is missing in the above table. This is the only input coming directly from TIU and contains the open/close Desk signal. 
    
  \subparagraph{From DMI Controller to DMI Management }
  In the following table are listed the outputs directed to DMI Management with a brief description:\\
  %\begin{table}[H]
    \begin{tabular}{| c | l | l | l | l |}
      \hline
      \textbf{NAME} & \textbf{DESCRIPTION} \\ \hline
      \texttt{DMI\_identifier} & Information about DMI (e.g. version, cabin identifier etc.)\\
      \texttt{DMI\_driver\_request} & Driver request or acknowledgement\\
      \texttt{DMI\_train\_data\_ack} & Train data acknowledgement\\
      \texttt{DMI\_status\_report} & The actual status of DMI (keep alive)\\
      \texttt{DMI\_text\_message\_ack} & Text message acknowledgement\\
      \texttt{DMI\_icons\_ack} & Icon acknowledgement\\

      \hline
    \end{tabular} 
  % \caption{Overview of input}
    \label{tbl:DMICtrToDMIMng}
  %\end{table}
    
  \subparagraph{Both ways direction }
  In the following table are listed the outputs/inputs  to/from DMI Management with a brief description:\\
  
  %\begin{table}[H]
    \begin{tabular}{| c | l | l | l | l |}
      \hline
      \textbf{NAME} & \textbf{DESCRIPTION} \\ \hline
      \texttt{DMI\_driver\_identifier} & Contains the default or entered driver identifier\\
      \texttt{DMI\_train\_running\_number} & Contains the default or entered train running number\\
      \texttt{DMI\_train\_data} & Contains the default or entered train data\\
      \hline
    \end{tabular} 
  % \caption{Overview of input}
    \label{tbl:DMICtrToDMIMng}
  %\end{table}
  
  

    
  \paragraph{Functional Design Description}
  \textbf{Please note}: \textit{DMI Controller is a project under construction, a lot of features and functionalities are missing, therefore the structure described below is a draft version and will be changing in the future.}\\
  The informations (received and sent) could be divided in two groups: Sporadic and Periodic. The first one are received/sent aperiodically in any time instead the second one are received/sent periodically, with a fixed deadline. Are part of Periodic group the output DM\_status\_report and the input DMI\_dynamic all other are Sporadic. Therefore, the structure of DMI Controller module consists of a first main state machine \textit{CabinSM} (Fig. \ref{fig:CabinSM}) triggered by  a \textit{OpenDesk} signal (from TIU). Inside the \textit{DeskIsOpen} state there are other two state machines :\textit{HandshakeSM} and \textit{DynamicInfoSM} (Fig. \ref{fig:DynSporSM}).\\
  HandshakeSM performs an initial handshake between DMI Controller and DMI Management. Before that, no data has to be sent or received to/from DMI Management. When the transition is fired a DMI\_identifier packet is sent to DMI Management with informations about the DMI (e.g. DMI identifier, DMI name etc.). At this point the DMI Controler is ready to manage the sporadic information (e.g. Enter or revalidate DriverID, Enter or revalidate Train running number etc.). \\
  The DynamicInfoSM state machine is triggered after the handshake, exactly when  HandshakeSM reaches the  DynInfo\_Activated state. At the time when the transition is fired a signal is emitted (startDMI\_status) and begins a periodic sending of DMI status information (keep alive) to DMI Management. Once reached DynamicInfo\_Active, the DMI Controller is ready to receive and manage the dynamic informations.\\
  
  \clearpage
   \begin{figure}[hbtp]
      	\centering
      	\includegraphics[scale=0.7]{images/CabinSM}
      	\caption{Cabin State Machine}
      	\label{fig:CabinSM}
   \end{figure}
      
  \begin{figure}[hbtp]
  \centering
  \includegraphics[scale=0.8]{images/DynSporInfoSM}
  \caption{HandshakeSM and DynamicInfoSM State Machines}
  \label{fig:DynSporSM}
  \end{figure}
  


  With the aim to improve the readability and for a better management of complexity, all the functions (modules, state machines etc.) implemented in each state are divided several diagrams.\\
  
  The \textit{SporadicInfo} consist of:
  \begin{itemize}
  	\item \textbf{diagram\_SporadicInfo\_Main}: Contains all the modules to manage the sporadic data like ``Enter revalidate Driver ID'', ``Enter or revalidate train running number'', enable buttons in menus. The WindowSM state machine manages the windows that should appear on the DMI(Fig. \ref{fig:win_sm}).
  	
  	\item \textbf{ diagram\_SporadicInfo\_TrainData}: Contains all the logic to store and adapt the incoming train data to a correct visualization on DMI Display.
  	
  	\item \textbf{diagram\_SporadicInfo\_Icon\_Management}: Contains the logic to show/hide one or several icons in area and manage the acknowledgement mechanism if It's required.
  	
  	\item \textbf{ diagram\_SporadicInfo\_DriverID\_TRN}: Contains the logic to store and sent the Train running number and the Driver ID.
  	
  	\item \textbf{diagram\_SporadicInfo\_Text\_Messages}: Contains the modules, state machines and all the logic to manage and display predefined and customized text messages.
  \end{itemize}
  
  The \textit{ DynamicInfo\_Active} state consists of:\\
  
  \begin{itemize}
  	\item \textbf{diagram\_DynamicInfo\_Main}: Contains modules to store and display the informations like the current mode, ETCS level, RBC connection status and location brake target.
  	\item \textbf{diagram\_SpeedSupervision}: Contains the module where are implemented the behaviour of the speed pointer and the circular speed gauge ( informations about speed target, speed permitted and speed release).
  \end{itemize}
  
  \begin{figure}[hbtp]
  	\centering
  	\includegraphics[scale=0.45]{images/WindowSM}
  	\caption{Windows state machine}\label{fig:win_sm}
  \end{figure}
  
  \paragraph{ Communication Protocol}
This section explains which messages are exchanged among DMI Controller, DMI Management and Start of mission procedure. As mentioned previously the DMI Controller is a passive component, It simply responds to requests, therefore is able to cover different scenarios. Below are some examples.
  
\clearpage
  \subparagraph{ Start Of Mission scenario} Are detailed, through a sequence diagram, all the activities (exchanged messages) that should be done to start. In this scenario we have three actors: DMI Controller, DMI Management and SoM procedure (the module where is implemented the start of mission procedure). It's assumed that a OpenDesk signal is received and the system starts in Stand By mode (Fig. \ref{fig:SeqDiaSoM}).
 
  \begin{figure}[hbtp]
  \centering
  \includegraphics[scale=0.45]{images/SeqDia_DMIctr_DMImng_SoMproc}
  \caption{Sequence Diagram of start of mission scenario}\label{fig:SeqDiaSoM}
  \end{figure}
  
  \clearpage
  \subparagraph{ Cyclic Exchange of messages}
The time between two messages has not yet been definitively established, It might change in the future. The DMI status packet implements a keep alive mechanism, this means, if the EVC does not receive any DMI status signal during the lapse time, It shall consider a failure in DMI. This check is not yet implemented.

    \begin{figure}[hbtp]
      \centering
      \includegraphics[scale=0.5]{images/DynamicPacket_SeqDia}
      \caption{ Sequence diagram of Dynamic data}\label{fig:SeqDiaDyn}
    \end{figure}
  
    \begin{figure}[hbtp]
      \centering
      \includegraphics[scale=0.5]{images/DMIStatus_SeqDia}
      \caption{Sequence Diagram of DMI status}\label{fig:SeqDiaStatus}
     \end{figure}
     
     
  

  \paragraph{Reference to the Scade Model}

    The SCADE model can be found on github under the following path:

    \tiny\url{https://github.com/openETCS/modeling/tree/master/model/Scade/System/DMI_Control}



\bibliographystyle{unsrt}
\bibliography{architecture}


\newpage
\addcontentsline{toc}{chapter}{Index}
\printindex
%===================================================
%Do NOT change anything below this line

\end{document}
